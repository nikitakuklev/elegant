The input file for this element uses a simple text format.  It is nearly identical
to the output in the {\tt printout} file generated by the {\tt matrix\_output}
and {\tt analyze\_map} commands.  For example, for a 1st-order matrix, the file would have the
following appearance:\\
C: {\em C1 C2 C3 C4 C5 C6}\\
R1: {\em R11 R12 R13 R14 R15 R16}\\
R2: {\em R21 R22 R23 R24 R25 R26}\\
R3: {\em R31 R32 R33 R34 R35 R36}\\
R4: {\em R41 R42 R43 R44 R45 R46}\\
R5: {\em R51 R52 R53 R54 R55 R56}\\
R6: {\em R61 R62 R63 R64 R65 R66}\\

Items in normal type must be entered exactly as shown, whereas those in
italics must be provided by the user.  The colons are important!
For this particular example, one would set {\tt ORDER=1} in the {\tt MATR}
definition.  Typically, the {\em Ci} are zero, except for {\em C5}, which
is usually equal to the length of the element (which must be specified with
the {\tt L} parameter in the {\tt MATR} definition).

As of release 2019.2, the required format changed slightly. 
In the new version, the start of the matrix is determined by reading through the file until 
a line starting with \verb|C:| is found.
In the past, 
instead of starting with \verb|C:|, the first line of the matrix could start
with any string terminated by a colon, but that line had to be the first line in the
file, which conflicted with the format emitted by \verb|analyze_map|.
