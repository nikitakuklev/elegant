This element provides a symplectic bending magnet with the exact
Hamiltonian.  For example, it retains all orders in the momentum offset
and curvature.  The field expansion is available to eighth order.

One pitfall of symplectic integration is the possibility of orbit and
path-length errors for the reference orbit if too few kicks are used.
This may be an issue for rings.  Hence, one must verify that a
sufficient number of kicks are being used by looking at the trajectory
closure and length of an on-axis particle by tracking.  Using 
{\tt INTEGRATION\_ORDER=4} is recommended to reduce the number of
required kicks.

As of version 28.0 and later, the \verb|REFERENCE_CORRECTION| feature is available
to compensate for errors inherent in the numerical integration of the trajectories.
In particular, depending on the number of kicks used, as well as the bending radius and angle,
an on-axis particle may emerge from the element with a non-zero trajectory and a path-length error.
With \verb|REFERENCE_CORRECTION| set to a non-zero value, these errors are subtracted from the
coordinates of all particles.
There are some pitfalls to using this feature: first, one may not realize that the number of kicks is too small to
provide good results, since the output trajectory of the central particle will always be
(nearly) identically zero. Second, in a magnet with a gradient or other field nonuniformities, 
a particle may emerge centered on the ideal trajectory yet still see the impact of the gradient, sextupole, etc.
For these reasons, this feature should be used with caution and only when the residual trajectory is 
large enough to cause problems.

{\bf Higher-order field components}

Normally, one specifies the higher-order components of the field with
the {\tt K{\em n}}, with $n=1$ through $8$. The field
expansion in the midplane is $B_y(x) = B_o * (1 +
\sum_{n=1}^8\frac{K_n\rho_o}{n!}x^n)$.  By setting the {\tt USE\_bN}
flag to a nonzero value, one may instead specify the {\tt b{\em n}}
parameters, which are defined by the expansion $B_y(x) = B_o
* (1 + \sum_{n=1}^8\frac{b_n}{n!}x^n)$.  This is convenient if one is
varying the dipole radius but wants to work in terms of constant field
quality.  

Setting {\tt NONLINEAR=0} turns off all the terms above {\tt K\_1} (or {\tt b\_1}) and
also turns off effects due to curvature that would
normally result in a gradient producing terms of higher order.

The \verb|EXPANSION_ORDER| parameter controls the order of the expansion of the nonlinear
fields, so that terms are limited to $x^i y^j$ with $i+j\leq$\verb|EXPANSION_ORDER|.
By default, when \verb|EXPANSION\_ORDER=0|, the expansion order is set automatically, as follows:
If the highest non-zero multipole order (specified by \verb|Kn|, \verb|Bn|, \verb|Fn|, or \verb|Gn|)
is $n$ (with $n=1$ being quadruople), then the expansion order is set to $n+3$.
However, the expansion order is never automatically set to less than 4, unless all the 
multipole terms are zero, in which case the expansion always yields a constant.
Since the number of polynomial terms increases like the square of the expansion order, using many
multipole terms can significantly increase run time.
The maximum value for the expansion order is 10.

{\bf Edge angles and edge effects} 

Some confusion may exist about the edge angles, particularly the signs.
For a sector magnet, we have of course \verb|E1=E2=0|.  For a symmetric rectangular
magnet, \verb|E1=E2=ANGLE/2|.  If \verb|ANGLE| is negative, then so are
\verb|E1| and \verb|E2|.  To understand this, imagine a rectangular magnet with positive \verb|ANGLE|.
If the magnet is flipped over, then \verb|ANGLE| becomes negative, as does the bending
radius $\rho$.    Hence, to keep the focal length
of the edge $1/f = -\tan E_i /\rho$ constant, we must also change the sign of
$E_i$.

Several models are available for edge (or fringe) effects. Which is used depends on the
settings of the \verb|EDGE_ORDER|, \verb|EDGE1_EFFECTS|, and \verb|EDGE2_EFFECTS| parameters
\verb|EDGE1_EFFECTS| controls entrance edge effects while \verb|EDGE2_EFFECTS| controls exit edge effects,
as follows:
\begin{itemize}
  \item 1: --- Edge effects using non-symplectic method \cite{KLBrown}. {\bf Not recommended.}
    \begin{itemize}
      \item \verb|EDGE_ORDER<2| --- linear edge focusing with $\delta$-dependence to all orders.
      \item \verb|EDGE_ORDER>=2| --- second-order matrix edge focusing with $\delta$-dependence to all orders.
    \end{itemize}              
  \item 2: --- Edge effects using K. Hwang's symplectic method \cite{KHwang}. Note that there will be a
  trajectory offset when using this method that is particularly evident for small bending radii. Adjustment of the length and FSE 
  parameter can be used to suppress this offset, which results from the extension of the fringe fields beyond the nominal
  boundaries of the magnet. One can also set \verb|REFERENCE_CORRECTION=1|, but this just sweeps the
  effect under the rug and is not recommended.
    \begin{itemize}
      \item \verb|EDGE_ORDER<2| --- include only terms linear in transverse coordinates, but $\delta$-dependence to all orders.
      {\bf Not recommended.}
      \item \verb|EDGE_ORDER>=2| --- include all terms. {\bf Recommended.}
    \end{itemize}              
  \item Other: --- No edge effects.
\end{itemize}

{\bf Radiation effects}

Incoherent synchrotron radiation, when requested with {\tt ISR=1},
normally uses gaussian distributions for the excitation of the electrons.
Setting {\tt USE\_RAD\_DIST=1} invokes a more sophisticated algorithm that
uses correct statistics for the photon energy and number distributions.
In addition, if {\tt USE\_RAD\_DIST=1} one may also set {\tt ADD\_OPENING\_ANGLE=1},
which includes the photon angular distribution when computing the effect on 
the emitting electron.  

{\bf Adding errors}

When adding errors, care should be taken to choose the right
parameters.  The \verb|FSE| and \verb|ETILT| parameters are used for
assigning errors to the strength and alignment relative to the ideal
values given by \verb|ANGLE| and \verb|TILT|.  One can also assign 
errors to \verb|ANGLE| and \verb|TILT|, but this has a different meaning:
in this case, one is assigning errors to the survey itself.  The reference
beam path changes, so there is no orbit/trajectory error. The most common
thing is to assign errors to \verb|FSE| and \verb|ETILT|.  Note that when
adding errors to \verb|FSE|, the error is assumed to come from the power
supply, which means that multipole strengths also change.

{\bf Splitting dipoles}

When dipoles are long, it is
common to want to split them into several pieces, to get a better look
at the interior optics.  When doing this, care must be exercised not
to change the optics.  {\tt elegant} has some special features that
are designed to reduce or manage potential problems. At issue is the
need to turn off edge effects between the portions of the same dipole.

First, one can simply use the \verb|divide_elements| command to set up
the splitting.  Using this command, {\tt elegant} takes care of everything.

Second, one can use a series of dipoles {\em with the same name}.  In this case,
elegant automatically turns off interior edge effects.  This is true when the
dipole elements directly follow one another or are separated by a MARK element.

Third, one can use a series of dipoles with different names.  In this case, one
must also use the \verb|EDGE1_EFFECTS| and \verb|EDGE2_EFFECTS| parameters to
turn off interior edge effects.  

