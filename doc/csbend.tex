This element provides a symplectic bending magnet with the exact
Hamiltonian.  For example, it retains all orders in the momentum offset
and curvature.  The field expansion is available to fourth order.

One pitfall of symplectic integration is the possibility of orbit and
path-length errors for the reference orbit if too few kicks are used.
This may be an issue for rings.  Hence, one must verify that a
sufficient number of kicks are being used by looking at the trajectory
closure and length of an on-axis particle by tracking.  Using 
{\tt INTEGRATION\_ORDER=4} is recommended to reduce the number of
required kicks.

Normally, one specifies the higher-order components of the field with
the {\tt K1}, {\tt K2}, {\tt K3}, and {\tt K4} parameters. The field
expansion in the midplane is $B_y(x) = B_o * (1 +
\sum_{n=1}^4\frac{K_n\rho_o}{n!}x^n)$.  By setting the {\tt USE\_bN}
flag to a nonzero value, one may instead specify the {\tt b1} through
{\tt b4} parameters, which are defined by the expansion $B_y(x) = B_o
* (1 + \sum_{n=1}^4\frac{b_n}{n!}x^n)$.  This is convenient if one is
varying the dipole radius but wants to work in terms of constant field
quality.  

Setting {\tt NONLINEAR=0} turns off all the terms above {\tt K\_1} (or {\tt b\_1}) and
also turns off effects due to curvature that would
normally result in a gradient producing terms of higher order.

Edge effects are included using a first- or second-order matrix.  The
order is controlled using the {\tt EDGE\_ORDER} parameter, which has a
default value of 1.  N.B.: if you choose the second-order matrix, it
is not symplectic.

Incoherent synchrotron radiation, when requested with {\tt ISR=1},
normally uses gaussian distributions for the excitation of the electrons.
Setting {\tt USE\_RAD\_DIST=1} invokes a more sophisticated algorithm that
uses correct statistics for the photon energy and number distributions.
In addition, if {\tt USE\_RAD\_DIST=1} one may also set {\tt ADD\_OPENING\_ANGLE=1},
which includes the photon angular distribution when computing the effect on 
the emitting electron.  

{\em Special note about splitting dipoles}: when dipoles are long, it is
common to want to split them into several pieces, to get a better look
at the interior optics.  When doing this, care must be exercised not
to change the optics.  {\tt elegant} has some special features that
are designed to reduce or manage potential problems. At issue is the
need to turn off edge effects between the portions of the same dipole.

First, one can simply use the \verb|divide_elements| command to set up
the splitting.  Using this command, {\tt elegant} takes care of everything.

Second, one can use a series of dipoles {\em with the same name}.  In this case,
elegant automatically turns off interior edge effects.  This is true when the
dipole elements directly follow one another or are separated by a MARK element.

Third, one can use a series of dipoles with different names.  In this case, you
must also use the \verb|EDGE1_EFFECTS| and \verb|EDGE2_EFFECTS| parameters to
turn off interior edge effects.  
