There are three modes for implementing alignment errors. Which is used
is controlled by the value of the \verb|MALIGN_METHOD| parameter:
\begin{itemize}
\item \verb|MALIGN_METHOD=0| --- This selects the original method, which was
  the only one available before version 2021.1. The misalignment is
  referenced to the entrance face. The \verb|YAW| and \verb|PITCH| parameters
  are ignored.
\item \verb|MALIGN_METHOD=1| --- This selects a method based on M. Venturini's
  work \cite{Venturini2021}, with misalignment referenced to the entrance face.
  The \verb|YAW| and \verb|PITCH| parameters are implemented.
\item \verb|MALIGN_METHOD=2| --- This selects a method based on M. Venturini's
  work \cite{Venturini2021}, with misalignment referenced to the magnet center.
  The \verb|YAW| and \verb|PITCH| parameters are implemented.
\end{itemize}

For elements with non-zero \verb|TILT|,  it is important to note that for \verb|MALIGN_METHOD=0|
displacements are performed in the lab frame, while for other methods the displacements are
performed in the tilted magnet frame.
