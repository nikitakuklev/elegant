This element simulates a planar undulator, together with an optional
co-propagating laser beam that can be used as a beam heater or
modulator.  The simulation is done by numerical integration of the
Lorentz equation.  It is not symplectic, and hence this element is not
recommended for long-term tracking simulation of undulators in storage
rings.  

The fields in the undulator can be expressed in one of three ways.
The FIELD\_EXPANSION parameter is used to control which method is used.
\begin{itemize}
\item The exact field, given by (see section 3.1.5 of the {\em Handbook of
Accelerator Physics and Engineering})
\begin{equation}
B_x = 0,
\end{equation}
\begin{equation}
B_y = B_0 \cosh k_u y \cos k_u z,
\end{equation}
and
\begin{equation}
B_z = B_0 \sinh k_u y \cos k_u z ,
\end{equation}
where $k_u = 2\pi/\lambda_u$ and $\lambda_u$ is the undulator period.
This is the most precise method, but also the slowest.  

\item The field expanded to leading order in $y$:
\begin{equation}
B_y = B_0 ( 1 + \frac{1}{2}(k_u y)^2 ) \cos k_u z,
\end{equation}
and
\begin{equation}
B_z = B_0 k_u y \cos k_u z.
\end{equation}
In most cases, this gives results that are very close to the exact fields,
at a savings of 10\% in computation time.

\item The ``ideal'' field:
\begin{equation}
B_y = B_0 \cos k_u z,
\end{equation}
\begin{equation}
B_z = 0.
\end{equation}
This is about 10\% faster than the leading-order mode, but less
precise.  Small differences from results with the exact field may be
seen.  Generally, these are too small to be a concern.  As a result,
this is the default mode.

By default, if the laser wavelength is not given, it is computed from the resonance
condition:
\begin{equation}
\lambda_l = \frac{\lambda_u}{2 \gamma^2} \left( 1 + \frac{1}{2} K^2 \right),
\end{equation}
where $\gamma$ is the relativistic factor for the beam and $K$ is the
undulator parameter.

The adaptive integrator doesn't work well for this element, probably
due to sudden changes in field derivatives in the first and last three
poles (a result of the implementation of the undulator terminations).
Hence, the default integrator is non-adaptive Runge-Kutta.  The
integration accuracy is controlled via the N\_STEPS parameter.
N\_STEPS should be about 100 times the number of undulator periods.

