NB: This element is new and considered experimental. Please report issues back to the developers.

This element provides serial or parallel simulation of the interaction of residual gas ions
with the electron beam.
It must be used in concert with the \verb|ion_effects| command, described in \ref{subsec:ioneffects}.

Modeling of residual ions has these features:
\begin{itemize}
\item s-dependent gas pressure profiles for any number of species.
\item Arbitrary ion species, specified by a user-provided file that includes the cross sections.
\item User-defined locations for ion generation. Each \verb|IONEFFECTS| element represents the
      ions present in a segment of the accelerator. The segments start and end half way between
      successive \verb|IONEFFECTS| elements.
\item Arbitrary fill patterns. Uniform fills can be set up using the \verb|bunched_beam| command,
      while custom fills can be set up by generating the beam externally and using the \verb|sdds_beam| command.
\item Multiple ionization of trapped ions.  For example, a CO+ ion could multiply ionize into CO++, or dissociate into C+.
\end{itemize}

Some limitations of the model include:
\begin{itemize}
\item Fields from electron bunches are computed based on gaussian parameters, which is a reasonably good
  approximation.
\item By default, fields from ions are computed based on gaussian parameters, which is often a somewhat
  poor approximation. Alternatively, a bi-gaussian form may be used, which uses a sum of two gaussians.
  This is a much better approximation to the typical distribution, which often has a hot core and long tails.
\item Ions move only transversely and exist only outside of magnets.
\end{itemize}

Performing ion simulations involves the following steps
\begin{enumerate}
\item Prepare file describing the ion properties, as described in \ref{subsec:ioneffects}.
  Each ion is generated by either a source gas or source ion.
\item Prepare file giving gas pressure vs s for the source gases described in the ion 
  properties file.
\item Insert \verb|IONEFFECTS| elements in the lattice. This can be performed using the
  \verb|insert_elements| command (described in \ref{subsec:insertelements}), or
  manually by editing the lattice file.
\item Insert \verb|ion_effects| command after the \verb|run_setup| command. See
  \ref{subsec:ioneffects} for syntax.
  Note that certain properties of the individual \verb|IONEFFECTS| elements can override the
  global settings given by in the \verb|ion_effects| command.
\item Generate a bunched beam, using either the \verb|bunched_beam| command or providing
  an externally-generated beam to the \verb|sdds_beam| command. Section \ref{sect:bunchedBeams}
  gives more information about bunched beams in \verb|elegant|.
\end{enumerate}


%For every turn, for every ion element, and for every bunch, the \verb|IONEFFECTS| element does the following:

For each bunch passage, the \verb|IONEFFECTS| element does the following:

\begin{enumerate}
\item Advance existing ions during bunch gap
\item Eliminate ions that are outside of given boundaries
\item Generate ions 
\item Apply kick from beam to ions 
\item Apply kick from ions to beam
\end{enumerate}

%\paragraph{Ion generation}

The line density of ions generated by a single bunch in a single pass is:
\begin{equation}
\lambda_{ion} = \sigma_{ion} \frac{P}{k_B T} N_b
\end{equation}
where $\sigma_{ion}$ is the ionization cross section, $P$ is the pressure, $k_B$ is the Boltzmann constant, $T$ is the temperature, and $N_b$ is the bunch population.

The resulting macroparticle charge is:
\begin{equation}
%Q_{macro} = 3.21 \sigma_{ion} P Q_{bunch} L_{eff} / n_{macro}
Q_{macro} = \frac{10^{-22} e}{7.5\times10^{-3} k_B} \frac{\sigma_{ion} P N_b L_{eff}}{n_{macro} T}
\end{equation}
Here $\sigma_{ion}$ has units of Mb, $P$ has units of Torr, $k_B = 1.38\times10^{-23}$ J/K, $e$ is the electron charge, $L_{eff}$ is the effective length of the ion element (in m), and $n_{macro}$ is the number of macroparticles generated.
%, and we have assumed $T=300K$.
The initial ion distribution follows the bunch distribution (assumed to be Gaussian).

The \verb|IONEFFECTS| element also supports multiple ionization.  In the \verb|ion_properties| file, one can define the \verb|SourceName| for a given \verb|IonName| to be another ion.  In this case, each macro-ion of type \verb|SourceName| has a chance of being multiply ionized into type \verb|IonName|.  The calculation is done every \verb|multiple_ionization_interval| bunch passes.  The probability of multiple ionization depends on the cross section and local beam density.

%Multiply ionized molecules (e.g. ions whose source is another ion If mulitply ionized molecules are included in ionProperties.sdds, then the source ion has a chance of being multiply ionzed by each bunch.  The probability of multiple ionization depends on the cross section and local beam density.

%\paragraph{Beam-ion interactions}

The kick on the ions from the beam is calculated using the
Basetti-Erskine formula~\cite{Bassetti}, which assumes the beam is
Gaussian in both transverse dimensions.  This may be a poor assumption for
the ions, in which case the \verb|field_calculation_method| parameter can be
set to \verb|bigaussian|, which uses a sum of two gaussians. This provides
a much better model for the actual distribution, at the expense of a considerable
increase in run time. The \verb|ion_bin_divisor| and \verb|ion_range_multiplier| parameters can
be used to control the bin size and range, respectively, of the histogram used to 
approximate the ion charge distribution.
The \verb|ion_bin_divisor| gives the ratio of the rms size of the electron bunch
in the plane in question to the bin size.
If positive, \verb|ion_range_multiplier| gives the range of the binned coordinates in 
units of the rms size of the ion distribution.
If negative, a rough (100-bin) histogram of the ion distribution is used to
estimate the range required to encompass 98\% of the ions; this value is multiplied
by the absolute value of \verb|ion_range_multiplier| to get the range of the
fill histogram; a value of -2 or -3 is suggested.
The \verb|ion_histogram_output| parameter and related parameters can be used to 
request output of the ion distribution and the bi-gaussian fit, which is advisable
when setting the binning parameters.

The change in momentum of an ion due to the bunch passage is:
\begin{equation}
%\begin{split}
\Delta p_y + i \Delta p_x  = \frac{c N_b r_e m_e }{\gamma} \sqrt{\frac{2 \pi}{\sigma_x^2 - \sigma_y^2}} \left[ w\left(\frac{x + i y}{\sqrt{2 (\sigma_x^2 - \sigma_y^2)}}\right) 
   - exp\left(\frac{-x^2}{2 \sigma_x^2} - \frac{y^2}{2 \sigma_y^2}\right)  w\left(\frac{\frac{\sigma_y}{\sigma_x} x + i \frac{\sigma_x}{\sigma_y} y}{\sqrt{2 (\sigma_x^2 - \sigma_y^2)}}\right) \right]
%\end{split}
\end{equation}
where $c$ is the speed of light, $N_b$ is the bunch population, $r_e$
is the classical electron radius ($2.82 \times 10^{-15}$ m), $m_e$ is
the electron mass, $\gamma$ is the relativistic factor ($\sim$ 1 for
the ions), $\sigma_{x,y}$ are the horizontal and vertical beam sizes,
$w$ is the complex error function, and $x$ and $y$ are the distance
from the ion to the bunch center.
