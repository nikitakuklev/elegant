This element provides serial or parallel simulation of the interaction of residual gas ions
with the electron beam.
It must be used in concert with the \verb|ion_effects| command, described in \ref{subsec:ioneffects}.

Modeling of residual ions has these features:
\begin{itemize}
\item s-dependent gas pressure profiles for any number of species.
\item Arbitrary ion species, specified by a user-provided file that includes the cross sections.
\item User-defined locations for ion generation. Each \verb|IONEFFECTS| element represents the
      ions present in a segment of the accelerator. The segments start and end half way between
      successive \verb|IONEFFECTS| elements.
\item Arbitrary fill patterns. Uniform fills can be set up using the \verb|bunched_beam| command,
      while custom fills can be set up by generating the beam externally and using the \verb|sdds_beam| command.
\end{itemize}

Some limitations of the model include:
\begin{itemize}
\item Fields from electron bunches and ions are computed based on gaussian parameters. 
      This is a good approximation for the former, but not terribly good for the latter.
\item Ions move only transversely and exist only outside of magnets.
\item No multiple ionization. This will be added in the future.
\end{itemize}

Performing ion simulations involves the following steps
\begin{enumerate}
\item Prepare file describing the ion properties, as described in \ref{subsec:ioneffects}.
  Each ion is generated by a source gas.
\item Prepare file giving gas pressure vs s for the source gases described in the ion 
  properties file.
\item Insert \verb|IONEFFECTS| elements in the lattice. This can be performed using the
  \verb|insert_elements| command (described in \ref{subsec:insertelements}), or
  manually by editing the lattice file.
\item Insert \verb|ion_effects| command after the \verb|run_setup| command. See
  \ref{subsec:ioneffects} for syntax.
  Note that certain properties of the individual \verb|IONEFFECTS| elements can override the
  global settings given by in the \verb|ion_effects| command.
\item Generate a bunched beam, using either the \verb|bunched_beam| command or providing
  an externally-generated beam to the \verb|sdds_beam| command. Section \ref{sect:bunchedBeams}
  gives more information about bunched beams in \verb|elegant|.
\end{enumerate}
