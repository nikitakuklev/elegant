This element is based on section 3.3.1 of the {\em Handbook of
Accelerator Physics and Engineering}, specifically, the
subsections {\bf Single Coulomb scattering of spin-${\rm \frac{1}{2}}$
particles}, {\bf Multiple Coulomb scattering through small angles},
and {\bf Radiation length}.
There are two aspects to this element: scattering and energy loss.

{\bf Scattering.}  The multiple Coulomb scattering formula is used
whenever the thickness of the material is greater than $0.001 X_o$,
where $X_o$ is the radiation length.  (Note that this is inaccurate
for materials thicker than $100 X_o$.)  For this regime, the user need
only specify the material thickness (L) and the radiation length (XO).

For materials thinner than $0.001 X_o$, the user must specify
additional parameters, namely, the atomic number (Z), atomic mass (A),
and mass density (RHO) of the material.  Note that the density is
given in units of $kg/m^3$.  (Multiply by $10^3$ to convert $g/cm^3$
to $kg/m^3$.)  In addition, the simulation parameter PLIMIT may be
modified.  

To understand this parameter, one must understand how {\tt elegant}
simulates the thin materials.  First, it computes the expected number
of scattering events per particle, $ E = \sigma_T n L = \frac{K_1
\pi^3 n L}{K_2^2 + K_2*\pi^2} $, where $n$ is the number density of
the material, L is the thickness of the material, $K_1 = (\frac{2 Z
r_e}{\beta^2 \gamma})^2$, and $K_2 = \frac{\alpha^2
Z^\frac{2}{3}}{(\beta\gamma)^2}$, with $r_e$ the classical electron radius
and $\alpha$ the fine structure constant.  The material is then broken
into $N$ slices, where $N = E/P_{limit}$.  For each slice, each
simulation particle has a probability $E/N$ of scattering.  If scattering
occurs, the location within the slice is computed using a uniform
distribution over the slice thickness.

For each scatter that occurs, the scattering angle, $\theta$ is
computed using the cumulative probability distribution
$F(\theta>\theta_o) = \frac{K_2 (\pi^2 - \theta_o^2)}{\pi^2 (K_2 +
\theta_o^2)}$.  This can be solved for $\theta_o$, giving
$\theta_o = \sqrt{\frac{(1-F)K_2\pi^2}{K_2 + F \pi^2}}$.  For each scatter,
$F$ is chosen from a uniform random distribution on $[0,1]$.

{\bf Energy loss.} There are two ways to compute energy loss in materials, using a simple minded approach and using the bremsstrahlung cross section.
The latter is recommended, but the former is kept for backward compatibility. 
\begin{itemize}
\item To enable bremsstrahlung simulation, simply set \verb|NUCLEAR_BREMSSTRAHLUNG=1|. Note that the energy loss is not correlated with the scattering
  angle, which is not entirely physical but should be reasonable for large numbers of scattering events.
\item To use the simplified approach:
      \begin{itemize}
      \item  Set \verb|ENERGY_DECAY=1|. Energy loss simulation is very simple.
The energy loss per unit distance traveled, $x$, is 
$\frac{dE}{dx} = -E/X_o$.  Hence, in traveling through a
material of thickness $L$, the energy of each particle is
transformed from $E$ to $E e^{-L/X_o}$.  
       \item Optionally, set \verb|ENERGY_STRAGGLE=1|. {\bf Not recomemnded. Exists only for backward compatibility.}
This adds variation in the energy lost
by particles.  The model is {\em very}, {\em very} crude and {\bf not recommended}.  It assumes that the standard deviation of the energy
loss is equal to half the mean energy loss.  This is an overestimate,
we think, and is provided to give an upper bound on the effects of
energy straggling until a real model can be developed.  Note one
obvious problem with this: if you split a MATTER element of length L
into two pieces of length L/2, the total energy loss will not not
change, but the induced energy spread will be about 30\% lower, due to
addition in quadrature.
\end{itemize}
\end{itemize}
