Some confusion may exist about the edge angles, particularly the signs.
For a sector magnet, we have of course \verb|E1=E2=0|.  For a symmetric rectangular
magnet, \verb|E1=E2=ANGLE/2|.  If \verb|ANGLE| is negative, then so are
\verb|E1| and \verb|E2|.  To understand this, imagine a rectangular magnet with positive \verb|ANGLE|.
If the magnet is flipped over, then \verb|ANGLE| becomes negative, as does the bending
radius $\rho$.    Hence, to keep the focal length
of the edge $1/f = -\tan E_i /\rho$ constant, we must also change the sign of
$E_i$.

{\em Special note about splitting dipoles}: when dipoles are long, it is
common to want to split them into several pieces, to get a better look
at the interior optics.  When doing this, care must be exercised not
to change the optics.  {\tt elegant} has some special features that
are designed to reduce or manage potential problems. At issue is the
need to turn off edge effects between the portions of the same dipole.

First, one can simply use the \verb|divide_elements| command to set up
the splitting.  Using this command, {\tt elegant} takes care of everything.

Second, one can use a series of dipoles {\em with the same name}.  In this case,
elegant automatically turns off interior edge effects.  This is true when the
dipole elements directly follow one another or are separated by a MARK element.

Third, one can use a series of dipoles with different names.  In this case, you
must also use the \verb|EDGE1_EFFECTS| and \verb|EDGE2_EFFECTS| parameters to
turn off interior edge effects.  
