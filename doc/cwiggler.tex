This element simulates a wiggler or undulator using Ying Wu's
canonical integration code for wigglers.  To use the element, one must
supply an SDDS file giving harmonic analysis of the wiggler field.
The field expansion used by the code for a horizontally-deflecting
wiggler is (Y. Wu, Duke University, private communication).
\begin{equation}
B_y = -\left|B_0\right| \sum_{m,n} C_{mn}\cos(k_{xl} x) \cosh (k_{ym} y)
\cos(k_{zn} z + \theta_{zn}),
\end{equation}
where $\left|B_0\right|$ is the peak value of the on-axis magnetic field,
the C_{mn} give the relative amplitudes of the harmonics, the wavenumbers
statisfy $k^2_{ym} = k^2_{xl} + k^2_{zn}$, and $\theta_{zn}$ is the phase.

The file must contain the following columns:
\begin{itemize}
\item The harmonic amplitude, $C_{mn}$, in column {\tt Cmn}.
\item The phase, in radians, in column {\tt Phase}.  The phase of the first
harmonic should be 0 or $\pi$ in order to have matched dispersion.
\item The three wave numbers, normalized to $k_w = 2\pi/\lambda_w$, where
 $\lambda_w$ is the wiggler period.  These are given in columns 
 {\tt KxOverKw}, {\tt KyOverKw}, and {\tt KzOverKw}.
\end{itemize}

For matrix and radiation integral computations, {\tt elegant} uses a
WIGGLER element when it encounters a CWIGGLER.  The effective bending
radius is $B\rho/B_0/\sqrt{\sum C_{mn}^2}$ (L. Emery, private
communication).  Tests show that this gives good agreement in the
tunes from tracking and Twiss parameter calculations.
