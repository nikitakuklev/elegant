This element simulates a wiggler or undulator using a modified version of Ying Wu's
canonical integration code for wigglers.  To use the element, one must
supply an SDDS file giving harmonic analysis of the wiggler field.
The field expansion used by the code for a horizontally-deflecting
wiggler is (Y. Wu, Duke University, private communication).
\begin{equation}
B_y = -\left|B_0\right| \sum_{m,n} C_{mn}\cos(k_{xl} x) \cosh (k_{ym} y)
\cos(k_{zn} z + \theta_{zn}),
\end{equation}
where $\left|B_0\right|$ is the peak value of the on-axis magnetic field,
the $C_{mn}$ give the relative amplitudes of the harmonics, the wavenumbers
statisfy $k^2_{ym} = k^2_{xl} + k^2_{zn}$, and $\theta_{zn}$ is the phase.

The file must contain the following columns:
\begin{itemize}
\item The harmonic amplitude, $C_{mn}$, in column {\tt Cmn}.
\item The phase, in radians, in column {\tt Phase}.  The phase of the first
harmonic should be 0 or $\pi$ in order to have matched dispersion.
\item The three wave numbers, normalized to $k_w = 2\pi/\lambda_w$, where
 $\lambda_w$ is the wiggler period.  These are given in columns 
 {\tt KxOverKw}, {\tt KyOverKw}, and {\tt KzOverKw}.
\end{itemize}

In Version 17.3 and later, for matrix computations {\tt elegant} uses
a first-order matrix derived from particle tracking when it
encounterse a CWIGGLER.  Tests show that this gives good agreement in
the tunes from tracking and Twiss parameter calculations.  For
radiation integrals, an idealized sinusoidal wiggler model is used
with bending radius equal to $B\rho/(B_0 \sum C_{mn})$ for each plane.
Energy loss, energy spread, and horizontal emittance should be estimated
accurately.

{\tt elegant} allows specifying field expansions for on-axis $B_y$ and
$B_x$ components, so one can model a helical wiggler.  However, in this
case one set of components should have $\theta_{zn} = 0$ or $\theta_{zn}=\pi$,
while the other should have $\theta_{zn} = \pm \pi/2$.
Using Wu's code, the latter set will not have matched dispersion.  Our
modified version solves this by delaying the beginning of the field
components in question by $\lambda/4$ and ending the field prematurely
by $3\lambda/4$.  This causes all the fields to start and end at the
crest, which ensures matched dispersion.  The downside is that the
(typically) vertical wiggler component is missing a full period of
field.  One can turn off this behavior by setting \verb|FORCE_MATCHED=0|.

{\bf Additional field expansions}

Y. Wu's code included field expansions for a vertically-deflecting
wiggler as well as the horizontally-deflecting wiggler given above.
In both cases, these expansions are suitable for a wiggler with two
poles that are above/below or left/right of the beam axis.  They are
not always suitable for devices with more complex pole geometries.

Another geometry that is important is a ``split pole'' wiggler, in which
each pole is made from two pieces.  Such configurations are seen, for example,
in devices used to produce variable polarization.  In such cases, the
expansion given above may not be appropriate.  Here, we summarize the form of the
various expansions that {\tt elegant} supports.  For brevity, we show the
form of a single harmonic component.

Horizontal wiggler, split poles, produces $B_y$ only on-axis.   Specified
by setting {\tt BY\_SPLIT\_POLE=0}, and giving {\tt BY\_FILE} or {\tt SINUSOIDAL=1} with {\tt VERTICAL=0}.
\begin{eqnarray}
B_x & = & \left|B_0\right| \frac{k_x \cos (k_z z + \phi) \sin (k_x x) \sinh (k_y y)}{k_y}\\
B_y & = & - \left|B_0\right| \cos (k_x x) \cos (k_z z + \phi) \cosh (k_y y)
\end{eqnarray}
where $k_y^2 = k_x^2 + k_z^2$.

Horizontal wiggler, split poles, produces $B_y$ only on-axis.   Specified
by setting {\tt BY\_SPLIT\_POLE=1}, and giving {\tt BY\_FILE} or {\tt SINUSOIDAL=1} with {\tt VERTICAL=0}.
\begin{eqnarray}
B_x & = & -\left|B_0\right| \frac{k_x \cos (k_z z + \phi) \sin (k_y y) \sinh (k_x x)}{k_y}\\
B_y & = & -\left|B_0\right| \cos (k_y y) \cos (k_z z + \phi) \cosh (k_x x)
\end{eqnarray}
where $k_y^2 = k_x^2 + k_z^2$.

Vertical wiggler, normal poles, produces $B_x$ only on-axis.   Specified
by setting {\tt BX\_SPLIT\_POLE=0}, and giving {\tt BX\_FILE} or {\tt SINUSOIDAL=1} with {\tt VERTICAL=1} or
{\tt HELICAL=1}.
\begin{eqnarray}
B_x & = & \left|B_0\right| \cos (k_y y) \cos (k_z z + \phi) \cosh (k_x x)\\
B_y & = & -\left|B_0\right| \frac{k_y \cos (k_z z + \phi) \sin (k_y y) \sinh (k_x x)}{k_x}
\end{eqnarray}
where $k_x^2 = k_y^2 + k_z^2$.

Vertical wiggler, normal poles, produces $B_x$ only on-axis.   Specified
by setting {\tt BX\_SPLIT\_POLE=1}, and giving {\tt BX\_FILE} or {\tt SINUSOIDAL=1} with {\tt VERTICAL=1} or
{\tt HELICAL=1}.
\begin{eqnarray}
B_x & = & \left|B_0\right| \cos (k_x x) \cos (k_z z + \phi) \cosh (k_y y)\\
B_y & = & \left|B_0\right| \frac{k_y \cos (k_z z + \phi) \sin (k_x x) \sinh (k_y y)}{k_x}
\end{eqnarray}
where $k_x^2 = k_y^2 + k_z^2$.
