This somewhat experimental element simulates transport through a 3D magnetic field constructed from
an off-axis expansion.
At present, it is restricted to non-bending elements and in fact to quadrupoles.

Unlike the generalized gradient expansion used by \verb|BGGEXP|, this method of expanding the fields is
prone to corruption by noise.
However, it uses data that can very readily be obtained from magnetic measurements with a Hall probe.
Users are cautioned to take care in deciding how far to trust the expansion.

For quadrupoles, we use the on-axis gradient $g(z)$ and its z derivatives $g^{(n)}(z)$
The scalar potential can be written 
\begin{eqnarray}
\Phi & \approx & \frac{1}{2} \left(-\frac{x^7 y g^{(6)}(z)}{5040}+\frac{1}{120} x^5 y g^{(4)}(z)-\frac{1}{6} x^3 y g^{(2)}(z)\right) \\
 & &  +\frac{1}{2} \left(-\frac{x y^7 g^{(6)}(z)}{5040}+\frac{1}{120} x y^5 g^{(4)}(z)-\frac{1}{6} x y^3 g^{(2)}(z)\right) \\
 & & + x y g(z)
\end{eqnarray}
From which we find
\begin{eqnarray}
B_x & = & \frac{1}{2} \left(-\frac{1}{720} x^6 y g^{(6)}(z)+\frac{1}{24} x^4 y g^{(4)}(z)-\frac{1}{2} x^2 y g^{(2)}(z)\right) \\
 & & + \frac{1}{2} \left(-\frac{y^7 g^{(6)}(z)}{5040}+\frac{1}{120} y^5 g^{(4)}(z)-\frac{1}{6} y^3 g^{(2)}(z)\right) \\ 
 & & +y g(z),
\end{eqnarray}
\begin{eqnarray}
B_y & = & \frac{1}{2} \left(-\frac{x^7 g^{(6)}(z)}{5040}+\frac{1}{120} x^5 g^{(4)}(z)-\frac{1}{6} x^3 g^{(2)}(z)\right) \\
& & + \frac{1}{2}  \left(-\frac{1}{720} x y^6 g^{(6)}(z)+\frac{1}{24} x y^4 g^{(4)}(z)-\frac{1}{2} x y^2 g^{(2)}(z)\right) \\
& & +x g(z),
\end{eqnarray}
and
\begin{eqnarray}
B_z & = & \frac{1}{2} \left(-\frac{x^7 y g^{(7)}(z)}{5040}+\frac{1}{120} x^5 y g^{(5)}(z)-\frac{1}{6} x^3 y g^{(3)}(z)\right) \\
& &  + \frac{1}{2} \left(-\frac{x y^7 g^{(7)}(z)}{5040}+\frac{1}{120} x y^5 g^{(5)}(z)-\frac{1}{6} x y^3 g^{(3)}(z)\right) \\
& & + x y g^{(1)}(z).
\end{eqnarray}

These equations satisfy Maxwell's curl equation exactly while satisfying the divergence equation to high order.
In particular
\begin{equation}
\vec{\nabla} \cdot \vec{B} = -\frac{x^7 y g^{(8)}(z)}{10080}-\frac{x y^7 g^{(8)}(z)}{10080}.
\end{equation}

The gradient $g(z)$ is specified in the column named by the \verb|GRADIENT_COLUMN| parameter.
The names for the columns containing z derivatives of $g(z)$ are constructed from the name of the gradient.
Assume for concreteness that \verb|GRADIENT_COLUMN="Gradient"|. 
{\tt elegant} looks for $g^{(1)}(z)$ in column \verb|GradientDeriv| and 
$g^{(n)}(z)$ for $n>1$ in columns \verb|GradientDeriv2|, \verb|GradientDeriv3|,  etc.
The expansion is truncated if the needed columns do not exist in the input file.
The needed derivatives can be obtained using the program \verb|sddsderiv|, e.g.,
\begin{verbatim}
sddsderiv gradient.sdds gradient1.sdds -differ=Gradient -versus=z -savitzky=7,7,7,1
sddsderiv gradient.sdds gradient2.sdds -differ=Gradient -versus=z -savitzky=7,7,7,2
sddsderiv gradient.sdds gradient3.sdds -differ=Gradient -versus=z -savitzky=7,7,7,3
sddsxref gradient.sdds gradient[123].sdds gradients.sdds -take=*Deriv* 
\end{verbatim}
(In this case, we use a Savitzky-Golay filter to compute the derivatives using a 7$^{th}$ order
fit with 7 points ahead of and behind the evaluation location.)
The file \verb|gradients.sdds| would then be given as the value of \verb|FILENAME|.

High-order numerical derivative are of course prone to corruption by measurement noise. 
Examining the derivatives is strongly recommended to ensure this is not an issue.
