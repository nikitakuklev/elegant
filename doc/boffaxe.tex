This somewhat experimental element simulates transport through a 3D magnetic field constructed from
an off-axis expansion.
At present, it is restricted to non-bending elements and in fact to quadrupoles.

This method of expanding the fields is prone to corruption by noise, to a much greater degree than
the generalized gradient expansion used by \verb|BGGEXP|.
However, it uses data that can very readily be obtained from magnetic measurements with a Hall probe.
Users are cautioned to take care in deciding how far to trust the expansion.

For quadrupoles, we use the on-axis gradient $g(z)$ and its z derivatives $g^{(n)}(z)$
The scalar potential can be written 
\begin{equation}
\Phi = -\frac{1}{12} x^3 y g''(z)\right) -\frac{1}{12} x y^3 g''(z) + x y g(z)
\end{equation}
From which we find
\begin{equation}
B_x = -\frac{1}{4} x^2 y g''(z)-\frac{1}{12} y^3 g''(z)+y g(z),
\end{equation}
\begin{equation}
B_y = -\frac{1}{12} x^3 g''(z)-\frac{1}{4} x y^2 g''(z)+x g(z),
\end{equation}
and
\begin{equation}
B_z = -\frac{1}{12} x^3 y g^{(3)}(z)-\frac{1}{12} x y^3 g^{(3)}(z)+x y g'(z).
\end{equation}
These equations satisfy Maxwell's curl equation exactly while satisfying the divergence equation to 
4$^{th}$ order.

The gradient $g(z)$ is specified in the column named by the \verb|GRADIENT_COLUMN| parameter.
The names for the columns containing z derivatives of $g(z)$ are constructed from the name of the gradient.
Assume for concreteness that \verb|GRADIENT_COLUMN="Gradient"|. 
{\tt elegant} looks for $g^{(1)}(z)$ in column \verb|GradientDeriv| and 
$g^{(n)}(z)$ for $n>1$ in columns \verb|GradientDeriv2|, \verb|GradientDeriv3|,  etc.
Also the expansion is limited to using $g^{(3)}(z)$, higher derivatives, if given, will be used
for interpolation with respect to $z$ if the \verb|Z_SUBDIVISIONS| parameter is larger than 1.
The expansion is truncated if the needed columns do not exist in the input file.

The needed derivatives can be obtained using the program \verb|sddsderiv|, e.g.,
\begin{verbatim}
sddsderiv gradient.sdds gradient1.sdds -differ=Gradient -versus=z -savitzky=7,7,7,1
sddsderiv gradient.sdds gradient2.sdds -differ=Gradient -versus=z -savitzky=7,7,7,2
sddsderiv gradient.sdds gradient3.sdds -differ=Gradient -versus=z -savitzky=7,7,7,3
sddsxref gradient.sdds gradient[123].sdds gradients.sdds -take=*Deriv* 
\end{verbatim}
(In this example, we use a Savitzky-Golay filter to compute the first three z derivatives of $g(z)$
using a 7$^{th}$ order fit with 7 points ahead of and behind the evaluation location.)
The file \verb|gradients.sdds| would then be given as the value of \verb|FILENAME|.

High-order numerical derivative are of course prone to corruption by measurement noise. 
Examining the derivatives is strongly recommended to ensure this is not an issue.
