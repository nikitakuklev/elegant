This element simulates a quadrupole using a matrix of first, second,
or third order.

As of version 29.2, this element incorporates the ability to have different values for the insertion
and effective lengths. This is invoked when \verb|LEFFECTIVE| is positive. In this case, the
\verb|L| parameter is understood to be the physical insertion length. Using \verb|LEFFECTIVE| is
a convenient way to incorporate the fact that the effective length may differ from the physical
length and even vary with excitation, without having to modify the drift spaces on either side of
the quadrupole element.

By default, the element has hard edges and constant field within the
defined length, {\tt L}.  However, this element supports two different methods of implementing fringe fields.
Which method is used is determined by the \verb|FRINGE_TYPE| parameter.

\paragraph{Edge integral method} The most recent and preferred implementation of fringe field effects is based on edge
integrals and is invoked by setting \verb|FRINGE_TYPE| to ``integrals''.  This method is compatible with the use of
\verb|LEFFECTIVE|. However, it provides a first-order matrix only.

The model is based on publications of D.  Zhuo {\em et al.} \cite{Zhou-IPAC10} and  J. Irwin {\em et
  al.} \cite{Irwin-PAC95}, as well as unpublished work of C. X. Wang (ANL).  The fringe field is characterized by 
10 integrals given in equations 19, 20, and 21 of \cite{Zhou-IPAC10}.  However, the values input into {\tt elegant}
should be normalized by $K_1$ or $K_1^2$, as appropriate.

For the exit-side fringe field, let $s_1$ be the center of the magnet, $s_0$ be the location of the nominal end of the magnet
(for a hard-edge model), and let $s_2$ be a point well outside the magnet.  
Using $K_{1,he}(s)$ to represent the hard edge model and $K_1(s)$ the actual field profile, we 
define the normalized difference as $\tilde{k}(s) = (K_1(s) - K_{1,he}(s))/K_1(s_1)$.  (Thus, $\tilde{k}(s) = \tilde{K}(s)/K_0$, using
the notation of Zhou {\em et al.})

The integrals to be input to {\tt elegant} are defined as 
\begin{eqnarray}
i_0^- = \int_{s_1}^{s_0} \tilde{k}(s) ds & & i_0^+ = \int_{s_0}^{s_2} \tilde{k}(s) ds \\
i_1^- = \int_{s_1}^{s_0} \tilde{k}(s) (s-s_0) ds & & i_1^+ = \int_{s_0}^{s_2} \tilde{k}(s) (s-s_0) ds \\
i_2^- = \int_{s_1}^{s_0} \tilde{k}(s) (s-s_0)^2 ds & & i_2^+ = \int_{s_0}^{s_2} \tilde{k}(s) (s-s_0)^2 ds \\
i_3^- = \int_{s_1}^{s_0} \tilde{k}(s) (s-s_0)^3 ds & & i_3^+ = \int_{s_0}^{s_2} \tilde{k}(s) (s-s_0)^3 ds \\
\lambda_2^- = \int_{s_1}^{s_0} ds \int_s^{s_0} ds\prime \tilde{k}(s) \tilde{k}(s\prime) (s\prime-s) & & 
\lambda_2^+ = \int_{s_0}^{s_2} ds \int_s^{s_2} ds\prime \tilde{k}(s) \tilde{k}(s\prime) (s\prime-s) 
\end{eqnarray}

Normally, the effects are dominated by $i_1^-$ and $i_1^+$.

\paragraph{Trapazoidal models}
This method is based on a third-order matrix formalism and the assumption that the 
fringe fields depend linearly on $z$.  Although the third-order matrix is computed, it is important
to note that the assumed fields do not satisfy Maxwell's equations.

To invoke this method, one specifies ``inset'' or
``fixed-strength'' for the \verb|FRINGE_TYPE| parameter and then provides
a non-zero value for {\tt FFRINGE}. If  {\tt FFRINGE} is zero (the default), then the magnet
is hard-edged regardless of the setting of \verb|FRINGE_TYPE|.  If {\tt FFRINGE} is positive, then the magnet has
linear fringe fields of length {\tt FFRINGE*L/2} at each end.  That
is, the total length of fringe field from both ends combined is {\tt
FFRINGE*L}.

Depending on the value of {\tt FRINGE\_TYPE}, the fringe fields are
modeled as contained within the length {\tt L} (``inset'' type) or
extending symmetrically outside the length {\tt L} (``fixed-strength''
type).

For ``inset'' type fringe fields, the length of the ``hard core'' part of
the quadrupole is {\tt L*(1-FFRINGE)}.  For ``fixed-strength'' type fringe fields,
the length of the hard core is {\tt L*(1-FFRINGE/2)}.  In the latter case,
the fringe gradient reaches 50\% of the hard core value at the nominal boundaries
of the magnet. This means that the integrated strength of the magnet does not
change as the {\tt FFRINGE} parameter is varied. This is not the case with
``inset'' type fringe fields.

