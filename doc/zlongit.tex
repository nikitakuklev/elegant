This element allows simulation of a longitudinal impedance using a
``broad-band'' resonator or an impedance function specified in a file.
The impedance is defined as the Fourier transform of the wake function
\begin{equation}
Z(\omega) = \int_{-\infty}^{+\infty} e^{-i \omega t} W(t) dt
\end{equation}
where $i = \sqrt{-1}$, $W(t)=0$ for $t<0$, and $W(t)$ has units of $V/C$.

For a resonator impedance, the functional form is
\begin{equation}
Z(\omega) = \frac{R_s}{1 + iQ(\frac{\omega}{\omega_r} - \frac{\omega_r}{\omega})},
\end{equation}
where $R_s$ is the shunt impedance in $Ohms$, $Q$ is the quality
factor, and $\omega_r$ is the resonant frequency.

When providing an impedance in a file, the user must be careful to conform to these
conventions.

Other notes:
\begin{enumerate}
\item The frequency data required from the input file is {\em not} $\omega$, but rather
  $f = \omega/(2 \pi)$.
\item The default smoothing setting ({\tt SG\_HALFWIDTH=4}), may apply too much smoothing.
  It is recommended that the user vary this parameter if smoothing is employed.
\item Using the broad-brand resonator model can often result in a very large number of bins
 being used, as {\tt elegant} will try to resolve the resonance peak and achieve the desired
 bin spacing. This can result in poor performance, particularly for the parallel version.
\end{enumerate}

Bunched-mode application of the impedance is possible using specially-prepared input
beams. 
See Section \ref{sect:bunchedBeams} for details.
The use of bunched mode for any particular \verb|ZLONGIT| element is controlled using the \verb|BUNCHED_BEAM_MODE| parameter.
