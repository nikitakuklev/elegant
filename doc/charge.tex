This is the preferred way to assign charge to a beam, which is needed for the use of CSR simulation (CSRCSBEND, CSRDRIFT),
wake simulation (WAKE, TRWAKE, LRWAKE, ZLONGIT, ZTRANSVERSE), rf mode simulation (RFMODE, TRFMODE, FRFMODE, RTRFMODE),
space charge simulation (LSCDRIFT, RFCW, SCMULT), and intrabeam scattering simulation (IBSCATTER).

This element is also used to declare the bunch structure of the beam, since this is also involved in calculations of
collective effects. This is done by specifying \verb|N_BUCKETS|, which declares how many buckets to divide the beam into for
the purposes of collective effects calculations. The number of buckets $b$ in this sense should not be confused with the number $h$ of rf buckets in a
storage ring, although typically $b$ divides evenly into $h$.
In \verb|STORAGE_RING_BUCKET_MODE=1| the revolution length is used to determine the
duration of each bucket. Otherwise, the bucket durations are determined from the beam; this can be problematical if,
for example, particles escape from the real rf bucket and incur large time offsets from their bunch. To avoid this, the \verb|CLEAN|
element or another filter may be needed.
