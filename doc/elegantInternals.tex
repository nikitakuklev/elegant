% How to create this document on UNIX systems:
%   latex elegant
%   dvi2ps elegant | lpr [-P<postscript-printer-name>]
%
\documentclass[11pt]{article}
\usepackage{html}
\pagestyle{plain}
%\voffset=-0.75in
\newenvironment{req}{\begin{equation} \rm}{\end{equation}}
\setlength{\topmargin}{0.15 in}
\setlength{\oddsidemargin}{0 in}
\setlength{\evensidemargin}{0 in} % not applicable anyway
\setlength{\textwidth}{6.5 in}
\setlength{\headheight}{-0.5 in} % for 11pt font size
%\setlength{\footheight}{0 in}
\setlength{\textheight}{9 in}
\begin{document}

\title{Developer's Guide to the Internals of {\tt elegant}}
\author{Advanced Photon Source\\Michael Borland\\ \date{\today}}
\maketitle

This guide is an on-going effort to document the internals of {\tt elegant}
in order to make it easier for developers to understand the code and
make changes.

\section{Particle Arrays}

Particles are stored in a two-dimensional array, the first index being
the particle index, and the second being the coordinate index.  The
coordinates are stored in the following order:
\begin{itemize}
\item x --- horizontal position.
\item x$^\prime$ --- horizontal slope.
\item y --- vertical position.
\item y$^\prime$ ---  vertical slope.
\item s --- Total equivalent distance traveled at the present velocity.  For
  particles that don't change energy, this is just the total path length.
\item $\delta$ --- The fractional momentum offset, $(p-p_0)/p_0$, where
  $p_0$ is the defined central momentum of the beamline.
\item particle identification number --- This is an integer value supplied by
  the user or generated internally.
\end{itemize}

There are several arrays used by {\tt elegant} to keep store information about
particles.  These are collected in the BEAM structure:
\begin{itemize}
  \item {\tt original} --- This stores the original particle coordinates.  These are
    needed in the event that they need to be reused in a subsequent tracking step.
    In some cases (where a new beam is used each time and the user asks to reduce
    memory usage), the array is NULL.  This array is not used in tracking (e.g.,
    by the {\tt do\_tracking} routine.  It is used only by the routines that
    prepare beam for tracking.
  \item {\tt particle} --- This stores the current particle coordinates.  I.e.,
    just before tracking it has the original coordinates, but as tracking proceeds,
    the values are updated. In addition, as particles are lost, they are swapped
    to the top of the array and the coordinate slots are used to store information
    about the particle at the time of loss.   This avoids having to use another
    array for lost particle information.  Note that in some subroutines, this array
    is called {\tt coord}.
  \item {\tt accepted} --- This array is optional.  It is used only if the user asks
    for acceptance output in the {\tt run\_setup} namelist.  Just before tracking,
    it will contain the initial coordinates of all particles.  When particles are
    lost, they are swapped to the top of the array, just as done for the {\tt particle}
    array.  In this way, the initial coordinates of both lost and surviving particles
    are kept.  Only the later are provided to the user.
  \item {\tt lostOnPass} --- Records the turn on which particles are lost.  This array
    is parallel to the {\tt particle} and  {\tt accepted} arrays.  It is filled from
    the top as lost particles are swapped up to the top of the list of surviving particles.
\end{itemize}


\end{document}
