This element imposes kicks on the beam according
to a Hamiltonian that is a polynomial function of x and y 
\begin{equation}
H_k\Delta s = \sum_{i=0}^{6} \sum_{j=0}^{6} K_{ij} x^i y^j 
\end{equation}
where $K_{00}$ is ignored.
The changes to the momenta are determined via Hamilton's equations, e.g.,
\begin{equation}
\Delta q_x = -\frac{\partial H_k\Delta s}{\partial x}
\end{equation}

It also implements a generalized drift that is described by another Hamiltonian
\begin{equation}
H_d\Delta s = (1+\delta) \sum_{i=0}^{6} \sum_{j=0}^{6} D_{ij} \left(\frac{q_x}{1+\delta}\right)^i \left(\frac{q_y}{1+\delta}\right)^j
\end{equation}
where $D_{00}$ is ignored.
Again, the changes to the positions are determined via Hamilton's equations, e.g., 
\begin{equation}
\Delta x = \frac{\partial H_d\Delta s}{\partial q_x}
\end{equation}

In version 2019.1.0, another option was added for the drift Hamiltonian.
This is activated by setting the paramter \verb|DRIFT_TYPE| to 2 (the default is 1)
and setting the \verb|E| values instead of the \verb|D| values.
In this case, the $\delta$ dependence is under user control
\begin{equation}
H_d\Delta s = \sum_{i=0}^{6} \sum_{j=0}^{6} \sum_{k=0}^{6} E_{ijk} q_x^i q_y^j \delta^k
\end{equation}
where $E_{000}$ is ignored.

In more detail, the drift Hamiltonian is applied on both sides of the kick Hamiltonian, but with
half strength. 

For example, a quadrupole of length $L$ with integrated strength $K_1 L$  could be specified by setting
$K_{20} = -K_{02} = K_1 L/2$ and $D_20 = D02 = L/2$.
A sextupole with integrated strength $K_2 L$ could be specified by setting $K_{30} = K_2 L/6$ and $K_{12} = K_2 L/2$
and $D_20 = D02 = L/2$.
The purpose, however, is not to simulate such elements, since they can be more conveniently simulated with 
\verb|KQUAD| or \verb|KSEXT|.
It is rather to simulate elements that may not be described by the usual multipoles.

