This element imposes kicks on the beam according
to a Hamiltonian that is a polynomial function of x and y 
\begin{equation}
H_k\Delta s = \sum_{i=0}^{4} \sum_{j=0}^{4} \sum_{k=0}^4 \sum_{l=0}^4 K_{ij} x^i y^j 
\end{equation}
where $K_{00} = 0$.
The changes to the momenta are determined via Hamilton's equations, e.g.,
\begin{equation}
\Delta q_x = -\frac{\partial H_k\Delta s}{\partial}
\end{equation}

It also implements a generalized drift that is described by another Hamiltonian
\begin{equation}
H_d\Delta s = \sum_{i=0}^{4} \sum_{j=0}^{4} \sum_{k=0}^4 \sum_{l=0}^4 D_{ij} q_x^i q_y^j 
\end{equation}
where $K_{00} = 0$.
Again, the changes to the positions are determined via Hamilton's equations, e.g., 
\begin{equation}
\Delta x = \frac{\partial H_d\Delta s}{\partial}
\end{equation}

In more detail, the drift Hamiltonian is applied on both sides of the kick Hamiltonian, but with
half strength. 

For example, a quadrupole of length $L$ with integrated strength $K_1 L$  could be specified by setting
$K_{20} = -K_{02} = K_1 L/2$ and $D_20 = D02 = L/2$.
A sextupole with integrated strength $K_2 L$ could be specified by setting $K_{3000} = K_2 L/6$ and $K_{1200} = K_2 L/2$
and $D_20 = D02 = L/2$.
The purpose, however, is not to simulate such elements, since they can be more conveniently simulated with 
\verb|KQUAD| or \verb|KSEXT|.
It is rather to simulate elements that may not be described by the usual multipoles.