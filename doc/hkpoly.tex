This element imposes kicks on the beam according
to a Hamiltonian that is a polynomial function of x and y 
\begin{equation}
H\Delta s = \sum_{i=0}^{4} \sum_{j=0}^{4} C_{ij} x^i y^j,
\end{equation}
where $C_{00} = 0$.

The changes to the momenta are
\begin{equation}
\Delta q_x = -\frac{\partial H}{\partial x}\Delta s = \sum_{i=1}^4 \sum_{j-0}^4 C_{ij} i x^{i-1} y^j
\end{equation}
and
\begin{equation}
\Delta q_y = -\frac{\partial H}{\partial y}\Delta s = \sum_{i=0}^4 \sum_{j=1}^4 C_{ij} j x^i y^{j-1}
\end{equation}
where
\begin{equation}
  q_x = \frac{(1+\delta) x^\prime}{\sqrt{1 + x^{\prime 2} + y^{\prime 2}}},
\end{equation}
and similarly for $q_y$.

For example, a quadrupole with integrated strength $K_1 L$  could be specified by setting $C_{20} = -C_{02} = K_1 L/2$.
A sextupole with integrated strength $K_2 L$ could be specified by setting $C_{30} = K_2 L/6$ and $C_{12} = K_2 L/2$.
The purpose, however, is not to simulate such elements, since they can be more conveniently simulated with 
\verb|KQUAD| or \verb|KSEXT|.
It is rather to simulate elements that may not be described by the usual multipoles.