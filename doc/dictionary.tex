\newlength{\descwidth}
\setlength{\descwidth}{2in}
\begin{latexonly}
\newpage
\begin{center}{\Large\verb|ALPH|}\end{center}
\end{latexonly}\subsection{ALPH}
An alpha magnet implemented as a matrix, up to 3rd order.  PART is used to split
the magnet into halves.  XS$<$n$>$ and DP$<$n$>$ allow momentum filtration at the midpoint.
\\
\begin{tabular}{|l|l|l|l|p{\descwidth}|} \hline
Parameter Name & Units & Type & Default & Description \\ \hline 
XMAX & $M$ & double &  0.0 & size of alpha  \\ \hline 
XS1 & $M$ & double &  0.0 & inner scraper position  \\ \hline 
XS2 & $M$ & double &  0.0 & outer scraper position  \\ \hline 
DP1 &  & double &   -1 & inner scraper momentum deviation  \\ \hline 
DP2 &  & double &   1 & outer scraper momentum deviation  \\ \hline 
XPUCK & $M$ & double &   -1 & position of scraper puck  \\ \hline 
WIDTHPUCK & $M$ & double &  0.0 & size of scraper puck  \\ \hline 
DX & $M$ & double &  0.0 & misalignment  \\ \hline 
DY & $M$ & double &  0.0 & misalignment  \\ \hline 
DZ & $M$ & double &  0.0 & misalignment  \\ \hline 
TILT &  & double &  0.0 & rotation about incoming longitudinal axis  \\ \hline 
PART &  & long &  \verb|0| & 0=full, 1=first half, 2=second half  \\ \hline 
ORDER &  & long &  \verb|0| & matrix order [1,3]  \\ \hline 
\end{tabular}

\begin{latexonly}
\newpage
\begin{center}{\Large\verb|BMAPXY|}\end{center}
\end{latexonly}\subsection{BMAPXY}
A map of Bx and By vs x and y.
\\
\begin{tabular}{|l|l|l|l|p{\descwidth}|} \hline
Parameter Name & Units & Type & Default & Description \\ \hline 
L & $M$ & double &  0.0 & length  \\ \hline 
STRENGTH & $NULL$ & double &  0.0 & factor by which to multiply field  \\ \hline 
ACCURACY & $NULL$ & double &  0.0 & integration accuracy  \\ \hline 
METHOD & $NULL$ & STRING &   NULL            & integration method (runge-kutta, bulirsch-stoer, modified-midpoint, two-pass modified-midpoint, leap-frog, non-adaptive runge-kutta  \\ \hline 
FILENAME & $NULL$ & STRING &   NULL            & name of file containing columns (x, y, Fx, Fy) giving normalized field (Fx, Fy) vs (x, y)  \\ \hline 
\end{tabular}

\begin{latexonly}
\newpage
\begin{center}{\Large\verb|BUMPER|}\end{center}
\end{latexonly}\subsection{BUMPER}
A time-dependent uniform-field rectangular kicker magnet with no fringe effects.
The waveform is in mpl format, with time in seconds and amplitude normalized to 1.
\\
\begin{tabular}{|l|l|l|l|p{\descwidth}|} \hline
Parameter Name & Units & Type & Default & Description \\ \hline 
L & $M$ & double &  0.0 & length  \\ \hline 
ANGLE & $RAD$ & double &  0.0 & kick angle  \\ \hline 
TILT & $RAD$ & double &  0.0 & rotation about longitudinal axis  \\ \hline 
TIME\_OFFSET & $S$ & double &  0.0 & time offset of waveform  \\ \hline 
PERIODIC &  & long &  \verb|0| & is waveform periodic?  \\ \hline 
PHASE\_REFERENCE &  & long &  \verb|0| & phase reference number (to link with other time-dependent elements)  \\ \hline 
FIRE\_ON\_PASS &  & long &  \verb|0| & pass number to fire on  \\ \hline 
WAVEFORM &  & STRING &   NULL            & $<$filename$>$=$<$x$>$+$<$y$>$ form specification of input file giving kick factor vs time  \\ \hline 
SPATIAL\_DEPENDENCE &  & STRING &   NULL            & rpn function of x and y giving spatial dependence of kick  \\ \hline 
\end{tabular}

\begin{latexonly}
\newpage
\begin{center}{\Large\verb|CENTER|}\end{center}
\end{latexonly}\subsection{CENTER}
An element that centers the beam transversely on the ideal trajectory.
\\
\begin{tabular}{|l|l|l|l|p{\descwidth}|} \hline
Parameter Name & Units & Type & Default & Description \\ \hline 
X &  & long &   1               & center x coordinates?  \\ \hline 
XP &  & long &   1               & center x' coordinates?  \\ \hline 
Y &  & long &   1               & center y coordinates?  \\ \hline 
YP &  & long &   1               & center y' coordinates?  \\ \hline 
ONCE\_ONLY &  & long &  \verb|0| & compute centering offsets for first beam only, apply to all?  \\ \hline 
\end{tabular}

\begin{latexonly}
\newpage
\begin{center}{\Large\verb|CEPL|}\end{center}
\end{latexonly}\subsection{CEPL}
A numerically-integrated linearly-ramped electric field deflector.
\\
\begin{tabular}{|l|l|l|l|p{\descwidth}|} \hline
Parameter Name & Units & Type & Default & Description \\ \hline 
L & $M$ & double &  0.0 & length  \\ \hline 
RAMP\_TIME & $S$ & double &   1e-09 & time to ramp to full strenth  \\ \hline 
TIME\_OFFSET & $S$ & double &  0.0 & offset of ramp-start time  \\ \hline 
VOLTAGE & $V$ & double &  0.0 & maximum voltage between plates due to ramp  \\ \hline 
GAP & $M$ & double &   0.01 & gap between plates  \\ \hline 
STATIC\_VOLTAGE & $V$ & double &  0.0 & static component of voltage  \\ \hline 
TILT & $RAD$ & double &  0.0 & rotation about longitudinal axis  \\ \hline 
ACCURACY &  & double &   0.0001 & integration accuracy  \\ \hline 
X\_MAX & $M$ & double &  0.0 & x half-aperture  \\ \hline 
Y\_MAX & $M$ & double &  0.0 & y half-aperture  \\ \hline 
DX & $M$ & double &  0.0 & misalignment  \\ \hline 
DY & $M$ & double &  0.0 & misalignment  \\ \hline 
PHASE\_REFERENCE &  & long &  \verb|0| & phase reference number (to link with other time-dependent elements)  \\ \hline 
N\_STEPS &  & long &   100             & number of steps (for nonadaptive integration)  \\ \hline 
METHOD & $ $ & STRING &   runge-kutta     & integration method (runge-kutta, bulirsch-stoer, non-adaptive runge-kutta, modified midpoint)  \\ \hline 
FIDUCIAL &  & STRING &   t,median        & \{t$|$p\},\{median$|$min$|$max$|$ave$|$first$|$light\} (e.g., "t,median")  \\ \hline 
\end{tabular}

\begin{latexonly}
\newpage
\begin{center}{\Large\verb|CHARGE|}\end{center}
\end{latexonly}\subsection{CHARGE}
An element to establish the total charge of a beam.  Active on first pass only.  If given, overrides all charge specifications on other elements.
\\
\begin{tabular}{|l|l|l|l|p{\descwidth}|} \hline
Parameter Name & Units & Type & Default & Description \\ \hline 
TOTAL & $C$ & double &  0.0 & total charge in beam  \\ \hline 
PER\_PARTICLE & $C$ & double &  0.0 & charge per macroparticle  \\ \hline 
\end{tabular}

\begin{latexonly}
\newpage
\begin{center}{\Large\verb|CLEAN|}\end{center}
\end{latexonly}\subsection{CLEAN}
Cleans the beam by removing outlier particles.
\\
\begin{tabular}{|l|l|l|l|p{\descwidth}|} \hline
Parameter Name & Units & Type & Default & Description \\ \hline 
MODE &  & STRING &   stdeviation     & stdeviation, absdeviation, or absvalue  \\ \hline 
XLIMIT &  & double &  0.0 & Limit for x  \\ \hline 
XPLIMIT &  & double &  0.0 & Limit for x'  \\ \hline 
YLIMIT &  & double &  0.0 & Limit for y  \\ \hline 
YPLIMIT &  & double &  0.0 & Limit for y'  \\ \hline 
TLIMIT &  & double &  0.0 & Limit for t  \\ \hline 
DELTALIMIT &  & double &  0.0 & Limit for (p-p0)/p0  \\ \hline 
\end{tabular}

\begin{latexonly}
\newpage
\begin{center}{\Large\verb|CSBEND|}\end{center}
\end{latexonly}\subsection{CSBEND}
A canonical kick sector dipole magnet.
\\
\begin{tabular}{|l|l|l|l|p{\descwidth}|} \hline
Parameter Name & Units & Type & Default & Description \\ \hline 
L & $M$ & double &  0.0 & arc length  \\ \hline 
ANGLE & $RAD$ & double &  0.0 & bend angle  \\ \hline 
K1 & $1/M^{2}$ & double &  0.0 & geometric quadrupole strength  \\ \hline 
K2 & $1/M^{3}$ & double &  0.0 & geometric sextupole strength  \\ \hline 
K3 & $1/M^{3}$ & double &  0.0 & geometric octupole strength  \\ \hline 
K4 & $1/M^{4}$ & double &  0.0 & geometric decapole strength  \\ \hline 
E1 & $RAD$ & double &  0.0 & entrance edge angle  \\ \hline 
E2 & $RAD$ & double &  0.0 & exit edge angle  \\ \hline 
TILT & $RAD$ & double &  0.0 & rotation about incoming longitudinal axis  \\ \hline 
H1 & $1/M$ & double &  0.0 & entrance pole-face curvature  \\ \hline 
H2 & $1/M$ & double &  0.0 & exit pole-face curvature  \\ \hline 
HGAP & $M$ & double &  0.0 & half-gap between poles  \\ \hline 
FINT &  & double &   0.5 & edge-field integral  \\ \hline 
DX & $M$ & double &  0.0 & misalignment  \\ \hline 
DY & $M$ & double &  0.0 & misalignment  \\ \hline 
DZ & $M$ & double &  0.0 & misalignment  \\ \hline 
FSE &  & double &  0.0 & fractional strength error  \\ \hline 
ETILT &  & double &  0.0 & error rotation about incoming longitudinal axis  \\ \hline 
N\_KICKS &  & long &   4               & number of kicks  \\ \hline 
NONLINEAR &  & long &   1               & include nonlinear field components?  \\ \hline 
SYNCH\_RAD &  & long &  \verb|0| & include classical synchrotron radiation?  \\ \hline 
EDGE1\_EFFECTS &  & long &   1               & include entrace edge effects?  \\ \hline 
EDGE2\_EFFECTS &  & long &   1               & include exit edge effects?  \\ \hline 
EDGE\_ORDER &  & long &   1               & order to which to include edge effects  \\ \hline 
INTEGRATION\_ORDER &  & long &   2               & integration order (2 or 4)  \\ \hline 
EDGE1\_KICK\_LIMIT &  & double &   -1 & maximum kick entrance edge can deliver  \\ \hline 
\end{tabular}

\begin{latexonly}
\newpage
\begin{center}{\Large\verb|CSBEND| continued}\end{center}
\end{latexonly}
A canonical kick sector dipole magnet.
\\
\begin{tabular}{|l|l|l|l|p{\descwidth}|} \hline
Parameter Name & Units & Type & Default & Description \\ \hline 
EDGE2\_KICK\_LIMIT &  & double &   -1 & maximum kick exit edge can deliver  \\ \hline 
KICK\_LIMIT\_SCALING &  & long &  \verb|0| & scale maximum edge kick with FSE?  \\ \hline 
USE\_BN &  & long &  \verb|0| & use B$<$n$>$ instead of K$<$n$>$?  \\ \hline 
B1 & $1/M$ & double &  0.0 & K1 = B1*rho, where rho is bend radius  \\ \hline 
B2 & $1/M^{2}$ & double &  0.0 & K2 = B2*rho  \\ \hline 
B3 & $1/M^{3}$ & double &  0.0 & K3 = B3*rho  \\ \hline 
B4 & $1/M^{4}$ & double &  0.0 & K4 = B4*rho  \\ \hline 
ISR &  & long &  \verb|0| & include incoherent synchrotron radiation (scattering)?  \\ \hline 
\end{tabular}

\begin{latexonly}
\newpage
\begin{center}{\Large\verb|CSRCSBEND|}\end{center}
\end{latexonly}\subsection{CSRCSBEND}
Like CSBEND, but incorporates a simulation of Coherent Synchrotron radiation.
\\
\begin{tabular}{|l|l|l|l|p{\descwidth}|} \hline
Parameter Name & Units & Type & Default & Description \\ \hline 
L & $M$ & double &  0.0 & arc length  \\ \hline 
ANGLE & $RAD$ & double &  0.0 & bend angle  \\ \hline 
K1 & $1/M^{2}$ & double &  0.0 & geometric quadrupole strength  \\ \hline 
K2 & $1/M^{3}$ & double &  0.0 & geometric sextupole strength  \\ \hline 
K3 & $1/M^{3}$ & double &  0.0 & geometric octupole strength  \\ \hline 
K4 & $1/M^{4}$ & double &  0.0 & geometric decapole strength  \\ \hline 
E1 & $RAD$ & double &  0.0 & entrance edge angle  \\ \hline 
E2 & $RAD$ & double &  0.0 & exit edge angle  \\ \hline 
TILT & $RAD$ & double &  0.0 & rotation about incoming longitudinal axis  \\ \hline 
H1 & $1/M$ & double &  0.0 & entrance pole-face curvature  \\ \hline 
H2 & $1/M$ & double &  0.0 & exit pole-face curvature  \\ \hline 
HGAP & $M$ & double &  0.0 & half-gap between poles  \\ \hline 
FINT &  & double &   0.5 & edge-field integral  \\ \hline 
DX & $M$ & double &  0.0 & misalignment  \\ \hline 
DY & $M$ & double &  0.0 & misalignment  \\ \hline 
DZ & $M$ & double &  0.0 & misalignment  \\ \hline 
FSE &  & double &  0.0 & fractional strength error  \\ \hline 
ETILT &  & double &  0.0 & error rotation about incoming longitudinal axis  \\ \hline 
N\_KICKS &  & long &   4               & number of kicks  \\ \hline 
NONLINEAR &  & long &   1               & include nonlinear field components?  \\ \hline 
LINEARIZE &  & long &  \verb|0| & use linear matrix instead of symplectic integrator?  \\ \hline 
SYNCH\_RAD &  & long &  \verb|0| & include classical synchrotron radiation?  \\ \hline 
EDGE1\_EFFECTS &  & long &   1               & include entrace edge effects?  \\ \hline 
EDGE2\_EFFECTS &  & long &   1               & include exit edge effects?  \\ \hline 
EDGE\_ORDER &  & long &   1               & order to which to include edge effects  \\ \hline 
INTEGRATION\_ORDER &  & long &   2               & integration order (2 or 4)  \\ \hline 
\end{tabular}

\begin{latexonly}
\newpage
\begin{center}{\Large\verb|CSRCSBEND| continued}\end{center}
\end{latexonly}
Like CSBEND, but incorporates a simulation of Coherent Synchrotron radiation.
\\
\begin{tabular}{|l|l|l|l|p{\descwidth}|} \hline
Parameter Name & Units & Type & Default & Description \\ \hline 
BINS &  & long &  \verb|0| & number of bins for CSR wake  \\ \hline 
BIN\_ONCE &  & long &  \verb|0| & bin only at the start of the dipole?  \\ \hline 
BIN\_RANGE\_FACTOR &  & double &   1.2 & Factor by which to increase the range of histogram compared to total bunch length.  Large value eliminates binning problems in CSRDRIFTs.  \\ \hline 
SG\_HALFWIDTH &  & long &  \verb|0| & Savitzky-Golay filter half-width for smoothing current histogram  \\ \hline 
SG\_ORDER &  & long &   1               & Savitzky-Golay filter order for smoothing current histogram  \\ \hline 
SGDERIV\_HALFWIDTH &  & long &  \verb|0| & Savitzky-Golay filter half-width for taking derivative of current histogram  \\ \hline 
SGDERIV\_ORDER &  & long &   1               & Savitzky-Golay filter order for taking derivative of current histogram  \\ \hline 
TRAPAZOID\_INTEGRATION &  & long &   1               & Select whether to use trapazoid-rule integration (default) or a simple sum.  \\ \hline 
OUTPUT\_FILE &  & STRING &   NULL            & output file for CSR wakes  \\ \hline 
OUTPUT\_INTERVAL &  & long &   1               & interval (in kicks) of output to OUTPUT\_FILE  \\ \hline 
OUTPUT\_LAST\_WAKE\_ONLY &  & long &  \verb|0| & output final wake only?  \\ \hline 
STEADY\_STATE &  & long &  \verb|0| & use steady-state wake equations?  \\ \hline 
USE\_BN &  & long &  \verb|0| & use B$<$n$>$ instead of K$<$n$>$?  \\ \hline 
B1 & $1/M$ & double &  0.0 & K1 = B1*rho, where rho is bend radius  \\ \hline 
B2 & $1/M^{2}$ & double &  0.0 & K2 = B2*rho  \\ \hline 
B3 & $1/M^{3}$ & double &  0.0 & K3 = B3*rho  \\ \hline 
B4 & $1/M^{4}$ & double &  0.0 & K4 = B4*rho  \\ \hline 
ISR &  & long &  \verb|0| & include incoherent synchrotron radiation (scattering)?  \\ \hline 
\end{tabular}

\begin{latexonly}
\newpage
\begin{center}{\Large\verb|CSRCSBEND| continued}\end{center}
\end{latexonly}
Like CSBEND, but incorporates a simulation of Coherent Synchrotron radiation.
\\
\begin{tabular}{|l|l|l|l|p{\descwidth}|} \hline
Parameter Name & Units & Type & Default & Description \\ \hline 
CSR &  & long &   1               & enable CSR computations?  \\ \hline 
BLOCK\_CSR &  & long &  \verb|0| & block CSR from entering CSRDRIFT?  \\ \hline 
DERBENEV\_CRITERION\_MODE &  & STRING &   disable         & disable, evaluate, or enforce  \\ \hline 
PARTICLE\_OUTPUT\_FILE &  & STRING &   NULL            & name of file for phase-space output  \\ \hline 
PARTICLE\_OUTPUT\_INTERVAL &  & long &  \verb|0| & interval (in kicks) of output to PARTICLE\_OUTPUT\_FILE  \\ \hline 
SLICE\_ANALYSIS\_INTERVAL &  & long &  \verb|0| & interval (in kicks) of output to slice analysis file (from slice\_analysis command)  \\ \hline 
HIGH\_FREQUENCY\_CUTOFF0 &  & double &   -1 & Spatial frequency at which smoothing filter begins.  If not positive, no frequency filter smoothing is done.  Frequency is in units of Nyquist (0.5/binsize).  \\ \hline 
HIGH\_FREQUENCY\_CUTOFF1 &  & double &   -1 & Spatial frequency at which smoothing filter is 0.  If not given, defaults to HIGH\_FREQUENCY\_CUTOFF0.  \\ \hline 
\end{tabular}

\vspace*{0.5in}
For a discussion of the method behind this element, see M. Borland,
``Simple method for particle tracking with coherent synchrotron
radiation,'' Phys. Rev. ST Accel. Beams 4, 070701 (2001) and
G. Stupakov and P. Emma, SLAC LCLS-TN-01-12 (2001).

{\bf Recommendations for using this element.}  The default values for
this element are not the best ones to use.  They are retained only for
consistency through upgrades.  In using this element, it is
recommended to have 50 to 100 k particle in the simulation.  Setting
{\tt BINS=600} and {\tt SG\_HALFWIDTH=1} is also recommended to allow
resolution of fine structure in the beam and to avoid excessive
smoothing.  It is strongly suggested that the user vary these
parameters and view the histogram output to verify that the
longitudinal distribution is well represented by the histograms (use
{\tt OUTPUT\_FILE} to obtain the histograms).  For LCLS simulations,
we find that the above parameters give essentially the same results as
obtained with 500 k particles and up to 3000 bins.

In order to verify that the 1D approximation is valid, the user should
also set {\tt DERBENEV\_CRITERION\_MODE = ``evaluate''} and view
the data in {\tt OUTPUT\_FILE}.  Generally, the criterion should be
much less than 1.  See equation 11 of \cite{Derbenev}.

In order respects, this element is just like the {\tt CSBEND} element,
which provides a symplectic bending magnet that is accurate to all
orders in momentum offset. Please see the manual page for {\tt CSBEND}
for more details about features not related to CSR.

{\bf Splitting dipoles}: 
For versions 19.X and ealier splitting dipoles is {\em not} recommended for {\tt
CSRCSBEND} because the coherent synchrotron radiation computations
start over at the beginning of each piece.  This is only acceptable
when using \verb|STEADY_STATE=1|.  This was changed in version 20, so that
for this and later versions splitting will work correctly with all CSR modes.


\begin{latexonly}
\newpage
\begin{center}{\Large\verb|CSRDRIFT|}\end{center}
\end{latexonly}\subsection{CSRDRIFT}
A follow-on element for CSRCSBEND that applies the CSR wake over a drift.
\\
\begin{tabular}{|l|l|l|l|p{\descwidth}|} \hline
Parameter Name & Units & Type & Default & Description \\ \hline 
L & $M$ & double &  0.0 & length  \\ \hline 
ATTENUATION\_LENGTH & $M$ & double &  0.0 & exponential attenuation length for wake  \\ \hline 
DZ &  & double &  0.0 & interval between kicks  \\ \hline 
N\_KICKS &  & long &   1               & number of kicks (if DZ is zero)  \\ \hline 
SPREAD &  & long &  \verb|0| & use spreading function?  \\ \hline 
USE\_OVERTAKING\_LENGTH &  & long &  \verb|0| & use overtaking length for ATTENUATION\_LENGTH?  \\ \hline 
OL\_MULTIPLIER &  & double &   1 & factor by which to multiply the overtaking length to get the attenuation length  \\ \hline 
USE\_SALDIN54 &  & long &  \verb|0| & Use Saldin et al eq. 54 (NIM A 398 (1997) 373-394 for decay vs z?  \\ \hline 
SALDIN54POINTS &  & long &   1000            & Number of values of position inside bunch to average for Saldin eq 54.  \\ \hline 
CSR &  & long &   1               & do CSR calcuations  \\ \hline 
SALDIN54NORM\_MODE &  & STRING &   peak            & peak or first  \\ \hline 
SPREAD\_MODE &  & STRING &   full            & full, simple, or radiation-only  \\ \hline 
WAVELENGTH\_MODE &  & STRING &   sigmaz          & sigmaz or peak-to-peak  \\ \hline 
BUNCHLENGTH\_MODE &  & STRING &   68-percentile   & rms, 68-percentile, or 90-percentile  \\ \hline 
SALDIN54\_OUTPUT &  & STRING &   NULL            & Filename for output of CSR intensity vs. z as computed using Saldin eq 54.  \\ \hline 
USE\_STUPAKOV &  & long &  \verb|0| & Use treatment from G. Stupakov's note of 9/12/2001?  \\ \hline 
STUPAKOV\_OUTPUT &  & STRING &   NULL            & Filename for output of CSR wake vs. s as computed using Stupakov's equations.  \\ \hline 
STUPAKOV\_OUTPUT\_INTERVAL &  & long &   1               & Interval (in kicks) between output of Stupakov wakes.  \\ \hline 
SLICE\_ANALYSIS\_INTERVAL &  & long &  \verb|0| & interval (in kicks) of output to slice analysis file (from slice\_analysis command)  \\ \hline 
\end{tabular}

\begin{latexonly}
\newpage
\begin{center}{\Large\verb|CSRDRIFT| continued}\end{center}
\end{latexonly}
A follow-on element for CSRCSBEND that applies the CSR wake over a drift.
\\
\begin{tabular}{|l|l|l|l|p{\descwidth}|} \hline
Parameter Name & Units & Type & Default & Description \\ \hline 
LINEARIZE &  & long &  \verb|0| & use linear optics for drift pieces?  \\ \hline 
\end{tabular}

\vspace*{0.5in}
This element has a number of models for simulation of CSR in drift
spaces following CSRCSBEND elements.  Note that all models allow
support splitting the drift into multiple CSRDRIFT elements.
One can also have intervening elements like quadrupoles,
as often happens in chicanes.  The CSR effects inside such
intervening elements are applied in the CSRDRIFT downstream of
the element.

For a discussion of some of the methods behind this element, see
M. Borland, ``Simple method for particle tracking with coherent
synchrotron radiation,'' Phys. Rev. ST Accel. Beams 4, 070701 (2001).

{\bf N.B.}: by default, this element uses 1 CSR kick (N\_KICKS=1) at the
center of the drift.  This is usually not a good choice.  I usually
use the DZ parameter instead of N\_KICKS, and set it to something
like 0.01 (meters).  The user should vary this parameter to assess
how small it needs to be.

The models are as following, in order of decreasing sophistication and accuracy:
\begin{itemize}
\item G. Stupakov's extension of Saldin et al.  Set USE\_STUPAKOV=1.
The most advanced model at present is based on a private communication
from G. Stupakov (SLAC), which extends equation 87 of the one-dimensional
treatment of Saldin et al. (NIM A 398 (1997) 373-394) to include the
post-dipole region.  This model includes not only the attenuation of the
CSR as one proceeds along the drift, but also the change in the shape of
the ``wake.''

This model has the most sophisticated treatment for intervening
elements of any of the models.  For example, if you have a sequence
{\tt CSRCSBEND}-{\tt CSRDRIFT}-{\tt CSRDRIFT} and compare it with the
sequence {\tt CSRCSBEND}-{\tt CSRDRIFT}-{\tt DRIFT} -{\tt CSRDRIFT},
keeping the total drift length constant, you'll find no change in the
CSR-induced energy modulation.  The model back-propagates to the
beginning of the intervening element and performs the CSR computations
starting from there.

This is the slowest model to run.  It uses the same binning and
smoothing parameters as the upstream CSRCSBEND.  If run time is a
problem, compare it to the other models and use only if you get
different answers.

\item M. Borland's model based on Saldin et al. equations 53 and 54.
Set USE\_SALDIN54=1.  This model computes the fall-off of the CSR wake
from the work of Saldin and coworkers, as described in the reference
above.  It does not compute the change in the shape of the wake.  The
fall-off is computed approximately as well, based on the fall-off for
a rectangular current distribution.  The length of this rectangular
bunch is taken to be twice the bunch length computed according to the
BUNCHLENGTH\_MODE parameter (see below).  If your bunch is nearly
rectangular, then you probably want BUNCHLENGTH\_MODE of
``90-percentile''.

\item Exponential attenuation of a CSR wake with unchanging shape.
There are two options here.  First, you can provide the attenuation
length yourself, using the ATTENUATION\_LENGTH parameter.  Second, you
can set USE\_OVERTAKING\_LENGTH=1 and let {\tt elegant} compute the
overtaking length for use as the attenuation length.  In addition, you
can multiply this result by a factor if you wish, using the
OL\_MULTIPLIER parameter.

\item Beam-spreading model.  This model is not recommended.  It is
based on the seemingly plausible idea that CSR spreads out just like
any synchrotron radiation, thus decreasing the intensity.  The model
doesn't reproduce experiments.

\end{itemize}

The ``Saldin 54'' and ``overtaking-length'' models rely on computation
of the bunch length, which is controlled with the BUNCHLENGTH\_MODE
parameter.  Nominally, one should use the true RMS, but when the beam
has temporal spikes, it isn't always clear that this is the best
choice.  The choices are ``rms'', ``68-percentile'', and
``90-percentile''.  The last two imply using half the length
determined from the given percentile in place of the rms bunch length.
I usually use 68-percentile, which is the default.


\begin{latexonly}
\newpage
\begin{center}{\Large\verb|DRIF|}\end{center}
\end{latexonly}\subsection{DRIF}
A drift space implemented as a matrix, up to 2nd order
\\
\begin{tabular}{|l|l|l|l|p{\descwidth}|} \hline
Parameter Name & Units & Type & Default & Description \\ \hline 
L & $M$ & double &  0.0 & length  \\ \hline 
ORDER &  & long &  \verb|0| & matrix order  \\ \hline 
\end{tabular}

\begin{latexonly}
\newpage
\begin{center}{\Large\verb|ECOL|}\end{center}
\end{latexonly}\subsection{ECOL}
An elliptical collimator.
\\
\begin{tabular}{|l|l|l|l|p{\descwidth}|} \hline
Parameter Name & Units & Type & Default & Description \\ \hline 
L & $M$ & double &  0.0 & length  \\ \hline 
X\_MAX & $M$ & double &  0.0 & half-axis in x  \\ \hline 
Y\_MAX & $M$ & double &  0.0 & half-axis in y  \\ \hline 
DX & $M$ & double &  0.0 & misalignment  \\ \hline 
DY & $M$ & double &  0.0 & misalignment  \\ \hline 
\end{tabular}

\begin{latexonly}
\newpage
\begin{center}{\Large\verb|ELSE|}\end{center}
\end{latexonly}\subsection{ELSE}
Not implemented.
\\
\begin{tabular}{|l|l|l|l|p{\descwidth}|} \hline
Parameter Name & Units & Type & Default & Description \\ \hline 
\end{tabular}

\begin{latexonly}
\newpage
\begin{center}{\Large\verb|ENERGY|}\end{center}
\end{latexonly}\subsection{ENERGY}
An element that matches the central momentum to the beam momentum, or changes
the central momentum or energy to a specified value.
\\
\begin{tabular}{|l|l|l|l|p{\descwidth}|} \hline
Parameter Name & Units & Type & Default & Description \\ \hline 
CENTRAL\_ENERGY & $MC^{2}$ & double &  0.0 & desired central gamma  \\ \hline 
CENTRAL\_MOMENTUM & $MC$ & double &  0.0 & desired central beta*gamma  \\ \hline 
MATCH\_BEAMLINE &  & long &  \verb|0| & if nonzero, beamline reference momentum is set to beam average momentum  \\ \hline 
MATCH\_PARTICLES &  & long &  \verb|0| & if nonzero, beam average momentum is set to beamline reference momentum  \\ \hline 
\end{tabular}

\begin{latexonly}
\newpage
\begin{center}{\Large\verb|FMULT|}\end{center}
\end{latexonly}\subsection{FMULT}
Multipole kick element with coefficient input from an SDDS file.
\\
\begin{tabular}{|l|l|l|l|p{\descwidth}|} \hline
Parameter Name & Units & Type & Default & Description \\ \hline 
L & $M$ & double &  0.0 & length  \\ \hline 
TILT & $RAD$ & double &  0.0 & rotation about longitudinal axis  \\ \hline 
DX & $M$ & double &  0.0 & misalignment  \\ \hline 
DY & $M$ & double &  0.0 & misalignment  \\ \hline 
DZ & $M$ & double &  0.0 & misalignment  \\ \hline 
FSE &  & double &  0.0 & fractional strength error  \\ \hline 
N\_KICKS &  & long &   1               & number of kicks  \\ \hline 
SYNCH\_RAD &  & long &  \verb|0| & include classical synchrotron radiation?  \\ \hline 
FILENAME &  & STRING &   NULL            & name of file containing multipole data  \\ \hline 
\end{tabular}

\vspace*{0.5in}
\begin{raggedright}
This element simulates a multipole element using a 4th-order sympletic
integration.  
Specification of the multipole strength is through an SDDS file.
The file is expected to contain a single page of
data with the following elements:
\end{raggedright}
\begin{enumerate}
\item An integer column named {\tt order} giving the order of the multipole.
The order is defined as $(N_{poles}-2)/2$, so a quadrupole has order 1, a
sextupole has order 2, and so on.
\item A floating point column named {\tt KnL} giving the integrated strength of
the multipole, $K_n L$, where $n$ is the order.  The units are $1/m^(n-1)$.
\item A floating point column named {\tt JnL} giving the integrated strength of
the skew multipole, $J_n L$, where $n$ is the order.  The units are $1/m^(n-1)$.
\end{enumerate}

The {\tt MULT} element is also available, which allows the same
functionality without an external file.

\begin{latexonly}
\newpage
\begin{center}{\Large\verb|HISTOGRAM|}\end{center}
\end{latexonly}\subsection{HISTOGRAM}
Request for histograms of particle coordinates to be output to SDDS file.
\\
\begin{tabular}{|l|l|l|l|p{\descwidth}|} \hline
Parameter Name & Units & Type & Default & Description \\ \hline 
FILENAME &  & STRING &                   & filename for histogram output  \\ \hline 
INTERVAL &  & long &   1               & interval in passes between output  \\ \hline 
START\_PASS &  & long &  \verb|0| & starting pass for output  \\ \hline 
BINS &  & long &   50              & number of bins  \\ \hline 
FIXED\_BIN\_SIZE &  & long &  \verb|0| & if nonzero, bin size is fixed at given value  \\ \hline 
X\_DATA &  & long &   1               & histogram x and x'?  \\ \hline 
Y\_DATA &  & long &   1               & histogram y and y'?  \\ \hline 
LONGIT\_DATA &  & long &   1               & histogram t and p?  \\ \hline 
BIN\_SIZE\_FACTOR &  & double &   1 & multiply computed bin size by this factor before histogramming  \\ \hline 
\end{tabular}

\begin{latexonly}
\newpage
\begin{center}{\Large\verb|HKICK|}\end{center}
\end{latexonly}\subsection{HKICK}
A horizontal steering dipole implemented as a matrix, up to 2nd order.
\\
\begin{tabular}{|l|l|l|l|p{\descwidth}|} \hline
Parameter Name & Units & Type & Default & Description \\ \hline 
L & $M$ & double &  0.0 & length  \\ \hline 
KICK & $RAD$ & double &  0.0 & kick strength  \\ \hline 
TILT & $RAD$ & double &  0.0 & rotation about longitudinal axis  \\ \hline 
B2 & $1/M^{2}$ & double &  0.0 & normalized sextupole strength (kick = KICK*(1+B2*x\^2))  \\ \hline 
CALIBRATION &  & double &   1 & strength multiplier  \\ \hline 
EDGE\_EFFECTS &  & long &  \verb|0| & include edge effects?  \\ \hline 
ORDER &  & long &  \verb|0| & matrix order  \\ \hline 
STEERING &  & long &   1               & use for steering?  \\ \hline 
\end{tabular}

\begin{latexonly}
\newpage
\begin{center}{\Large\verb|HMON|}\end{center}
\end{latexonly}\subsection{HMON}
A horizontal position monitor, accepting a rpn equation for the readout as a
function of the actual position (x).
\\
\begin{tabular}{|l|l|l|l|p{\descwidth}|} \hline
Parameter Name & Units & Type & Default & Description \\ \hline 
L & $M$ & double &  0.0 & length  \\ \hline 
DX & $M$ & double &  0.0 & misalignment  \\ \hline 
DY & $M$ & double &  0.0 & misalignment  \\ \hline 
WEIGHT &  & double &   1 & weight in correction  \\ \hline 
TILT &  & double &  0.0 & rotation about longitudinal axis  \\ \hline 
CALIBRATION &  & double &   1 & calibration factor for readout  \\ \hline 
ORDER &  & long &  \verb|0| & matrix order  \\ \hline 
READOUT &  & STRING &   NULL            & rpn expression for readout (actual position supplied in variable x)  \\ \hline 
\end{tabular}

\begin{latexonly}
\newpage
\begin{center}{\Large\verb|IBSCATTER|}\end{center}
\end{latexonly}\subsection{IBSCATTER}
A simulation of intra-beam scattering.
\\
\begin{tabular}{|l|l|l|l|p{\descwidth}|} \hline
Parameter Name & Units & Type & Default & Description \\ \hline 
COUPLING &  & double &   1 & x-y coupling  \\ \hline 
FACTOR &  & double &   1 & factor by which to multiply growth rates before using  \\ \hline 
CHARGE & $C$ & double &  0.0 & beam charge (or use CHARGE element)  \\ \hline 
DO\_X &  & long &   1               & do x-plane scattering?  \\ \hline 
DO\_Y &  & long &   1               & do y-plane scattering?  \\ \hline 
DO\_Z &  & long &   1               & do z-plane scattering?  \\ \hline 
SMOOTH &  & long &  \verb|0| & Use smooth method instead of random numbers?  \\ \hline 
\end{tabular}

\begin{latexonly}
\newpage
\begin{center}{\Large\verb|KICKER|}\end{center}
\end{latexonly}\subsection{KICKER}
A combined horizontal-vertical steering magnet implemented as a matrix, up to
2nd order.
\\
\begin{tabular}{|l|l|l|l|p{\descwidth}|} \hline
Parameter Name & Units & Type & Default & Description \\ \hline 
L & $M$ & double &  0.0 & length  \\ \hline 
HKICK & $RAD$ & double &  0.0 & x kick angle  \\ \hline 
VKICK & $RAD$ & double &  0.0 & y kick angle  \\ \hline 
TILT & $RAD$ & double &  0.0 & rotation about longitudinal axis  \\ \hline 
B2 & $1/M^{2}$ & double &  0.0 & normalized sextupole strength (e.g., kick = KICK*(1+B2*x\^2))  \\ \hline 
HCALIBRATION &  & double &   1 & factor applied to obtain x kick  \\ \hline 
VCALIBRATION &  & double &   1 & factor applied to obtain y kick  \\ \hline 
EDGE\_EFFECTS &  & long &  \verb|0| & include edge effects?  \\ \hline 
ORDER &  & long &  \verb|0| & matrix order  \\ \hline 
STEERING &  & long &   1               & use for steering?  \\ \hline 
\end{tabular}

\begin{latexonly}
\newpage
\begin{center}{\Large\verb|KPOLY|}\end{center}
\end{latexonly}\subsection{KPOLY}
A thin kick element with polynomial dependence on the coordinates in one plane.
\\
\begin{tabular}{|l|l|l|l|p{\descwidth}|} \hline
Parameter Name & Units & Type & Default & Description \\ \hline 
COEFFICIENT & $M^{-ORDER}$ & double &  0.0 & coefficient of polynomial  \\ \hline 
TILT & $RAD$ & double &  0.0 & rotation about longitudinal axis  \\ \hline 
DX & $M$ & double &  0.0 & misalignment  \\ \hline 
DY & $M$ & double &  0.0 & misalignment  \\ \hline 
DZ & $M$ & double &  0.0 & misalignment  \\ \hline 
FACTOR &  & double &   1 & additional factor to apply  \\ \hline 
ORDER &  & long &  \verb|0| & order of polynomial  \\ \hline 
PLANE &  & STRING &   x               & plane to kick (x, y)  \\ \hline 
\end{tabular}

\begin{latexonly}
\newpage
\begin{center}{\Large\verb|KQUAD|}\end{center}
\end{latexonly}\subsection{KQUAD}
A canonical kick quadrupole, which differs from the MULT element with ORDER=1 in
that it can be used for tune correction.
\\
\begin{tabular}{|l|l|l|l|p{\descwidth}|} \hline
Parameter Name & Units & Type & Default & Description \\ \hline 
L & $M$ & double &  0.0 & length  \\ \hline 
K1 & $1/M^{2}$ & double &  0.0 & geometric strength  \\ \hline 
TILT & $RAD$ & double &  0.0 & rotation about longitudinal axis  \\ \hline 
BORE & $M$ & double &  0.0 & bore radius  \\ \hline 
B & $T$ & double &  0.0 & pole tip field (used if bore nonzero)  \\ \hline 
DX & $M$ & double &  0.0 & misalignment  \\ \hline 
DY & $M$ & double &  0.0 & misalignment  \\ \hline 
DZ & $M$ & double &  0.0 & misalignment  \\ \hline 
FSE & $M$ & double &  0.0 & fractional strength error  \\ \hline 
N\_KICKS &  & long &   4               & number of kicks  \\ \hline 
SYNCH\_RAD &  & long &  \verb|0| & include classical synchrotron radiation?  \\ \hline 
SYSTEMATIC\_MULTIPOLES &  & STRING &   NULL            & input file for systematic multipoles  \\ \hline 
RANDOM\_MULTIPOLES &  & STRING &   NULL            & input file for random multipoles  \\ \hline 
INTEGRATION\_ORDER &  & long &   4               & integration order (2 or 4)  \\ \hline 
\end{tabular}

\vspace*{0.5in}
This element simulates a quadrupole using a kick method based on
symplectic integration.  The user specifies the number of kicks and
the order of the integration.  For computation of twiss parameters and
response matrices, this element is treated like a standard thick-lens
quadrupole; i.e., the number of kicks and the integration order become
irrelevant.

\begin{raggedright}
Specification of systematic and random multipole errors is supported
through the \verb|SYSTEMATIC_MULTIPOLES|, \verb|EDGE_MULTIPOLES|, and 
\verb|RANDOM_MULTIPOLES|
fields.  These specify, respectively, fixed multipole strengths for the
body of the element, fixed multipole strengths for the edges of the element,
and random multipole strengths for the body of the element.
These fields give the names of SDDS files that supply the
multipole data.  The files are expected to contain a single page of
data with the following elements:
\end{raggedright}
\begin{enumerate}
\item Floating point parameter {\tt referenceRadius} giving the reference
 radius for the multipole data.
\item An integer column named {\tt order} giving the order of the multipole.
The order is defined as $(N_{poles}-2)/2$, so a quadrupole has order 1, a
sextupole has order 2, and so on.
\item Floating point columns {\tt normal} and {\tt skew} giving the values for the
normal and skew multipole strengths, respectively.  
(N.B.: previous versions used the names {\tt an} and {\tt bn}, respectively. This is still accepted but deprecated)
These are defined as a fraction 
of the main field strength measured at the reference radius, R: 
$f_n  = \frac{K_n R^n / n!}{K_m R^m / m!}$, where 
$m=1$ is the order of the main field and $n$ is the order of the error multipole.
A similar relationship holds for the skew multipole fractional strengths.
For random multipoles, the values are interpreted as rms values for the distribution.
\end{enumerate}

Specification of systematic higher multipoles due to steering fields is
supported through the \verb|STEERING_MULTIPOLES| field.  This field gives the
name of an SDDS file that supplies the multipole data.  The file is
expected to contain a single page of data with the following elements:
\begin{enumerate}
\item Floating point parameter {\tt referenceRadius} giving the reference
 radius for the multipole data.
\item An integer column named {\tt order} giving the order of the multipole.
The order is defined as $(N_{poles}-2)/2$.  The order must be an even number
because of the quadrupole symmetry.
\item Floating point column {\tt normal} giving the values for the normal
multipole strengths, which are driven by the horizontal steering field.
(N.B.: previous versions used the name {\tt an} for this data. This is still accepted but deprecated)
{\tt normal} is specifies the multipole strength as a fraction $f_n$ of the steering field strength measured at the reference radius, R: 
$f_n = \frac{K_n R^n / n!}{K_m R^m / m!}$, where 
$m=0$ is the order of the steering field and $n$ is the order of the error multipole.
The skew values (for vertical steering) are deduced from the {\tt normal} values, specifically,
$g_n = f_n*(-1)^{n/2}$.
\end{enumerate}

The dominant systematic multipole term in the steering field is a
sextupole.  Note that {\tt elegant} presently {\em does not} include
such sextupole contributions in the computation of the chromaticity
via the {\tt twiss\_output} command.  However, these chromatic effects
will be seen in tracking.

Apertures specified via an upstream \verb|MAXAMP| element or an \verb|aperture_input|
command will be imposed inside this element.

As of version 29.2, this element incorporates the ability to have different values for the insertion
and effective lengths. This is invoked when \verb|LEFFECTIVE| is positive. In this case, the
\verb|L| parameter is understood to be the physical insertion length. Using \verb|LEFFECTIVE| is
a convenient way to incorporate the fact that the effective length may differ from the physical
length and even vary with excitation, without having to modify the drift spaces on either side of
the quadrupole element.

Fringe field effects  are based on publications of D.  Zhuo {\em et al.} \cite{Zhou-IPAC10} and  J. Irwin {\em et
  al.} \cite{Irwin-PAC95}, as well as unpublished work of C. X. Wang (ANL).  The fringe field is characterized by 
10 integrals given in equations 19, 20, and 21 of \cite{Zhou-IPAC10}.  However, the values input into {\tt elegant}
should be normalized by $K_1$ or $K_1^2$, as appropriate.

For the exit-side fringe field, let $s_1$ be the center of the magnet, $s_0$ be the location of the nominal end of the magnet
(for a hard-edge model), and let $s_2$ be a point well outside the magnet.  
Using $K_{1,he}(s)$ to represent the hard edge model and $K_1(s)$ the actual field profile, we 
define the normalized difference as $\tilde{k}(s) = (K_1(s) - K_{1,he}(s))/K_1(s_1)$.  (Thus, $\tilde{k}(s) = \tilde{K}(s)/K_0$, using
the notation of Zhou {\em et al.})

The integrals to be input to {\tt elegant} are defined as 
\begin{eqnarray}
i_0^- = \int_{s_1}^{s_0} \tilde{k}(s) ds & & i_0^+ = \int_{s_0}^{s_2} \tilde{k}(s) ds \\
i_1^- = \int_{s_1}^{s_0} \tilde{k}(s) (s-s_0) ds & & i_1^+ = \int_{s_0}^{s_2} \tilde{k}(s) (s-s_0) ds \\
i_2^- = \int_{s_1}^{s_0} \tilde{k}(s) (s-s_0)^2 ds & & i_2^+ = \int_{s_0}^{s_2} \tilde{k}(s) (s-s_0)^2 ds \\
i_3^- = \int_{s_1}^{s_0} \tilde{k}(s) (s-s_0)^3 ds & & i_3^+ = \int_{s_0}^{s_2} \tilde{k}(s) (s-s_0)^3 ds \\
\lambda_2^- = \int_{s_1}^{s_0} ds \int_s^{s_0} ds^\prime \tilde{k}(s) \tilde{k}(s^\prime) (s^\prime-s) & & 
\lambda_2^+ = \int_{s_0}^{s_2} ds \int_s^{s_2} ds^\prime \tilde{k}(s) \tilde{k}(s^\prime) (s^\prime-s) 
\end{eqnarray}

Normally, the effects are dominated by $i_1^-$ and $i_1^+$.  The script \verb|computeQuadFringeIntegrals|,
packaged with \verb|elegant|, allows computing these integrals and the effective length if provided with 
data giving the gradient vs s.

The \verb|EDGE1_EFFECTS| and \verb|EDGE2_EFFECTS| parameters can be used to turn fringe field effects on and off, but also
to control the order of the implementation.  If the value is 1, linear fringe effects are included.  If the value is 2, 
leading-order (cubic) nonlinear effects are included.  If the value is 3 or higher, higher order effects are included.

In order to improve performance, the horizontal and vertical steering kicks are only applied at the entrance and exit
of the element. E.g., if a horizontal kick of $\Delta x^\prime$ is specified, $\Delta x^\prime/2$ is applied at
the entrance and at the exit.


\begin{latexonly}
\newpage
\begin{center}{\Large\verb|KSBEND|}\end{center}
\end{latexonly}\subsection{KSBEND}
A kick bending magnet which is NOT canonical, but is better than a 2nd order
matrix implementation.  Recommend using CSBEND instead.
\\
\begin{tabular}{|l|l|l|l|p{\descwidth}|} \hline
Parameter Name & Units & Type & Default & Description \\ \hline 
L & $M$ & double &  0.0 & arc length  \\ \hline 
ANGLE & $RAD$ & double &  0.0 & bend angle  \\ \hline 
K1 & $1/M^{2}$ & double &  0.0 & geometric quadrupole strength  \\ \hline 
K2 & $1/M^{3}$ & double &  0.0 & geometric sextupole strength  \\ \hline 
K3 & $1/M^{3}$ & double &  0.0 & geometric octupole strength  \\ \hline 
K4 & $1/M^{4}$ & double &  0.0 & geometric decapole strength  \\ \hline 
E1 & $RAD$ & double &  0.0 & entrance edge angle  \\ \hline 
E2 & $RAD$ & double &  0.0 & exit edge angle  \\ \hline 
TILT & $RAD$ & double &  0.0 & rotation about incoming longitudinal axis  \\ \hline 
H1 & $1/M$ & double &  0.0 & entrance pole-face curvature  \\ \hline 
H2 & $1/M$ & double &  0.0 & exit pole-face curvature  \\ \hline 
HGAP & $M$ & double &  0.0 & half-gap between poles  \\ \hline 
FINT &  & double &   0.5 & edge-field integral  \\ \hline 
DX & $M$ & double &  0.0 & misalignment  \\ \hline 
DY & $M$ & double &  0.0 & misalignment  \\ \hline 
DZ & $M$ & double &  0.0 & misalignment  \\ \hline 
FSE &  & double &  0.0 & fractional strength error  \\ \hline 
ETILT &  & double &  0.0 & error rotation about incoming longitudinal axis  \\ \hline 
N\_KICKS &  & long &   4               & number of kicks  \\ \hline 
NONLINEAR &  & long &   1               & include nonlinear field components?  \\ \hline 
SYNCH\_RAD &  & long &  \verb|0| & include classical synchrotron radiation?  \\ \hline 
EDGE1\_EFFECTS &  & long &   1               & include entrace edge effects?  \\ \hline 
EDGE2\_EFFECTS &  & long &   1               & include exit edge effects?  \\ \hline 
EDGE\_ORDER &  & long &   1               & edge matrix order  \\ \hline 
PARAXIAL &  & long &  \verb|0| & use paraxial approximation?  \\ \hline 
TRANSPORT &  & long &  \verb|0| & use (incorrect) TRANSPORT equations for T436 of edge?  \\ \hline 
METHOD &  & STRING &   modified-midpoint & integration method (modified-midpoint, leap-frog  \\ \hline 
\end{tabular}

\begin{latexonly}
\newpage
\begin{center}{\Large\verb|KSEXT|}\end{center}
\end{latexonly}\subsection{KSEXT}
A canonical kick sextupole, which differs from the MULT element with ORDER=2 in
that it can be used for chromaticity correction.
\\
\begin{tabular}{|l|l|l|l|p{\descwidth}|} \hline
Parameter Name & Units & Type & Default & Description \\ \hline 
L & $M$ & double &  0.0 & length  \\ \hline 
K2 & $1/M^{3}$ & double &  0.0 & geometric strength  \\ \hline 
TILT & $RAD$ & double &  0.0 & rotation about longitudinal axis  \\ \hline 
BORE & $M$ & double &  0.0 & bore radius  \\ \hline 
B & $T$ & double &  0.0 & field at pole tip (used if bore nonzero)  \\ \hline 
DX & $M$ & double &  0.0 & misalignment  \\ \hline 
DY & $M$ & double &  0.0 & misalignment  \\ \hline 
DZ & $M$ & double &  0.0 & misalignment  \\ \hline 
FSE & $M$ & double &  0.0 & fractional strength error  \\ \hline 
N\_KICKS &  & long &   4               & number of kicks  \\ \hline 
SYNCH\_RAD &  & long &  \verb|0| & include classical synchrotron radiation?  \\ \hline 
SYSTEMATIC\_MULTIPOLES &  & STRING &   NULL            & input file for systematic multipoles  \\ \hline 
RANDOM\_MULTIPOLES &  & STRING &   NULL            & input file for random multipoles  \\ \hline 
INTEGRATION\_ORDER &  & long &   4               & integration order (2 or 4)  \\ \hline 
\end{tabular}

\vspace*{0.5in}
This element simulates a sextupole using a kick method based on
symplectic integration.  The user specifies the number of kicks and
the order of the integration.  For computation of twiss parameters,
chromaticities, and response matrices, this element is treated like a
standard thick-lens sextuupole; i.e., the number of kicks and the
integration order become irrelevant.

\begin{raggedright}
Specification of systematic and random multipole errors is supported
through the \verb|SYSTEMATIC_MULTIPOLES| and 
\verb|RANDOM_MULTIPOLES|
fields.  These fields give the names of SDDS files that supply the
multipole data.  The files are expected to contain a single page of
data with the following elements:
\end{raggedright}
\begin{enumerate}
\item Floating point parameter {\tt referenceRadius} giving the reference
 radius for the multipole data.
\item An integer column named {\tt order} giving the order of the multipole.
The order is defined as $(N_{poles}-2)/2$, so a quadrupole has order 1, a
sextupole has order 2, and so on.
\item Floating point columns {\tt an} and {\tt bn} giving the values for the
normal and skew multipole strengths, respectively.  These are defined as a fraction 
of the main field strength measured at the reference radius, R: 
$a_n  = \frac{K_n r^n / n!}{K_m r^m / m!}$, where 
$m=2$ is the order of the main field and $n$ is the order of the error multipole.
A similar relationship holds for the skew multipoles.
For random multipoles, the values are interpreted as rms values for the distribution.
\end{enumerate}


\begin{latexonly}
\newpage
\begin{center}{\Large\verb|MAGNIFY|}\end{center}
\end{latexonly}\subsection{MAGNIFY}
An element that allows multiplication of phase-space coordinates of all particles
by constants.
\\
\begin{tabular}{|l|l|l|l|p{\descwidth}|} \hline
Parameter Name & Units & Type & Default & Description \\ \hline 
MX &  & double &   1 & factor for x coordinates  \\ \hline 
MXP &  & double &   1 & factor for x' coordinates  \\ \hline 
MY &  & double &   1 & factor for y coordinates  \\ \hline 
MYP &  & double &   1 & factor for y' coordinates  \\ \hline 
MS &  & double &   1 & factor for s coordinates  \\ \hline 
MDP &  & double &   1 & factor for (p-pCentral)/pCentral  \\ \hline 
\end{tabular}

\begin{latexonly}
\newpage
\begin{center}{\Large\verb|MALIGN|}\end{center}
\end{latexonly}\subsection{MALIGN}
A misalignment of the beam, implemented as a zero-order matrix.
\\
\begin{tabular}{|l|l|l|l|p{\descwidth}|} \hline
Parameter Name & Units & Type & Default & Description \\ \hline 
DXP &  & double &  0.0 & delta x'  \\ \hline 
DYP &  & double &  0.0 & delta y'  \\ \hline 
DX & $M$ & double &  0.0 & delta x  \\ \hline 
DY & $M$ & double &  0.0 & delta y  \\ \hline 
DZ & $M$ & double &  0.0 & delta z  \\ \hline 
DT & $S$ & double &  0.0 & delta t  \\ \hline 
DP &  & double &  0.0 & delta p/pCentral  \\ \hline 
DE &  & double &  0.0 & delta gamma/gammaCentral  \\ \hline 
ON\_PASS &  & long &   -1              & pass on which to apply  \\ \hline 
FORCE\_MODIFY\_MATRIX &  & long &  \verb|0| & modify the matrix even if on\_pass$>$=0  \\ \hline 
\end{tabular}

\begin{latexonly}
\newpage
\begin{center}{\Large\verb|MAPSOLENOID|}\end{center}
\end{latexonly}\subsection{MAPSOLENOID}
A numerically-integrated solenoid specified as a map of (Bz, Br) vs (z, r).
\\
\begin{tabular}{|l|l|l|l|p{\descwidth}|} \hline
Parameter Name & Units & Type & Default & Description \\ \hline 
L & $M$ & double &  0.0 & length  \\ \hline 
DX & $M$ & double &  0.0 & misalignment  \\ \hline 
DY & $M$ & double &  0.0 & misalignment  \\ \hline 
ETILT & $RAD$ & double &  0.0 & misalignment  \\ \hline 
EYAW & $RAD$ & double &  0.0 & misalignment  \\ \hline 
EPITCH & $RAD$ & double &  0.0 & misalignment  \\ \hline 
N\_STEPS &  & long &   100             & number of steps (for nonadaptive integration)  \\ \hline 
INPUTFILE &  & STRING &   NULL            & SDDS file containing (Br, Bz) vs (r, z).  Each page should have values for a fixed r.  \\ \hline 
RCOLUMN &  & STRING &   NULL            & column containing r values  \\ \hline 
ZCOLUMN &  & STRING &   NULL            & column containing z values  \\ \hline 
BRCOLUMN &  & STRING &   NULL            & column containing Br values  \\ \hline 
BZCOLUMN &  & STRING &   NULL            & column containing Bz values  \\ \hline 
FACTOR &  & double &   0.0001 & factor by which to multiply fields in file  \\ \hline 
BXUNIFORM &  & double &  0.0 & uniform horizontal field to superimpose on solenoid field  \\ \hline 
BYUNIFORM &  & double &  0.0 & uniform vertical field to superimpose on solenoid field  \\ \hline 
LUNIFORM &  & double &  0.0 & length of uniform field superimposed on solenoid field  \\ \hline 
ACCURACY &  & double &   0.0001 & integration accuracy  \\ \hline 
METHOD & $ $ & STRING &   runge-kutta     & integration method (runge-kutta, bulirsch-stoer, non-adaptive runge-kutta, modified midpoint)  \\ \hline 
\end{tabular}

\begin{latexonly}
\newpage
\begin{center}{\Large\verb|MARK|}\end{center}
\end{latexonly}\subsection{MARK}
A marker, equivalent to a zero-length drift space.
\\
\begin{tabular}{|l|l|l|l|p{\descwidth}|} \hline
Parameter Name & Units & Type & Default & Description \\ \hline 
FITPOINT &  & long &  \verb|0| & supply Twiss parameters, moments, floor coordinates for optimization?  \\ \hline 
\end{tabular}

\begin{latexonly}
\newpage
\begin{center}{\Large\verb|MATR|}\end{center}
\end{latexonly}\subsection{MATR}
Explicit matrix input from a text file, in the format written by the print\_matrix
command.
\\
\begin{tabular}{|l|l|l|l|p{\descwidth}|} \hline
Parameter Name & Units & Type & Default & Description \\ \hline 
L & $M$ & double &  0.0 & length  \\ \hline 
FILENAME &  & STRING &                   & input file  \\ \hline 
ORDER &  & long &   1               & matrix order  \\ \hline 
\end{tabular}

\vspace*{0.5in}
The input file for this element uses a simple text format.  It is nearly identical
to the output in the {\tt printout} file generated by the {\tt matrix\_output}
and {\tt analyze\_map} commands.  For example, for a 1st-order matrix, the file would have the
following appearance:\\
C: {\em C1 C2 C3 C4 C5 C6}\\
R1: {\em R11 R12 R13 R14 R15 R16}\\
R2: {\em R21 R22 R23 R24 R25 R26}\\
R3: {\em R31 R32 R33 R34 R35 R36}\\
R4: {\em R41 R42 R43 R44 R45 R46}\\
R5: {\em R51 R52 R53 R54 R55 R56}\\
R6: {\em R61 R62 R63 R64 R65 R66}\\

Items in normal type must be entered exactly as shown, whereas those in
italics must be provided by the user.  The colons are important!
For this particular example, one would set {\tt ORDER=1} in the {\tt MATR}
definition.  Typically, the {\em Ci} are zero, except for {\em C5}, which
is usually equal to the length of the element (which must be specified with
the {\tt L} parameter in the {\tt MATR} definition).

As of release 2019.2, the required format changed slightly. 
In the new version, the start of the matrix is determined by reading through the file until 
a line starting with \verb|C:| is found.
In the past, 
instead of starting with \verb|C:|, the first line of the matrix could start
with any string terminated by a colon, but that line had to be the first line in the
file, which conflicted with the format emitted by \verb|analyze_map|.

The \verb|FRACTION| parameter can be used to interpolate the matrix elements between the
matrix $M_0$ read from \verb|FILENAME| and the identity matrix $I$, according to
\begin{equation}
M = fM_0 + (1-f)I.
\end{equation}
This can be used, for example, to gradually ramp in the effect as part of an optimization.
N.B.: in general, the matrix does not have unit determinant unless $f=0$ or $f=1$, so this 
feature should be used only as a knob to assist finding a solution with $f=1$.
Exceptions are when $M_0$ is a drift space or thin-lens quadrupole matrix, in which cases
the determinant of $M$ is always 1.




\begin{latexonly}
\newpage
\begin{center}{\Large\verb|MATTER|}\end{center}
\end{latexonly}\subsection{MATTER}
A Coulomb-scattering and energy-absorbing element simulating material in the
beam path.
\\
\begin{tabular}{|l|l|l|l|p{\descwidth}|} \hline
Parameter Name & Units & Type & Default & Description \\ \hline 
L & $M$ & double &  0.0 & length  \\ \hline 
XO & $M$ & double &  0.0 & radiation length  \\ \hline 
ELASTIC &  & long &  \verb|0| & elastic scattering?  \\ \hline 
ENERGY\_STRAGGLE &  & long &  \verb|0| & use simple-minded energy straggling model?  \\ \hline 
Z &  & long &  \verb|0| & Atomic number  \\ \hline 
A & $AMU$ & double &  0.0 & Atomic mass  \\ \hline 
RHO & $KG/M^3$ & double &  0.0 & Density  \\ \hline 
PLIMIT &  & double &   0.05 & Probability cutoff for each slice  \\ \hline 
\end{tabular}

\vspace*{0.5in}
This element is based on section 3.3.1 of the {\em Handbook of
Accelerator Physics and Engineering}, specifically, the
subsections {\bf Single Coulomb scattering of spin-${\rm \frac{1}{2}}$
particles}, {\bf Multiple Coulomb scattering through small angles},
and {\bf Radiation length}.
There are two aspects to this element: scattering and energy loss.

{\bf Scattering.}  The multiple Coulomb scattering formula is used
whenever the thickness of the material is greater than $0.001 X_o$,
where $X_o$ is the radiation length.  (Note that this is inaccurate
for materials thicker than $100 X_o$.)  For this regime, the user need
only specify the material thickness (L) and the radiation length (XO).

For materials thinner than $0.001 X_o$, the user must specify
additional parameters, namely, the atomic number (Z), atomic mass (A),
and mass density (RHO) of the material.  Note that the density is
given in units of $kg/m^3$.  (Multiply by $10^3$ to convert $g/cm^3$
to $kg/m^3$.)  In addition, the simulation parameter PLIMIT may be
modified.  

To understand this parameter, one must understand how {\tt elegant}
simulates the thin materials.  First, it computes the expected number
of scattering events per particle, $ E = \sigma_T n L = \frac{K_1
\pi^3 n L}{K_2^2 + K_2*\pi^2} $, where $n$ is the number density of
the material, L is the thickness of the material, $K_1 = (\frac{2 Z
r_e}{\beta^2 \gamma})^2$, and $K_2 = \frac{\alpha^2
Z^\frac{2}{3}}{(\beta\gamma)^2}$, with $r_e$ the classical electron radius
and $\alpha$ the fine structure constant.  The material is then broken
into $N$ slices, where $N = E/P_{limit}$.  For each slice, each
simulation particle has a probability $E/N$ of scattering.  If scattering
occurs, the location within the slice is computed using a uniform
distribution over the slice thickness.

For each scatter that occurs, the scattering angle, $\theta$ is
computed using the cumulative probability distribution
$F(\theta>\theta_o) = \frac{K_2 (\pi^2 - \theta_o^2)}{\pi^2 (K_2 +
\theta_o^2)}$.  This can be solved for $\theta_o$, giving
$\theta_o = \sqrt{\frac{(1-F)K_2\pi^2}{K_2 + F \pi^2}}$.  For each scatter,
$F$ is chosen from a uniform random distribution on $[0,1]$.

{\bf Energy loss.} There are two ways to compute energy loss in materials, using a simple minded approach and using the bremsstrahlung cross section.
The latter is recommended, but the former is kept for backward compatibility. 
\begin{itemize}
\item To enable bremsstrahlung simulation, simply set \verb|NUCLEAR_BREMSSTRAHLUNG=1|. Note that the energy loss is not correlated with the scattering
  angle, which is not entirely physical but should be reasonable for large numbers of scattering events.
\item To use the simplified approach:
      \begin{itemize}
      \item  Set \verb|ENERGY_DECAY=1|. Energy loss simulation is very simple.
The energy loss per unit distance traveled, $x$, is 
$\frac{dE}{dx} = -E/X_o$.  Hence, in traveling through a
material of thickness $L$, the energy of each particle is
transformed from $E$ to $E e^{-L/X_o}$.  
       \item Optionally, set \verb|ENERGY_STRAGGLE=1|. {\bf Not recomemnded. Exists only for backward compatibility.}
This adds variation in the energy lost
by particles.  The model is {\em very}, {\em very} crude and {\bf not recommended}.  It assumes that the standard deviation of the energy
loss is equal to half the mean energy loss.  This is an overestimate,
we think, and is provided to give an upper bound on the effects of
energy straggling until a real model can be developed.  Note one
obvious problem with this: if you split a MATTER element of length L
into two pieces of length L/2, the total energy loss will not not
change, but the induced energy spread will be about 30\% lower, due to
addition in quadrature.
\end{itemize}
\end{itemize}

{\bf Slotted absorber.} If the \verb|WIDTH| and \verb|SPACING| parameters are set to non-zero values, then a 
slotted absorber is simulated. The number of slots is by default infinite, but can be limited by setting
\verb|N_SLOTS| to a positive value; in this case, the slot array is centered about the transverse coordinate
given by the \verb|CENTER| parameter.



\begin{latexonly}
\newpage
\begin{center}{\Large\verb|MAXAMP|}\end{center}
\end{latexonly}\subsection{MAXAMP}
A collimating element that sets the maximum transmitted particle amplitudes for
all following elements, until the next MAXAMP.
\\
\begin{tabular}{|l|l|l|l|p{\descwidth}|} \hline
Parameter Name & Units & Type & Default & Description \\ \hline 
X\_MAX & $M$ & double &  0.0 & x half-aperture  \\ \hline 
Y\_MAX & $M$ & double &  0.0 & y half-aperture  \\ \hline 
ELLIPTICAL &  & long &  \verb|0| & is aperture elliptical?  \\ \hline 
\end{tabular}

\begin{latexonly}
\newpage
\begin{center}{\Large\verb|MODRF|}\end{center}
\end{latexonly}\subsection{MODRF}
A first-order matrix RF cavity with exact phase dependence, plus optional amplitude
and phase modulation.
\\
\begin{tabular}{|l|l|l|l|p{\descwidth}|} \hline
Parameter Name & Units & Type & Default & Description \\ \hline 
L & $M$ & double &  0.0 & length  \\ \hline 
VOLT & $V$ & double &  0.0 & nominal voltage  \\ \hline 
PHASE & $DEG$ & double &  0.0 & nominal phase  \\ \hline 
FREQ & $Hz$ & double &   500000000 & nominal frequency  \\ \hline 
Q &  & double &  0.0 & cavity Q  \\ \hline 
PHASE\_REFERENCE &  & long &  \verb|0| & phase reference number (to link with other time-dependent elements)  \\ \hline 
AMMAG &  & double &  0.0 & magnitude of amplitude modulation  \\ \hline 
AMPHASE & $DEG$ & double &  0.0 & phase of amplitude modulation  \\ \hline 
AMFREQ & $Hz$ & double &  0.0 & frequency of amplitude modulation  \\ \hline 
PMMAG & $DEG$ & double &  0.0 & magnitude of phase modulation  \\ \hline 
PMPHASE & $DEG$ & double &  0.0 & phase of phase modulation  \\ \hline 
PMFREQ & $Hz$ & double &  0.0 & frequency of phase modulation  \\ \hline 
FIDUCIAL &  & STRING &   NULL            & mode for determining fiducial arrival time (light, tmean, first, pmaximum)  \\ \hline 
\end{tabular}

\begin{latexonly}
\newpage
\begin{center}{\Large\verb|MONI|}\end{center}
\end{latexonly}\subsection{MONI}
A two-plane position monitor, accepting two rpn equations for the readouts
as a function of the actual positions (x and y).
\\
\begin{tabular}{|l|l|l|l|p{\descwidth}|} \hline
Parameter Name & Units & Type & Default & Description \\ \hline 
L & $M$ & double &  0.0 & length  \\ \hline 
DX & $M$ & double &  0.0 & misalignment  \\ \hline 
DY & $M$ & double &  0.0 & misalignment  \\ \hline 
WEIGHT &  & double &   1 & weight in correction  \\ \hline 
TILT &  & double &  0.0 & rotation about longitudinal axis  \\ \hline 
XCALIBRATION &  & double &   1 & calibration factor for x readout  \\ \hline 
YCALIBRATION &  & double &   1 & calibration factor for y readout  \\ \hline 
ORDER &  & long &  \verb|0| & matrix order  \\ \hline 
XREADOUT &  & STRING &   NULL            & rpn expression for x readout (actual position supplied in variables x, y  \\ \hline 
YREADOUT &  & STRING &   NULL            & rpn expression for y readout (actual position supplied in variables x, y  \\ \hline 
\end{tabular}

\begin{latexonly}
\newpage
\begin{center}{\Large\verb|MULT|}\end{center}
\end{latexonly}\subsection{MULT}
A canonical kick multipole.
\\
\begin{tabular}{|l|l|l|l|p{\descwidth}|} \hline
Parameter Name & Units & Type & Default & Description \\ \hline 
L & $M$ & double &  0.0 & length  \\ \hline 
KNL & $M^{(1-ORDER)}$ & double &  0.0 & integrated geometric strength  \\ \hline 
TILT & $RAD$ & double &  0.0 & rotation about longitudinal axis  \\ \hline 
BORE & $M$ & double &  0.0 & bore radius  \\ \hline 
BNL & $T M$ & double &  0.0 & integrated field at pole tip, used if BORE nonzero  \\ \hline 
DX & $M$ & double &  0.0 & misalignment  \\ \hline 
DY & $M$ & double &  0.0 & misalignment  \\ \hline 
DZ & $M$ & double &  0.0 & misalignment  \\ \hline 
FACTOR &  & double &   1 & factor by which to multiply strength  \\ \hline 
ORDER &  & long &   1               & multipole order  \\ \hline 
N\_KICKS &  & long &   4               & number of kicks  \\ \hline 
SYNCH\_RAD &  & long &  \verb|0| & include classical synchrotron radiation?  \\ \hline 
\end{tabular}

\begin{latexonly}
\newpage
\begin{center}{\Large\verb|NIBEND|}\end{center}
\end{latexonly}\subsection{NIBEND}
A numerically-integrated dipole magnet with various extended-fringe-field models.
\\
\begin{tabular}{|l|l|l|l|p{\descwidth}|} \hline
Parameter Name & Units & Type & Default & Description \\ \hline 
L & $M$ & double &  0.0 & arc length  \\ \hline 
ANGLE & $RAD$ & double &  0.0 & bending angle  \\ \hline 
E1 & $RAD$ & double &  0.0 & entrance edge angle  \\ \hline 
E2 & $RAD$ & double &  0.0 & exit edge angle  \\ \hline 
TILT &  & double &  0.0 & rotation about incoming longitudinal axis  \\ \hline 
DX & $M$ & double &  0.0 & misalignment  \\ \hline 
DY & $M$ & double &  0.0 & misalignment  \\ \hline 
DZ & $M$ & double &  0.0 & misalignment  \\ \hline 
FINT &  & double &   0.5 & edge-field integral  \\ \hline 
HGAP & $M$ & double &  0.0 & half-gap between poles  \\ \hline 
FP1 & $M$ & double &   10 & fringe parameter (tanh model)  \\ \hline 
FP2 & $M$ & double &   1 & not used  \\ \hline 
FSE &  & double &  0.0 & fractional strength error  \\ \hline 
ETILT &  & double &  0.0 & error rotation about incoming longitudinal axis  \\ \hline 
ACCURACY &  & double &   0.0001 & integration accuracy  \\ \hline 
MODEL &  & STRING &   linear          & fringe model (hard-edge, linear, cubic-spline, tanh, quintic  \\ \hline 
METHOD &  & STRING &   runge-kutta     & integration method (runge-kutta, bulirsch-stoer, modified-midpoint, two-pass modified-midpoint, leap-frog, non-adaptive runge-kutta  \\ \hline 
SYNCH\_RAD &  & long &  \verb|0| & include classical synchrotron radiation?  \\ \hline 
\end{tabular}

\begin{latexonly}
\newpage
\begin{center}{\Large\verb|NISEPT|}\end{center}
\end{latexonly}\subsection{NISEPT}
A numerically-integrated dipole magnet with a Cartesian gradient.
\\
\begin{tabular}{|l|l|l|l|p{\descwidth}|} \hline
Parameter Name & Units & Type & Default & Description \\ \hline 
L & $M$ & double &  0.0 & arc length  \\ \hline 
ANGLE & $RAD$ & double &  0.0 & bend angle  \\ \hline 
E1 & $RAD$ & double &  0.0 & entrance edge angle  \\ \hline 
B1 & $1/M$ & double &  0.0 & normalized gradient (K1=B1*L/ANGLE)  \\ \hline 
Q1REF & $M$ & double &  0.0 & distance from septum at which bending radius is L/ANGLE  \\ \hline 
FLEN & $M$ & double &  0.0 & fringe field length  \\ \hline 
ACCURACY &  & double &   0.0001 & integration accuracy  \\ \hline 
METHOD &  & STRING &   runge-kutta     & integration method (runge-kutta, bulirsch-stoer, modified-midpoint, two-pass modified-midpoint, leap-frog, non-adaptive runge-kutta  \\ \hline 
MODEL &  & STRING &   linear          & fringe model (hard-edge, linear, cubic-spline, tanh, quintic  \\ \hline 
\end{tabular}

\begin{latexonly}
\newpage
\begin{center}{\Large\verb|OCTU|}\end{center}
\end{latexonly}\subsection{OCTU}
Not implemented--use the MULT element.
\\
\begin{tabular}{|l|l|l|l|p{\descwidth}|} \hline
Parameter Name & Units & Type & Default & Description \\ \hline 
\end{tabular}

\begin{latexonly}
\newpage
\begin{center}{\Large\verb|PEPPOT|}\end{center}
\end{latexonly}\subsection{PEPPOT}
A pepper-pot plate.
\\
\begin{tabular}{|l|l|l|l|p{\descwidth}|} \hline
Parameter Name & Units & Type & Default & Description \\ \hline 
L & $M$ & double &  0.0 & length  \\ \hline 
RADII & $M$ & double &  0.0 & hole radius  \\ \hline 
TRANSMISSION &  & double &  0.0 & transmission of material  \\ \hline 
TILT & $RAD$ & double &  0.0 & rotation about longitudinal axis  \\ \hline 
THETA\_RMS & $RAD$ & double &  0.0 & rms scattering from material  \\ \hline 
N\_HOLES &  & long &  \verb|0| & number of holes  \\ \hline 
\end{tabular}

\begin{latexonly}
\newpage
\begin{center}{\Large\verb|PFILTER|}\end{center}
\end{latexonly}\subsection{PFILTER}
An element for energy and momentum filtration.
\\
\begin{tabular}{|l|l|l|l|p{\descwidth}|} \hline
Parameter Name & Units & Type & Default & Description \\ \hline 
DELTALIMIT &  & double &   -1 & maximum fractional momentum deviation  \\ \hline 
LOWERFRACTION &  & double &  0.0 & fraction of lowest-momentum particles to remove  \\ \hline 
UPPERFRACTION &  & double &  0.0 & fraction of highest-momentum particles to remove  \\ \hline 
FIXPLIMITS &  & long &  \verb|0| & fix the limits in p from LOWERFRACTION and UPPERFRACTION applied to first beam  \\ \hline 
BEAMCENTERED &  & long &  \verb|0| & if nonzero, center for DELTALIMIT is average beam momentum  \\ \hline 
\end{tabular}

\begin{latexonly}
\newpage
\begin{center}{\Large\verb|QUAD|}\end{center}
\end{latexonly}\subsection{QUAD}
A quadrupole implemented as a matrix, up to 2nd order.
\\
\begin{tabular}{|l|l|l|l|p{\descwidth}|} \hline
Parameter Name & Units & Type & Default & Description \\ \hline 
L & $M$ & double &  0.0 & length  \\ \hline 
K1 & $1/M^{2}$ & double &  0.0 & geometric strength  \\ \hline 
TILT & $RAD$ & double &  0.0 & rotation about longitudinal axis  \\ \hline 
FFRINGE &  & double &  0.0 & fraction of length occupied by linear fringe region  \\ \hline 
DX & $M$ & double &  0.0 & misalignment  \\ \hline 
DY & $M$ & double &  0.0 & misalignment  \\ \hline 
DZ & $M$ & double &  0.0 & misalignment  \\ \hline 
FSE & $M$ & double &  0.0 & fractional strength error  \\ \hline 
ORDER &  & long &  \verb|0| & matrix order  \\ \hline 
\end{tabular}

\begin{latexonly}
\newpage
\begin{center}{\Large\verb|QUFRINGE|}\end{center}
\end{latexonly}\subsection{QUFRINGE}
An element consisting of a linearly increasing or decreasing quadrupole field.
\\
\begin{tabular}{|l|l|l|l|p{\descwidth}|} \hline
Parameter Name & Units & Type & Default & Description \\ \hline 
L & $M$ & double &  0.0 & length  \\ \hline 
K1 & $1/M^{2}$ & double &  0.0 & peak geometric strength  \\ \hline 
TILT & $RAD$ & double &  0.0 & rotation about longitudinal axis  \\ \hline 
DX & $M$ & double &  0.0 & misalignment  \\ \hline 
DY & $M$ & double &  0.0 & misalignment  \\ \hline 
DZ & $M$ & double &  0.0 & misalignment  \\ \hline 
FSE & $M$ & double &  0.0 & fractional strength error  \\ \hline 
DIRECTION &  & long &  \verb|0| & 1=entrance, -1=exit  \\ \hline 
ORDER &  & long &  \verb|0| & matrix order  \\ \hline 
\end{tabular}

\begin{latexonly}
\newpage
\begin{center}{\Large\verb|RAMPP|}\end{center}
\end{latexonly}\subsection{RAMPP}
A momentum-ramping element that changes the central momentum according to a mpl
format file of the momentum factor vs time in seconds.
\\
\begin{tabular}{|l|l|l|l|p{\descwidth}|} \hline
Parameter Name & Units & Type & Default & Description \\ \hline 
WAVEFORM &  & STRING &   NULL            & $<$filename$>$=$<$x$>$+$<$y$>$ form specification of input file giving momentum factor vs time  \\ \hline 
\end{tabular}

\begin{latexonly}
\newpage
\begin{center}{\Large\verb|RAMPRF|}\end{center}
\end{latexonly}\subsection{RAMPRF}
A voltage-ramped RF cavity, implemented like RFCA.  The voltage ramp pattern is
given by a mpl-format file of the voltage factor vs time in seconds.
\\
\begin{tabular}{|l|l|l|l|p{\descwidth}|} \hline
Parameter Name & Units & Type & Default & Description \\ \hline 
L & $M$ & double &  0.0 & length  \\ \hline 
VOLT & $V$ & double &  0.0 & nominal voltage  \\ \hline 
PHASE & $DEG$ & double &  0.0 & nominal phase  \\ \hline 
FREQ & $Hz$ & double &   500000000 & nominal frequency  \\ \hline 
PHASE\_REFERENCE &  & long &  \verb|0| & phase reference number (to link with other time-dependent elements)  \\ \hline 
VOLT\_WAVEFORM &  & STRING &   NULL            & $<$filename$>$=$<$x$>$+$<$y$>$ form specification of input file giving voltage waveform factor vs time  \\ \hline 
PHASE\_WAVEFORM &  & STRING &   NULL            & $<$filename$>$=$<$x$>$+$<$y$>$ form specification of input file giving phase offset vs time (requires FREQ\_WAVEFORM)  \\ \hline 
FREQ\_WAVEFORM &  & STRING &   NULL            & $<$filename$>$=$<$x$>$+$<$y$>$ form specification of input file giving frequency factor vs time (requires PHASE\_WAVEFORM)  \\ \hline 
FIDUCIAL &  & STRING &   NULL            & mode for determining fiducial arrival time (light, tmean, first, pmaximum)  \\ \hline 
\end{tabular}

\begin{latexonly}
\newpage
\begin{center}{\Large\verb|RBEN|}\end{center}
\end{latexonly}\subsection{RBEN}
A rectangular dipole, implemented as a SBEND with edge angles.
\\
\begin{tabular}{|l|l|l|l|p{\descwidth}|} \hline
Parameter Name & Units & Type & Default & Description \\ \hline 
L & $M$ & double &  0.0 & arc length  \\ \hline 
ANGLE & $RAD$ & double &  0.0 & bend angle  \\ \hline 
K1 & $1/M^{2}$ & double &  0.0 & geometric focusing strength  \\ \hline 
E1 & $RAD$ & double &  0.0 & entrance edge angle  \\ \hline 
E2 & $RAD$ & double &  0.0 & exit edge angle  \\ \hline 
TILT & $RAD$ & double &  0.0 & rotation about incoming longitudinal axis  \\ \hline 
K2 & $1/M^{3}$ & double &  0.0 & geometric sextupole strength  \\ \hline 
H1 & $1/M$ & double &  0.0 & entrace pole-face curvature  \\ \hline 
H2 & $1/M$ & double &  0.0 & exit pole-face curvature  \\ \hline 
HGAP & $M$ & double &  0.0 & half-gap between poles  \\ \hline 
FINT &  & double &   0.5 & edge-field integral  \\ \hline 
DX & $M$ & double &  0.0 & misaligment of entrance  \\ \hline 
DY & $M$ & double &  0.0 & misalignment of entrace  \\ \hline 
DZ & $M$ & double &  0.0 & misalignment of entrance  \\ \hline 
FSE &  & double &  0.0 & fractional strength error  \\ \hline 
ETILT & $RAD$ & double &  0.0 & error rotation about incoming longitudinal axis  \\ \hline 
EDGE1\_EFFECTS &  & long &   1               & include entrace edge effects?  \\ \hline 
EDGE2\_EFFECTS &  & long &   1               & include exit edge effects?  \\ \hline 
ORDER &  & long &  \verb|0| & matrix order  \\ \hline 
EDGE\_ORDER &  & long &  \verb|0| & edge matrix order  \\ \hline 
TRANSPORT &  & long &  \verb|0| & use (incorrect) TRANSPORT equations for T436 of edge?  \\ \hline 
USE\_BN &  & long &  \verb|0| & use B1 and B2 instead of K1 and K2 values?  \\ \hline 
B1 & $1/M$ & double &  0.0 & K1 = B1*rho, where rho is bend radius  \\ \hline 
B2 & $1/M^{2}$ & double &  0.0 & K2 = B2*rho  \\ \hline 
\end{tabular}

\begin{latexonly}
\newpage
\begin{center}{\Large\verb|RCOL|}\end{center}
\end{latexonly}\subsection{RCOL}
A rectangular collimator.
\\
\begin{tabular}{|l|l|l|l|p{\descwidth}|} \hline
Parameter Name & Units & Type & Default & Description \\ \hline 
L & $M$ & double &  0.0 & length  \\ \hline 
X\_MAX & $M$ & double &  0.0 & half-width in x  \\ \hline 
Y\_MAX & $M$ & double &  0.0 & half-width in y  \\ \hline 
DX & $M$ & double &  0.0 & misalignment  \\ \hline 
DY & $M$ & double &  0.0 & misalignment  \\ \hline 
\end{tabular}

\begin{latexonly}
\newpage
\begin{center}{\Large\verb|RECIRC|}\end{center}
\end{latexonly}\subsection{RECIRC}
An element that defines the point to which particles recirculate in multi-pass
tracking
\\
\begin{tabular}{|l|l|l|l|p{\descwidth}|} \hline
Parameter Name & Units & Type & Default & Description \\ \hline 
I\_RECIRC\_ELEMENT &  & long &  \verb|0| & \\ \hline 
\end{tabular}

\begin{latexonly}
\newpage
\begin{center}{\Large\verb|REFLECT|}\end{center}
\end{latexonly}\subsection{REFLECT}
Reflects the beam back on itself, which is useful for multiple beamline matching.
\\
\begin{tabular}{|l|l|l|l|p{\descwidth}|} \hline
Parameter Name & Units & Type & Default & Description \\ \hline 
DUMMY &  & long &  \verb|0| & \\ \hline 
\end{tabular}

\begin{latexonly}
\newpage
\begin{center}{\Large\verb|REMCOR|}\end{center}
\end{latexonly}\subsection{REMCOR}
An element to remove correlations from the tracked beam to simulate certain types of correction.
\\
\begin{tabular}{|l|l|l|l|p{\descwidth}|} \hline
Parameter Name & Units & Type & Default & Description \\ \hline 
X &  & long &   1               & remove correlations in x?  \\ \hline 
XP &  & long &   1               & remove correlations in x'?  \\ \hline 
Y &  & long &   1               & remove correlations in y?  \\ \hline 
YP &  & long &   1               & remove correlations in y'?  \\ \hline 
WITH &  & long &   6               & coordinate to remove correlations with (1,2,3,4,5,6)=(x,x',y,y',s,dP/Po)  \\ \hline 
ONCE\_ONLY &  & long &  \verb|0| & compute correction only for first beam, apply to all?  \\ \hline 
\end{tabular}

\begin{latexonly}
\newpage
\begin{center}{\Large\verb|RFCA|}\end{center}
\end{latexonly}\subsection{RFCA}
A first-order matrix RF cavity with exact phase dependence.
\\
\begin{tabular}{|l|l|l|l|p{\descwidth}|} \hline
Parameter Name & Units & Type & Default & Description \\ \hline 
L & $M$ & double &  0.0 & length  \\ \hline 
VOLT & $V$ & double &  0.0 & peak voltage  \\ \hline 
PHASE & $DEG$ & double &  0.0 & phase  \\ \hline 
FREQ & $Hz$ & double &   500000000 & frequency  \\ \hline 
Q &  & double &  0.0 & cavity Q  \\ \hline 
PHASE\_REFERENCE &  & long &  \verb|0| & phase reference number (to link with other time-dependent elements)  \\ \hline 
CHANGE\_P0 &  & long &  \verb|0| & does cavity change central momentum?  \\ \hline 
CHANGE\_T &  & long &  \verb|0| & not recommended  \\ \hline 
FIDUCIAL &  & STRING &   NULL            & mode for determining fiducial arrival time (light, tmean, first, pmaximum)  \\ \hline 
END1\_FOCUS &  & long &  \verb|0| & include focusing at entrance?  \\ \hline 
END2\_FOCUS &  & long &  \verb|0| & include focusing at exit?  \\ \hline 
BODY\_FOCUS\_MODEL &  & STRING &   NULL            & None (default) or SRS (simplified Rosenzweig/Serafini for standing wave)  \\ \hline 
N\_KICKS &  & long &   1               & number of kicks to use.  Set to zero for matrix method.  \\ \hline 
DX & $M$ & double &  0.0 & misalignment  \\ \hline 
DY & $M$ & double &  0.0 & misalignment  \\ \hline 
T\_REFERENCE & $S$ & double &   -1 & arrival time of reference particle  \\ \hline 
LINEARIZE &  & long &  \verb|0| & Linearize phase dependence?  \\ \hline 
\end{tabular}

\vspace*{0.5in}
The phase convention is as follows, assuming a positive rf voltage:
\verb|PHASE=90| is the crest for acceleration.  \verb|PHASE=180| is the stable
phase for a storage ring above transition without energy losses.

The body-focusing model is based on Rosenzweig and Serafini, Phys. Rev. E 49 (2),
1599.  As suggested by N. Towne (NSLS), I simplified this to assume a pure pi-mode
standing wave.

The \verb|CHANGE_T| parameter may be needed for reasons that stem from
{\tt elegant}'s internal use of the total time-of-flight as the
longitudinal coordinate.  If the accelerator is very long or a large
number of turns are being tracked, rounding error may affect the
simulation, introducing spurious phase jumps.  By setting
\verb|CHANGE_T=1|, you can force {\tt elegant} to modify the time
coordinates of the particles to subtract off $N T_{rf}$, where
$T_{tf}$ is the rf period and $N = \lfloor t/T_{tf}+0.5\rfloor$.  If
you are tracking a ring with rf at some harmonic $h$ of the revolution
frequency, this will result in the time coordinates being relative to
the ideal revolution period, $T_{rf}*h$.  If you have multiple rf
cavities in a ring, you need only use this feature on one of them.
Also, you can use \verb|CHANGE_T=1| if you simply prefer to have the
offset time coordinates in output files and analysis. 

N.B.: {\em Do not use \verb|CHANGE_T=1| if you have rf cavities that
are not at harmonics of one another or if you have other
time-dependent elements that are not resonant.}

\begin{latexonly}
\newpage
\begin{center}{\Large\verb|RFCW|}\end{center}
\end{latexonly}\subsection{RFCW}
A combination of RFCA, WAKE, and TRWAKE.
\\
\begin{tabular}{|l|l|l|l|p{\descwidth}|} \hline
Parameter Name & Units & Type & Default & Description \\ \hline 
L & $M$ & double &  0.0 & length  \\ \hline 
CELL\_LENGTH & $M$ & double &  0.0 & cell length (used to scale wakes)  \\ \hline 
VOLT & $V$ & double &  0.0 & voltage  \\ \hline 
PHASE & $DEG$ & double &  0.0 & phase  \\ \hline 
FREQ & $Hz$ & double &   500000000 & frequency  \\ \hline 
Q &  & double &  0.0 & cavity Q  \\ \hline 
PHASE\_REFERENCE &  & long &  \verb|0| & phase reference number (to link with other time-dependent elements)  \\ \hline 
CHANGE\_P0 &  & long &  \verb|0| & does element change central momentum?  \\ \hline 
CHANGE\_T &  & long &  \verb|0| & not recommended  \\ \hline 
FIDUCIAL &  & STRING &   NULL            & mode for determining fiducial arrival time (light, tmean, first, pmaximum)  \\ \hline 
END1\_FOCUS &  & long &  \verb|0| & include focusing at entrance?  \\ \hline 
END2\_FOCUS &  & long &  \verb|0| & include focusing at exit?  \\ \hline 
BODY\_FOCUS\_MODEL &  & STRING &   NULL            & None (default) or SRS (simplified Rosenzweig/Serafini for standing wave)  \\ \hline 
N\_KICKS &  & long &   1               & number of kicks to use.  Set to zero for matrix method.  \\ \hline 
WAKEFILE &  & STRING &   NULL            & name of file containing Green functions  \\ \hline 
ZWAKEFILE &  & STRING &   NULL            & if WAKEFILE=NULL, optional name of file containing longitudinal Green function  \\ \hline 
TRWAKEFILE &  & STRING &   NULL            & if WAKEFILE=NULL, optional name of file containing transverse Green functions  \\ \hline 
TCOLUMN &  & STRING &   NULL            & column containing time data  \\ \hline 
WXCOLUMN &  & STRING &   NULL            & column containing x Green function  \\ \hline 
WYCOLUMN &  & STRING &   NULL            & column containing y Green function  \\ \hline 
\end{tabular}

\begin{latexonly}
\newpage
\begin{center}{\Large\verb|RFCW| continued}\end{center}
\end{latexonly}
A combination of RFCA, WAKE, and TRWAKE.
\\
\begin{tabular}{|l|l|l|l|p{\descwidth}|} \hline
Parameter Name & Units & Type & Default & Description \\ \hline 
WZCOLUMN &  & STRING &   NULL            & column containing longitudinal Green function  \\ \hline 
N\_BINS &  & long &  \verb|0| & number of bins for current histogram  \\ \hline 
INTERPOLATE &  & long &  \verb|0| & interpolate wake?  \\ \hline 
SMOOTHING &  & long &  \verb|0| & smooth current histogram?  \\ \hline 
SG\_HALFWIDTH &  & long &   4               & Savitzky-Golay filter half-width for smoothing  \\ \hline 
SG\_ORDER &  & long &   1               & Savitzky-Golay filter order for smoothing  \\ \hline 
DX & $M$ & double &  0.0 & misalignment  \\ \hline 
DY & $M$ & double &  0.0 & misalignment  \\ \hline 
LINEARIZE &  & long &  \verb|0| & Linearize phase dependence?  \\ \hline 
\end{tabular}

\vspace*{0.5in}
This element is a combination of the {\tt RFCA}, {\tt WAKE}, and {\tt
TRWAKE} elements.  As such, it provides combined simulation of an rf
cavity with longitudinal and transverse wakes, as well as longitudinal
space charge.

For the wakes, the input files and their interpretation are identical
to {\tt WAKE} and {\tt TRWAKE}, except that the transverse and
longitudinal wakes are interpreted as the wakes for a single cell of
length given by the {\tt CELL\_LENGTH} parameter.

Users should read the entries for {\tt WAKE}, {\tt TRWAKE}, and {\tt RFCA}
for more details on this element.

This element simulates longitudinal space charge using the
method described in \cite{Huang2004}.  This is based on the 
longitudinal space charge impedance per unit length
\begin{equation}
  Z_{lsc}(k) = \frac{i Z_0}{\pi k r_b^2} \left[ 1 - \frac{kr_b}{\gamma}K_1 \left(\frac{kr_b}{\gamma}\right)\right]
\end{equation}

N.B.: {\em Do not use \verb|CHANGE_T=1| if you have rf cavities that
are not at harmonics of one another or if you have other
time-dependent elements that are not resonant. Also, if you have harmonic
cavities, only use CHANGE_T on the cavity with the lowest frequency.
Failure to follow these rules can result in wrong results and crashes.}

\begin{latexonly}
\newpage
\begin{center}{\Large\verb|RFDF|}\end{center}
\end{latexonly}\subsection{RFDF}
A simple traveling-wave (beta=1) deflecting RF cavity.
\\
\begin{tabular}{|l|l|l|l|p{\descwidth}|} \hline
Parameter Name & Units & Type & Default & Description \\ \hline 
L & $M$ & double &  0.0 & length  \\ \hline 
PHASE & $DEG$ & double &  0.0 & phase  \\ \hline 
TILT & $RAD$ & double &  0.0 & rotation about longitudinal axis  \\ \hline 
FREQUENCY & $HZ$ & double &   2856000000 & frequency  \\ \hline 
VOLTAGE & $V$ & double &  0.0 & voltage  \\ \hline 
TIME\_OFFSET & $S$ & double &  0.0 & time offset (adds to phase)  \\ \hline 
N\_KICKS &  & long &   1               & number of kicks  \\ \hline 
PHASE\_REFERENCE &  & long &  \verb|0| & phase reference number (to link with other time-dependent elements)  \\ \hline 
\end{tabular}

\begin{latexonly}
\newpage
\begin{center}{\Large\verb|RFMODE|}\end{center}
\end{latexonly}\subsection{RFMODE}
A simulation of a beam-driven TM monopole mode of a RF cavity.
\\
\begin{tabular}{|l|l|l|l|p{\descwidth}|} \hline
Parameter Name & Units & Type & Default & Description \\ \hline 
RA & $Ohm$ & double &  0.0 & shunt impedance  \\ \hline 
RS & $Ohm$ & double &  0.0 & shunt impedance (Ra=2*Rs)  \\ \hline 
Q &  & double &  0.0 & cavity Q  \\ \hline 
FREQ & $Hz$ & double &  0.0 & frequency  \\ \hline 
CHARGE & $C$ & double &  0.0 & beam charge (or use CHARGE element)  \\ \hline 
INITIAL\_V & $V$ & double &  0.0 & initial voltage  \\ \hline 
INITIAL\_PHASE & $RAD$ & double &  0.0 & initial phase  \\ \hline 
INITIAL\_T & $S$ & double &  0.0 & time at which INITIAL\_V and INITIAL\_PHASE held  \\ \hline 
BETA &  & double &  0.0 & normalized load impedance  \\ \hline 
BIN\_SIZE & $S$ & double &  0.0 & bin size for current histogram (use 0 for autosize)  \\ \hline 
N\_BINS &  & long &   20              & number of bins for current histogram  \\ \hline 
PRELOAD &  & long &  \verb|0| & preload cavity with steady-state field  \\ \hline 
PRELOAD\_FACTOR &  & double &   1 & multiply preloaded field by this value  \\ \hline 
RIGID\_UNTIL\_PASS &  & long &  \verb|0| & don't affect the beam until this pass  \\ \hline 
DETUNED\_UNTIL\_PASS &  & long &  \verb|0| & cavity is completely detuned until this pass  \\ \hline 
SAMPLE\_INTERVAL &  & long &   1               & passes between output to RECORD file  \\ \hline 
RECORD &  & STRING &   NULL            & output file for cavity fields  \\ \hline 
SINGLE\_PASS &  & long &  \verb|0| & if nonzero, don't accumulate field from pass to pass  \\ \hline 
PASS\_INTERVAL &  & long &   1               & interval in passes at which to apply PASS\_INTERVAL times the field (may increase speed)  \\ \hline 
\end{tabular}

\begin{latexonly}
\newpage
\begin{center}{\Large\verb|RFTMEZ0|}\end{center}
\end{latexonly}\subsection{RFTMEZ0}
A TM-mode RF cavity specified by the on-axis Ez field.
\\
\begin{tabular}{|l|l|l|l|p{\descwidth}|} \hline
Parameter Name & Units & Type & Default & Description \\ \hline 
L & $M$ & double &  0.0 & length  \\ \hline 
FREQUENCY & $HZ$ & double &   2856000000 & frequency  \\ \hline 
PHASE & $RAD$ & double &  0.0 & phase  \\ \hline 
EZ\_PEAK & $V$ & double &  0.0 & Peak on-axis longitudinal electric field  \\ \hline 
TIME\_OFFSET & $S$ & double &  0.0 & time offset (adds to phase)  \\ \hline 
PHASE\_REFERENCE &  & long &  \verb|0| & phase reference number (to link to other time-dependent elements)  \\ \hline 
DX & $M$ & double &  0.0 & misalignment  \\ \hline 
DY & $M$ & double &  0.0 & misalignment  \\ \hline 
ETILT & $RAD$ & double &  0.0 & misalignment  \\ \hline 
EYAW & $RAD$ & double &  0.0 & misalignment  \\ \hline 
EPITCH & $RAD$ & double &  0.0 & misalignment  \\ \hline 
N\_STEPS &  & long &   100             & number of steps (for nonadaptive integration)  \\ \hline 
RADIAL\_ORDER &  & long &   1               & highest order in off-axis expansion  \\ \hline 
CHANGE\_P0 &  & long &  \verb|0| & does element change central momentum?  \\ \hline 
INPUTFILE &  & STRING &   NULL            & file containing Ez vs z at r=0  \\ \hline 
ZCOLUMN &  & STRING &   NULL            & column containing z values  \\ \hline 
EZCOLUMN &  & STRING &   NULL            & column containing Ez values  \\ \hline 
SOLENOID\_FILE &  & STRING &   NULL            & file containing map of Bz and Br vs z and r.  Each page contains values for a single r.  \\ \hline 
SOLENOID\_ZCOLUMN &  & STRING &   NULL            & column containing z values for solenoid map.  \\ \hline 
SOLENOID\_RCOLUMN &  & STRING &   NULL            & column containing r values for solenoid map.  \\ \hline 
SOLENOID\_BZCOLUMN &  & STRING &   NULL            & column containing Bz values for solenoid map.  \\ \hline 
SOLENOID\_BRCOLUMN &  & STRING &   NULL            & column containing Br values for solenoid map.  \\ \hline 
\end{tabular}

\begin{latexonly}
\newpage
\begin{center}{\Large\verb|RFTMEZ0| continued}\end{center}
\end{latexonly}
A TM-mode RF cavity specified by the on-axis Ez field.
\\
\begin{tabular}{|l|l|l|l|p{\descwidth}|} \hline
Parameter Name & Units & Type & Default & Description \\ \hline 
SOLENOID\_FACTOR &  & double &   1 & factor by which to multiply solenoid fields.  \\ \hline 
ACCURACY &  & double &   0.0001 & integration accuracy  \\ \hline 
METHOD & $ $ & STRING &   runge-kutta     & integration method (runge-kutta, bulirsch-stoer, non-adaptive runge-kutta, modified midpoint)  \\ \hline 
FIDUCIAL &  & STRING &   t,median        & \{t$|$p\},\{median$|$min$|$max$|$ave$|$first$|$light\} (e.g., "t,median")  \\ \hline 
\end{tabular}

\begin{latexonly}
\newpage
\begin{center}{\Large\verb|RMDF|}\end{center}
\end{latexonly}\subsection{RMDF}
A linearly-ramped electric field deflector, using an approximate analytical solution FOR LOW ENERGY PARTICLES.
\\
\begin{tabular}{|l|l|l|l|p{\descwidth}|} \hline
Parameter Name & Units & Type & Default & Description \\ \hline 
L & $M$ & double &  0.0 & length  \\ \hline 
TILT & $RAD$ & double &  0.0 & rotation about longitudinal axis  \\ \hline 
RAMP\_TIME & $S$ & double &   1e-09 & length of ramp  \\ \hline 
VOLTAGE & $V$ & double &  0.0 & full voltage  \\ \hline 
GAP & $M$ & double &   0.01 & gap between plates  \\ \hline 
TIME\_OFFSET & $S$ & double &  0.0 & time offset of ramp start  \\ \hline 
N\_SECTIONS &  & long &   10              & number of sections  \\ \hline 
PHASE\_REFERENCE &  & long &  \verb|0| & phase reference number (to link with other time-dependent elements)  \\ \hline 
DX & $M$ & double &  0.0 & misalignment  \\ \hline 
DY & $M$ & double &  0.0 & misalignment  \\ \hline 
\end{tabular}

\begin{latexonly}
\newpage
\begin{center}{\Large\verb|ROTATE|}\end{center}
\end{latexonly}\subsection{ROTATE}
An element that rotates the beam coordinates about the longitudinal axis.
\\
\begin{tabular}{|l|l|l|l|p{\descwidth}|} \hline
Parameter Name & Units & Type & Default & Description \\ \hline 
TILT & $RAD$ & double &  0.0 & rotation about longitudinal axis  \\ \hline 
\end{tabular}

\begin{latexonly}
\newpage
\begin{center}{\Large\verb|SAMPLE|}\end{center}
\end{latexonly}\subsection{SAMPLE}
An element that reduces the number of particles in the beam by interval-based or
random sampling.
\\
\begin{tabular}{|l|l|l|l|p{\descwidth}|} \hline
Parameter Name & Units & Type & Default & Description \\ \hline 
FRACTION &  & double &   1 & fraction to keep  \\ \hline 
INTERVAL &  & long &   1               & interval between sampled particles  \\ \hline 
\end{tabular}

\begin{latexonly}
\newpage
\begin{center}{\Large\verb|SBEN|}\end{center}
\end{latexonly}\subsection{SBEN}
A sector dipole implemented as a matrix, up to 2nd order.
\\
\begin{tabular}{|l|l|l|l|p{\descwidth}|} \hline
Parameter Name & Units & Type & Default & Description \\ \hline 
L & $M$ & double &  0.0 & arc length  \\ \hline 
ANGLE & $RAD$ & double &  0.0 & bend angle  \\ \hline 
K1 & $1/M^{2}$ & double &  0.0 & geometric focusing strength  \\ \hline 
E1 & $RAD$ & double &  0.0 & entrance edge angle  \\ \hline 
E2 & $RAD$ & double &  0.0 & exit edge angle  \\ \hline 
TILT & $RAD$ & double &  0.0 & rotation about incoming longitudinal axis  \\ \hline 
K2 & $1/M^{3}$ & double &  0.0 & geometric sextupole strength  \\ \hline 
H1 & $1/M$ & double &  0.0 & entrace pole-face curvature  \\ \hline 
H2 & $1/M$ & double &  0.0 & exit pole-face curvature  \\ \hline 
HGAP & $M$ & double &  0.0 & half-gap between poles  \\ \hline 
FINT &  & double &   0.5 & edge-field integral  \\ \hline 
DX & $M$ & double &  0.0 & misaligment of entrance  \\ \hline 
DY & $M$ & double &  0.0 & misalignment of entrace  \\ \hline 
DZ & $M$ & double &  0.0 & misalignment of entrance  \\ \hline 
FSE &  & double &  0.0 & fractional strength error  \\ \hline 
ETILT & $RAD$ & double &  0.0 & error rotation about incoming longitudinal axis  \\ \hline 
EDGE1\_EFFECTS &  & long &   1               & include entrace edge effects?  \\ \hline 
EDGE2\_EFFECTS &  & long &   1               & include exit edge effects?  \\ \hline 
ORDER &  & long &  \verb|0| & matrix order  \\ \hline 
EDGE\_ORDER &  & long &  \verb|0| & edge matrix order  \\ \hline 
TRANSPORT &  & long &  \verb|0| & use (incorrect) TRANSPORT equations for T436 of edge?  \\ \hline 
USE\_BN &  & long &  \verb|0| & use B1 and B2 instead of K1 and K2 values?  \\ \hline 
B1 & $1/M$ & double &  0.0 & K1 = B1*rho, where rho is bend radius  \\ \hline 
B2 & $1/M^{2}$ & double &  0.0 & K2 = B2*rho  \\ \hline 
\end{tabular}

\begin{latexonly}
\newpage
\begin{center}{\Large\verb|SCATTER|}\end{center}
\end{latexonly}\subsection{SCATTER}
A scattering element to add gaussian random numbers to particle coordinates.
\\
\begin{tabular}{|l|l|l|l|p{\descwidth}|} \hline
Parameter Name & Units & Type & Default & Description \\ \hline 
X & $M$ & double &  0.0 & rms scattering level for x  \\ \hline 
XP & $M$ & double &  0.0 & rms scattering level for x'  \\ \hline 
Y & $M$ & double &  0.0 & rms scattering level for y  \\ \hline 
YP & $M$ & double &  0.0 & rms scattering level for y'  \\ \hline 
DP & $M$ & double &  0.0 & rms scattering level for (p-pCentral)/pCentral  \\ \hline 
\end{tabular}

\begin{latexonly}
\newpage
\begin{center}{\Large\verb|SCRAPER|}\end{center}
\end{latexonly}\subsection{SCRAPER}
A collimating element that sticks into the beam from one side only.  The
directions 0, 1, 2, and 3 are from +x, +y, -x, and -y, respectively.
\\
\begin{tabular}{|l|l|l|l|p{\descwidth}|} \hline
Parameter Name & Units & Type & Default & Description \\ \hline 
L & $M$ & double &  0.0 & length  \\ \hline 
POSITION & $M$ & double &  0.0 & position of edge  \\ \hline 
DX & $M$ & double &  0.0 & misalignment  \\ \hline 
DY & $M$ & double &  0.0 & misalignment  \\ \hline 
XO & $M$ & double &  0.0 & radiation length  \\ \hline 
INSERT\_FROM &  & STRING &   NULL            & direction from which inserted (+x, -x, +y, -y  \\ \hline 
ELASTIC &  & long &  \verb|0| & elastic scattering?  \\ \hline 
DIRECTION &  & long &   -1              & obsolete  \\ \hline 
\end{tabular}

\begin{latexonly}
\newpage
\begin{center}{\Large\verb|SCRIPT|}\end{center}
\end{latexonly}\subsection{SCRIPT}
An element that allows transforming the beam using an external script.
\\
\begin{tabular}{|l|l|l|l|p{\descwidth}|} \hline
Parameter Name & Units & Type & Default & Description \\ \hline 
L & $M$ & double &  0.0 & Length to be used for matrix-based operations such as twiss parameter computation.  \\ \hline 
COMMAND &  & STRING &   NULL            & SDDS-compliant command to apply to the beam.  Use the sequence \%i to represent the input filename and \%o to represent the output filename.  \\ \hline 
ROOTNAME &  & STRING &   NULL            & Rootname for use in naming input and output files.  \%s may be used to represent the run rootname.  \\ \hline 
INPUT\_EXTENSION &  & STRING &   in              & Extension for the script input file.  \\ \hline 
OUTPUT\_EXTENSION &  & STRING &   out             & Extension for the script output file.  \\ \hline 
KEEP\_FILES &  & long &  \verb|0| & If nonzero, then script input and output files are not deleted after use.  By default, they are deleted.  \\ \hline 
NP0 &  & double &  0.0 & User-defined numerical parameter for command substitution for sequence \%np0  \\ \hline 
NP1 &  & double &  0.0 & User-defined numerical parameter for command substitution for sequence \%np1  \\ \hline 
NP2 &  & double &  0.0 & User-defined numerical parameter for command substitution for sequence \%np2  \\ \hline 
NP3 &  & double &  0.0 & User-defined numerical parameter for command substitution for sequence \%np3  \\ \hline 
NP4 &  & double &  0.0 & User-defined numerical parameter for command substitution for sequence \%np4  \\ \hline 
NP5 &  & double &  0.0 & User-defined numerical parameter for command substitution for sequence \%np5  \\ \hline 
\end{tabular}

\begin{latexonly}
\newpage
\begin{center}{\Large\verb|SCRIPT| continued}\end{center}
\end{latexonly}
An element that allows transforming the beam using an external script.
\\
\begin{tabular}{|l|l|l|l|p{\descwidth}|} \hline
Parameter Name & Units & Type & Default & Description \\ \hline 
NP6 &  & double &  0.0 & User-defined numerical parameter for command substitution for sequence \%np6  \\ \hline 
NP7 &  & double &  0.0 & User-defined numerical parameter for command substitution for sequence \%np7  \\ \hline 
NP8 &  & double &  0.0 & User-defined numerical parameter for command substitution for sequence \%np8  \\ \hline 
NP9 &  & double &  0.0 & User-defined numerical parameter for command substitution for sequence \%np9  \\ \hline 
SP0 &  & STRING &   NULL            & User-defined string parameter for command substitution for sequence \%sp0  \\ \hline 
SP1 &  & STRING &   NULL            & User-defined string parameter for command substitution for sequence \%sp1  \\ \hline 
SP2 &  & STRING &   NULL            & User-defined string parameter for command substitution for sequence \%sp2  \\ \hline 
SP3 &  & STRING &   NULL            & User-defined string parameter for command substitution for sequence \%sp3  \\ \hline 
SP4 &  & STRING &   NULL            & User-defined string parameter for command substitution for sequence \%sp4  \\ \hline 
SP5 &  & STRING &   NULL            & User-defined string parameter for command substitution for sequence \%sp5  \\ \hline 
SP6 &  & STRING &   NULL            & User-defined string parameter for command substitution for sequence \%sp6  \\ \hline 
SP7 &  & STRING &   NULL            & User-defined string parameter for command substitution for sequence \%sp7  \\ \hline 
\end{tabular}

\begin{latexonly}
\newpage
\begin{center}{\Large\verb|SCRIPT| continued}\end{center}
\end{latexonly}
An element that allows transforming the beam using an external script.
\\
\begin{tabular}{|l|l|l|l|p{\descwidth}|} \hline
Parameter Name & Units & Type & Default & Description \\ \hline 
SP8 &  & STRING &   NULL            & User-defined string parameter for command substitution for sequence \%sp8  \\ \hline 
SP9 &  & STRING &   NULL            & User-defined string parameter for command substitution for sequence \%sp9  \\ \hline 
\end{tabular}

\begin{latexonly}
\newpage
\begin{center}{\Large\verb|SEXT|}\end{center}
\end{latexonly}\subsection{SEXT}
A sextupole implemented as a matrix, up to 2nd order
\\
\begin{tabular}{|l|l|l|l|p{\descwidth}|} \hline
Parameter Name & Units & Type & Default & Description \\ \hline 
L & $M$ & double &  0.0 & length  \\ \hline 
K2 & $1/M^{3}$ & double &  0.0 & geometric strength  \\ \hline 
TILT & $RAD$ & double &  0.0 & rotation about longitudinal axis  \\ \hline 
DX & $M$ & double &  0.0 & misalignment  \\ \hline 
DY & $M$ & double &  0.0 & misalignment  \\ \hline 
DZ & $M$ & double &  0.0 & misalignment  \\ \hline 
FSE & $M$ & double &  0.0 & fractional strength error  \\ \hline 
ORDER &  & long &  \verb|0| & matrix order  \\ \hline 
\end{tabular}

\begin{latexonly}
\newpage
\begin{center}{\Large\verb|SOLE|}\end{center}
\end{latexonly}\subsection{SOLE}
A solenoid implemented as a matrix, up to 2nd order.
\\
\begin{tabular}{|l|l|l|l|p{\descwidth}|} \hline
Parameter Name & Units & Type & Default & Description \\ \hline 
L & $M$ & double &  0.0 & length  \\ \hline 
KS & $RAD/M$ & double &  0.0 & geometric strength  \\ \hline 
B & $T$ & double &  0.0 & field strength (used if KS is zero)  \\ \hline 
DX & $M$ & double &  0.0 & misalignment  \\ \hline 
DY & $M$ & double &  0.0 & misalignment  \\ \hline 
DZ & $M$ & double &  0.0 & misalignment  \\ \hline 
ORDER &  & long &  \verb|0| & matrix order  \\ \hline 
\end{tabular}

\begin{latexonly}
\newpage
\begin{center}{\Large\verb|SREFFECTS|}\end{center}
\end{latexonly}\subsection{SREFFECTS}
Simulation of synchrotron radiation effects (damping and quantum excitation).
\\
\begin{tabular}{|l|l|l|l|p{\descwidth}|} \hline
Parameter Name & Units & Type & Default & Description \\ \hline 
JX &  & double &   1 & x damping partition number  \\ \hline 
JY &  & double &   1 & y damping partition number  \\ \hline 
JDELTA &  & double &   2 & momentum damping partition number  \\ \hline 
EXREF & $m$ & double &  0.0 & reference equilibrium x emittance  \\ \hline 
EYREF & $m$ & double &  0.0 & reference equilibrium y emittance  \\ \hline 
SDELTAREF & $m$ & double &  0.0 & reference equilibrium fractional momentum spread  \\ \hline 
DDELTAREF &  & double &  0.0 & reference fractional momentum loss (per turn)  \\ \hline 
PREF & $m_{e}c$ & double &  0.0 & reference momentum (to which other reference values pertain)  \\ \hline 
COUPLING &  & double &  0.0 & x-y coupling  \\ \hline 
FRACTION &  & double &   1 & fraction of implied SR effect to simulate with each instance  \\ \hline 
DAMPING &  & long &   1               & include damping?  \\ \hline 
QEXCITATION &  & long &   1               & include quantum excitation?  \\ \hline 
LOSSES &  & long &   1               & include average losses?  \\ \hline 
\end{tabular}

\vspace*{0.5in}
This element allows simulation of synchrotron radiation effects in a
lumped fashion for quick, approximate results.  There are two ways to
set up the element: explicit initialization or automatic
initialization.  

In explicit initialization, the user supplies the quantities {\tt
EXREF}, {\tt EYREF}, {\tt SDELTAREF}, {\tt DDELTAREF}, and {\tt
PREF}.  These are, respectively, the reference values for the x-plane
emittance, y-plane emittance, fractional momentum spread, energy loss
per turn, and momentum.  The first four values pertain to the
reference momentum.  {\tt JX}, {\tt JY}, and {\tt JDELTA} may also
be given, although the defaults work for typical lattices.

In automatic initialization, the user turns on the radiation integral
feature in {\tt twiss\_output}, causing {\tt elegant} to automatically
compute the above quantities.  This will occur only if {\tt PREF=0}.
The {\tt COUPLING} parameter can be used to change the partitioning of
quantum excitation between the horizontal and vertical planes.

N.B.: Computation of Twiss parameters does not fully include the
effects of synchrotron radiation losses when these are imposed using
{\tt SREFFECTS} elements.  If {\tt PREF=0} (automatic initialization),
these effects are completely missing.  If {\tt PREF} is non-zero, then
{\tt elegant} will use the {\tt DDELTAREF} parameter to compute the
energy offset from the element, and thus its effect on the beam
trajectory.




\begin{latexonly}
\newpage
\begin{center}{\Large\verb|STRAY|}\end{center}
\end{latexonly}\subsection{STRAY}
A stray field element with local and global components.  Global components are
defined relative to the initial beamline direction.  ** Not correct if there are tilts
in the beamline. **
\\
\begin{tabular}{|l|l|l|l|p{\descwidth}|} \hline
Parameter Name & Units & Type & Default & Description \\ \hline 
L & $M$ & double &  0.0 & length  \\ \hline 
LBX & $T$ & double &  0.0 & local Bx  \\ \hline 
LBY & $T$ & double &  0.0 & local By  \\ \hline 
GBX & $T$ & double &  0.0 & global Bx  \\ \hline 
GBY & $T$ & double &  0.0 & global By  \\ \hline 
GBZ & $T$ & double &  0.0 & global Bz  \\ \hline 
ORDER &  & long &  \verb|0| & matrix order  \\ \hline 
\end{tabular}

\begin{latexonly}
\newpage
\begin{center}{\Large\verb|TMCF|}\end{center}
\end{latexonly}\subsection{TMCF}
A numerically-integrated accelerating TM RF cavity with spatially-constant fields.
\\
\begin{tabular}{|l|l|l|l|p{\descwidth}|} \hline
Parameter Name & Units & Type & Default & Description \\ \hline 
L & $M$ & double &  0.0 & length  \\ \hline 
FREQUENCY & $HZ$ & double &   2856000000 & frequency  \\ \hline 
PHASE & $S$ & double &  0.0 & phase  \\ \hline 
TIME\_OFFSET & $S$ & double &  0.0 & time offset (adds to phase)  \\ \hline 
RADIAL\_OFFSET & $M$ & double &   1 & not recommended  \\ \hline 
TILT & $RAD$ & double &  0.0 & rotation about longitudinal axis  \\ \hline 
ER & $V$ & double &  0.0 & radial electric field  \\ \hline 
BPHI & $T$ & double &  0.0 & azimuthal magnetic field  \\ \hline 
EZ & $V$ & double &  0.0 & longitudinal electric field  \\ \hline 
ACCURACY &  & double &   0.0001 & integration accuracy  \\ \hline 
X\_MAX & $M$ & double &  0.0 & x half-aperture  \\ \hline 
Y\_MAX & $M$ & double &  0.0 & y half-aperture  \\ \hline 
DX & $M$ & double &  0.0 & misalignment  \\ \hline 
DY & $M$ & double &  0.0 & misalignment  \\ \hline 
PHASE\_REFERENCE &  & long &  \verb|0| & phase reference number (to link with other time-dependent elements)  \\ \hline 
N\_STEPS &  & long &   100             & number of steps (for nonadaptive integration)  \\ \hline 
METHOD & $ $ & STRING &   runge-kutta     & integration method (runge-kutta, bulirsch-stoer, non-adaptive runge-kutta, modified midpoint)  \\ \hline 
FIDUCIAL &  & STRING &   t,median        & \{t$|$p\},\{median$|$min$|$max$|$ave$|$first$|$light\} (e.g., "t,median")  \\ \hline 
\end{tabular}

\begin{latexonly}
\newpage
\begin{center}{\Large\verb|TRCOUNT|}\end{center}
\end{latexonly}\subsection{TRCOUNT}
An element that defines the point from which transmission calculations are made.
\\
\begin{tabular}{|l|l|l|l|p{\descwidth}|} \hline
Parameter Name & Units & Type & Default & Description \\ \hline 
DUMMY &  & long &  \verb|0| & \\ \hline 
\end{tabular}

\begin{latexonly}
\newpage
\begin{center}{\Large\verb|TRFMODE|}\end{center}
\end{latexonly}\subsection{TRFMODE}
A simulation of a beam-driven TM dipole mode of a RF cavity.
\\
\begin{tabular}{|l|l|l|l|p{\descwidth}|} \hline
Parameter Name & Units & Type & Default & Description \\ \hline 
RA & $Ohm$ & double &  0.0 & shunt impedance  \\ \hline 
RS & $Ohm$ & double &  0.0 & shunt impedance (Ra=2*Rs)  \\ \hline 
Q &  & double &  0.0 & cavity Q  \\ \hline 
FREQ & $Hz$ & double &  0.0 & frequency  \\ \hline 
CHARGE & $C$ & double &  0.0 & beam charge (or use CHARGE element)  \\ \hline 
BETA &  & double &  0.0 & normalized load impedance  \\ \hline 
BIN\_SIZE & $S$ & double &  0.0 & bin size for current histogram (use 0 for autosize)  \\ \hline 
N\_BINS &  & long &   20              & number of bins for current histogram  \\ \hline 
PLANE &  & STRING &   both            & x, y, or both  \\ \hline 
SINGLE\_PASS &  & long &  \verb|0| & if nonzero, don't accumulate field from pass to pass  \\ \hline 
DX & $M$ & double &  0.0 & misalignment  \\ \hline 
DY & $M$ & double &  0.0 & misalignment  \\ \hline 
\end{tabular}

\begin{latexonly}
\newpage
\begin{center}{\Large\verb|TRWAKE|}\end{center}
\end{latexonly}\subsection{TRWAKE}
Transverse wake specified as a function of time lag behind the particle.
\\
\begin{tabular}{|l|l|l|l|p{\descwidth}|} \hline
Parameter Name & Units & Type & Default & Description \\ \hline 
INPUTFILE &  & STRING &   NULL            & name of file giving Green functions  \\ \hline 
TCOLUMN &  & STRING &   NULL            & column in INPUTFILE containing time data  \\ \hline 
WXCOLUMN &  & STRING &   NULL            & column in INPUTFILE containing x Green function  \\ \hline 
WYCOLUMN &  & STRING &   NULL            & column in INPUTFILE containing y Green function  \\ \hline 
CHARGE & $C$ & double &  0.0 & beam charge (or use CHARGE element)  \\ \hline 
FACTOR & $C$ & double &   1 & factor to multiply wake by  \\ \hline 
N\_BINS &  & long &   128             & number of bins for current histogram  \\ \hline 
INTERPOLATE &  & long &  \verb|0| & interpolate wake?  \\ \hline 
SMOOTHING &  & long &  \verb|0| & smooth current histogram?  \\ \hline 
SG\_HALFWIDTH &  & long &   4               & Savitzky-Golay filter half-width for smoothing  \\ \hline 
SG\_ORDER &  & long &   1               & Savitzky-Golay filter order for smoothing  \\ \hline 
DX & $M$ & double &  0.0 & misalignment  \\ \hline 
DY & $M$ & double &  0.0 & misalignmnet  \\ \hline 
\end{tabular}

\vspace*{0.5in}
The input file for this element gives the transverse-wake Green
functions, $W_x(t)$ and $W_y(t)$, versus time behind the particle. The
units of the wakes are V/C/m, so this element simulates the integrated
wake of some structure (e.g., a cell or series of cells).  If you
have, for example, the wake for a cell and you need the wake for N
cells, then you may use the {\tt FACTOR} parameter to make the
appropriate multiplication.  The values of the time coordinate should
begin at 0 and be equi-spaced.  A positive value of time represents
the distance behind the exciting particle.   Time values must be equally
spaced.

The sign convention for $W_q$ ($q$ being $x$ or $y$) is as follows: a
particle with $q>0$ will impart a positive kick ($\Delta q^\prime >
0$) to a trailing particle following $t$ seconds behind if $W_q(t)>0$.
A physical wake function should be zero at $t=0$ and also be initially
positive as $t$ increases from 0.

Use of the {\tt CHARGE} parameter on the {\tt TRWAKE} element is
disparaged.  It is preferred to use the {\tt CHARGE} element as part
of your beamline to define the charge.  

Setting the {\tt N\_BINS} paramater to 0 is recommended.  This results
in auto-scaling of the number of bins to accomodate the beam.  The bin
size is fixed by the spacing of the time points in the wake.

The default degree of smoothing ({\tt SG\_HALFWIDTH=4}) may be excessive.
It is suggested that users vary this parameter to verify that results
are reliable if smoothing is employed ({\tt SMOOTHING=1}).

The {\tt XFACTOR} and {\tt YFACTOR} parameters can be used to adjust
the strength of the wakes if the location at which you place the {\tt
TRWAKE} element has different beta functions than the location at
which the object that causes the wake actually resides.  

The {\tt XPOWER} and {\tt YPOWER} parameters can be used to change the
dependence of the wake on the x and y coordinates, respectively, of
the particles.  Normally, {\tt XPOWER=1} and {\tt YPOWER=1}.  This is
an ordinary dipole wake in a (supposedly) symmetric chamber.  If you
have an asymmetric chamber, then you will have a transverse wake kick
even if the beam is centered.  This part of the transverse wake is
described with {\tt XPOWER=0} and {\tt YPOWER=0}.  (Of course, you'll
need a 3-D wake code like GdfidL or MAFIA to compute this wake.)

If {\tt XPOWER=0} or {\tt YPOWER=0}, the units for the x or y wake
(respectively) must be $V/C$.  A negative value of the wake
corresponds to a kick toward negative x (or y).  

\begin{latexonly}
\newpage
\begin{center}{\Large\verb|TUBEND|}\end{center}
\end{latexonly}\subsection{TUBEND}
A special rectangular bend element for top-up backtracking.
\\
\begin{tabular}{|l|l|l|l|p{\descwidth}|} \hline
Parameter Name & Units & Type & Default & Description \\ \hline 
L & $M$ & double &  0.0 & arc length  \\ \hline 
ANGLE & $RAD$ & double &  0.0 & bend angle  \\ \hline 
FSE &  & double &  0.0 & fractional strength error  \\ \hline 
OFFSET &  & double &  0.0 & horizontal offset of magnet center from arc center  \\ \hline 
MAGNET\_WIDTH &  & double &  0.0 & horizontal width of the magnet pole  \\ \hline 
MAGNET\_ANGLE &  & double &  0.0 & angle that the magnet was designed for  \\ \hline 
\end{tabular}

\begin{latexonly}
\newpage
\begin{center}{\Large\verb|TWISS|}\end{center}
\end{latexonly}\subsection{TWISS}
Sets Twiss parameter values.
\\
\begin{tabular}{|l|l|l|l|p{\descwidth}|} \hline
Parameter Name & Units & Type & Default & Description \\ \hline 
BETAX & $M$ & double &   1 & horizontal beta function  \\ \hline 
BETAY & $M$ & double &   1 & vertical beta function  \\ \hline 
ALPHAX &  & double &  0.0 & horizontal alpha function  \\ \hline 
ALPHAY &  & double &  0.0 & vertical alpha function  \\ \hline 
FROM\_BEAM &  & long &  \verb|0| & compute correction from tracked beam properties instead of Twiss parameters?  \\ \hline 
ONCE\_ONLY &  & long &  \verb|0| & compute correction only for first beam or input twiss parameters, apply to all?  \\ \hline 
\end{tabular}

\begin{latexonly}
\newpage
\begin{center}{\Large\verb|TWLA|}\end{center}
\end{latexonly}\subsection{TWLA}
A numerically-integrated first-space-harmonic traveling-wave linear accelerator.
\\
\begin{tabular}{|l|l|l|l|p{\descwidth}|} \hline
Parameter Name & Units & Type & Default & Description \\ \hline 
L & $M$ & double &  0.0 & length  \\ \hline 
FREQUENCY & $HZ$ & double &   2856000000 & frequency  \\ \hline 
PHASE & $RAD$ & double &  0.0 & phase  \\ \hline 
TIME\_OFFSET & $S$ & double &  0.0 & time offset (adds to phase)  \\ \hline 
EZ & $V/M$ & double &  0.0 & electric field  \\ \hline 
B\_SOLENOID & $T$ & double &  0.0 & solenoid field  \\ \hline 
ACCURACY &  & double &   0.0001 & integration accuracy  \\ \hline 
X\_MAX & $M$ & double &  0.0 & x half-aperture  \\ \hline 
Y\_MAX & $M$ & double &  0.0 & y half-aperture  \\ \hline 
DX & $M$ & double &  0.0 & misalignment  \\ \hline 
DY & $M$ & double &  0.0 & misalignment  \\ \hline 
BETA\_WAVE &  & double &   1 & (phase velocity)/c  \\ \hline 
ALPHA & $1/M$ & double &  0.0 & field attenuation factor  \\ \hline 
PHASE\_REFERENCE &  & long &  \verb|0| & phase reference number (to link with other time-dependent elements)  \\ \hline 
N\_STEPS &  & long &   100             & number of steps (for nonadaptive integration)  \\ \hline 
FOCUSSING &  & long &   1               & include focusing effects?  \\ \hline 
METHOD & $ $ & STRING &   runge-kutta     & integration method (runge-kutta, bulirsch-stoer, non-adaptive runge-kutta, modified midpoint)  \\ \hline 
FIDUCIAL &  & STRING &   t,median        & \{t$|$p\},\{median$|$min$|$max$|$ave$|$first$|$light\} (e.g., "t,median")  \\ \hline 
CHANGE\_P0 &  & long &  \verb|0| & does element change central momentum?  \\ \hline 
\end{tabular}

\begin{latexonly}
\newpage
\begin{center}{\Large\verb|TWMTA|}\end{center}
\end{latexonly}\subsection{TWMTA}
A numerically-integrated traveling-wave muffin-tin accelerator.
\\
\begin{tabular}{|l|l|l|l|p{\descwidth}|} \hline
Parameter Name & Units & Type & Default & Description \\ \hline 
L & $M$ & double &  0.0 & length  \\ \hline 
FREQUENCY & $HZ$ & double &   2856000000 & frequency  \\ \hline 
PHASE & $RAD$ & double &  0.0 & phase  \\ \hline 
EZ & $V/M$ & double &  0.0 & electric field  \\ \hline 
ACCURACY &  & double &   0.0001 & integration accuracy  \\ \hline 
X\_MAX & $M$ & double &  0.0 & x half-aperture  \\ \hline 
Y\_MAX & $M$ & double &  0.0 & y half-aperture  \\ \hline 
DX & $M$ & double &  0.0 & misalignment  \\ \hline 
DY & $M$ & double &  0.0 & misalignment  \\ \hline 
KX & $1/M$ & double &  0.0 & horizontal wave number  \\ \hline 
BETA\_WAVE &  & double &   1 & (phase velocity)/c  \\ \hline 
BSOL &  & double &  0.0 & solenoid field  \\ \hline 
ALPHA & $1/M$ & double &  0.0 & field attenuation factor  \\ \hline 
PHASE\_REFERENCE &  & long &  \verb|0| & phase reference number (to link with other time-dependent elements)  \\ \hline 
N\_STEPS &  & long &   100             & number of kicks  \\ \hline 
METHOD & $ $ & STRING &   runge-kutta     & integration method (runge-kutta, bulirsch-stoer, non-adaptive runge-kutta, modified midpoint)  \\ \hline 
FIDUCIAL &  & STRING &   t,median        & \{t$|$p\},\{median$|$min$|$max$|$ave$|$first$|$light\} (e.g., "t,median")  \\ \hline 
\end{tabular}

\begin{latexonly}
\newpage
\begin{center}{\Large\verb|TWPL|}\end{center}
\end{latexonly}\subsection{TWPL}
A numerically-integrated traveling-wave stripline deflector.
\\
\begin{tabular}{|l|l|l|l|p{\descwidth}|} \hline
Parameter Name & Units & Type & Default & Description \\ \hline 
L & $M$ & double &  0.0 & length  \\ \hline 
RAMP\_TIME & $S$ & double &   1e-09 & time to ramp to full strenth  \\ \hline 
TIME\_OFFSET & $S$ & double &  0.0 & offset of ramp-start time  \\ \hline 
VOLTAGE & $V$ & double &  0.0 & maximum voltage between plates due to ramp  \\ \hline 
GAP & $M$ & double &   0.01 & gap between plates  \\ \hline 
STATIC\_VOLTAGE & $V$ & double &  0.0 & static component of voltage  \\ \hline 
TILT & $RAD$ & double &  0.0 & rotation about longitudinal axis  \\ \hline 
ACCURACY &  & double &   0.0001 & integration accuracy  \\ \hline 
X\_MAX & $M$ & double &  0.0 & x half-aperture  \\ \hline 
Y\_MAX & $M$ & double &  0.0 & y half-aperture  \\ \hline 
DX & $M$ & double &  0.0 & misalignment  \\ \hline 
DY & $M$ & double &  0.0 & misalignment  \\ \hline 
PHASE\_REFERENCE &  & long &  \verb|0| & phase reference number (to link with other time-dependent elements)  \\ \hline 
N\_STEPS &  & long &   100             & number of steps (for nonadaptive integration)  \\ \hline 
METHOD & $ $ & STRING &   runge-kutta     & integration method (runge-kutta, bulirsch-stoer, non-adaptive runge-kutta, modified midpoint)  \\ \hline 
FIDUCIAL &  & STRING &   t,median        & \{t$|$p\},\{median$|$min$|$max$|$ave$|$first$|$light\} (e.g., "t,median")  \\ \hline 
\end{tabular}

\begin{latexonly}
\newpage
\begin{center}{\Large\verb|VKICK|}\end{center}
\end{latexonly}\subsection{VKICK}
A vertical steering dipole implemented as a matrix, up to 2nd order.
\\
\begin{tabular}{|l|l|l|l|p{\descwidth}|} \hline
Parameter Name & Units & Type & Default & Description \\ \hline 
L & $M$ & double &  0.0 & length  \\ \hline 
KICK & $RAD$ & double &  0.0 & kick strength  \\ \hline 
TILT & $RAD$ & double &  0.0 & rotation about longitudinal axis  \\ \hline 
B2 & $1/M^{2}$ & double &  0.0 & normalized sextupole strength (kick = KICK*(1+B2*y\^2))  \\ \hline 
CALIBRATION &  & double &   1 & strength multiplier  \\ \hline 
EDGE\_EFFECTS &  & long &  \verb|0| & include edge effects?  \\ \hline 
ORDER &  & long &  \verb|0| & matrix order  \\ \hline 
STEERING &  & long &   1               & use for steering?  \\ \hline 
\end{tabular}

\begin{latexonly}
\newpage
\begin{center}{\Large\verb|VMON|}\end{center}
\end{latexonly}\subsection{VMON}
A vertical position monitor, accepting a rpn equation for the readout as a
function of the actual position (y).
\\
\begin{tabular}{|l|l|l|l|p{\descwidth}|} \hline
Parameter Name & Units & Type & Default & Description \\ \hline 
L & $M$ & double &  0.0 & length  \\ \hline 
DX & $M$ & double &  0.0 & misalignment  \\ \hline 
DY & $M$ & double &  0.0 & misalignment  \\ \hline 
WEIGHT &  & double &   1 & weight in correction  \\ \hline 
TILT &  & double &  0.0 & rotation about longitudinal axis  \\ \hline 
CALIBRATION &  & double &   1 & calibration factor for readout  \\ \hline 
ORDER &  & long &  \verb|0| & matrix order  \\ \hline 
READOUT &  & STRING &   NULL            & rpn expression for readout (actual position supplied in variable y)  \\ \hline 
\end{tabular}

\begin{latexonly}
\newpage
\begin{center}{\Large\verb|WAKE|}\end{center}
\end{latexonly}\subsection{WAKE}
Longitudinal wake specified as a function of time lag behind the particle.
\\
\begin{tabular}{|l|l|l|l|p{\descwidth}|} \hline
Parameter Name & Units & Type & Default & Description \\ \hline 
INPUTFILE &  & STRING &   NULL            & name of file giving Green function  \\ \hline 
TCOLUMN &  & STRING &   NULL            & column in INPUTFILE containing time data  \\ \hline 
WCOLUMN &  & STRING &   NULL            & column in INPUTFILE containing Green function  \\ \hline 
CHARGE & $C$ & double &  0.0 & beam charge (or use CHARGE element)  \\ \hline 
FACTOR & $C$ & double &   1 & factor to multiply wake by  \\ \hline 
N\_BINS &  & long &   128             & number of bins for current histogram  \\ \hline 
INTERPOLATE &  & long &  \verb|0| & interpolate wake?  \\ \hline 
SMOOTHING &  & long &  \verb|0| & smooth current histogram?  \\ \hline 
SG\_HALFWIDTH &  & long &   4               & Savitzky-Golay filter half-width for smoothing  \\ \hline 
SG\_ORDER &  & long &   1               & Savitzky-Golay filter order for smoothing  \\ \hline 
CHANGE\_P0 &  & long &  \verb|0| & change central momentum?  \\ \hline 
ALLOW\_LONG\_BEAM &  & long &  \verb|0| & allow beam longer than wake data?  \\ \hline 
\end{tabular}

\vspace*{0.5in}
The input file for this element gives the longitudinal Green function,
$W(t)$ versus time behind the particle. The units of the wake are V/C,
so this element simulates the integrated wake of some structure (e.g.,
a cell or series of cells).  If you have, for example, the wake for a
cell and you need the wake for N cells, then you may use the {\tt
FACTOR} parameter to make the appropriate multiplication.  The values
of the time coordinate should begin at 0 and be equi-spaced, and be expressed in seconds.
A positive value of time represents the distance behind the exciting
particle.  

A positive value of $W(t)$ results in energy {\em loss}.  A physical
wake function should be positive at $t=0$.
Causality requires that $W(t)=0$ for $t<0$. Acasual wakes are supported, 
provided the user sets \verb|ACAUSAL_ALLOWED=0|. The data file must contain
a value of $W(t)$ at $t=0$, and should have equal spans of time to the
negative and positive side of $t=0$.

Use of the {\tt CHARGE} parameter on the {\tt WAKE} element is
disparaged.  It is preferred to use the {\tt CHARGE} element as part
of your beamline to define the charge.  

Setting the {\tt N\_BINS} paramater to 0 is recommended.  This results
in auto-scaling of the number of bins to accomodate the beam.  The bin
size is fixed by the spacing of the time points in the wake.

The default degree of smoothing ({\tt SG\_HALFWIDTH=4}) may be excessive.
It is suggested that users vary this parameter to verify that results
are reliable if smoothing is employed ({\tt SMOOTHING=1}).

The algorithm for the wake element is as follows:
\begin{enumerate}
\item Compute the arrival time of each particle at the wake element. This
 is necessary because {\tt elegant} uses the longitudinal coordinate $s=\beta c t$.
\item Find the mean, minimum, and maximum arrival times ($t_{mean}$, $t_{min}$, and
 $t_{max}$, respectively).  If $t_{max}-t_{min}$ is greater than the duration of 
 the wakefield data, then {\tt elegant} either exits (default) or issues a warning (if 
 \verb|ALLOW_LONG_BEAM| is nonzero).  In the latter case, that part of the beam that
 is furthest from $t_{mean}$ is ignored for computation of the wake.
\item If the user has specified a fixed number of bins (not recommended), then {\tt elegant}
 centers those bins on $t_{mean}$.  Otherwise, the binning range encompasses $t_{min}-\Delta t$
 to $t_{max}+\Delta t$, where $\Delta t$ is the spacing of data in the wake file.
\item Create the arrival time histogram.  If any particles are outside the histogram range,
 issue a warning.
\item If \verb|SMOOTHING| is nonzero, smooth the arrival time histogram.
\item Convolve the arrival time histogram with the wake function.
\item Multiply the resultant wake by the charge and any user-defined factor.
\item Apply the energy changes for each particle.  This is done in such a way that
 the transverse momentum are conserved.
\item If \verb|CHANGE_P0| is nonzero, change the reference momentum of the beamline to 
 match the average momentum of the beam.
\end{enumerate}

Bunched-mode application of the short-range wake is possible using specially-prepared input
beams. 
See Section \ref{sect:bunchedBeams} for details.
The use of bunched mode for any particular \verb|WAKE| element is controlled using the \verb|BUNCHED_BEAM_MODE| parameter.

\begin{latexonly}
\newpage
\begin{center}{\Large\verb|WATCH|}\end{center}
\end{latexonly}\subsection{WATCH}
A beam property/motion monitor--allowed modes are centroid, parameter, coordinate, and fft.
\\
\begin{tabular}{|l|l|l|l|p{\descwidth}|} \hline
Parameter Name & Units & Type & Default & Description \\ \hline 
FRACTION &  & double &   1 & fraction of particles to dump (coordinate mode)  \\ \hline 
INTERVAL &  & long &   1               & interval for data output (in turns)  \\ \hline 
START\_PASS &  & long &  \verb|0| & pass on which to start  \\ \hline 
FILENAME &  & STRING &                   & output filename  \\ \hline 
LABEL &  & STRING &                   & output label  \\ \hline 
MODE &  & STRING &   coordinates     & coordinate, parameter, or centroid  \\ \hline 
X\_DATA &  & long &   1               & include x data in coordinate mode?  \\ \hline 
Y\_DATA &  & long &   1               & include y data in coordinate mode?  \\ \hline 
LONGIT\_DATA &  & long &   1               & include longitudinal data in coordinate mode?  \\ \hline 
EXCLUDE\_SLOPES &  & long &  \verb|0| & exclude slopes in coordinate mode?  \\ \hline 
FLUSH\_INTERVAL &  & long &  \verb|0| & file flushing interval (parameter or centroid mode)  \\ \hline 
\end{tabular}

\begin{latexonly}
\newpage
\begin{center}{\Large\verb|WIGGLER|}\end{center}
\end{latexonly}\subsection{WIGGLER}
A wiggler or undulator for damping or excitation of the beam.  Does not include focusing effects.
\\
\begin{tabular}{|l|l|l|l|p{\descwidth}|} \hline
Parameter Name & Units & Type & Default & Description \\ \hline 
L & $M$ & double &  0.0 & length  \\ \hline 
RADIUS & $M$ & double &  0.0 & peak bending radius  \\ \hline 
POLES &  & long &  \verb|0| & number of wiggler poles  \\ \hline 
\end{tabular}

\begin{latexonly}
\newpage
\begin{center}{\Large\verb|ZLONGIT|}\end{center}
\end{latexonly}\subsection{ZLONGIT}
A simulation of a single-pass broad-band or functionally specified longitudinal
impedance.
\\
\begin{tabular}{|l|l|l|l|p{\descwidth}|} \hline
Parameter Name & Units & Type & Default & Description \\ \hline 
CHARGE & $C$ & double &  0.0 & beam charge (or use CHARGE element)  \\ \hline 
BROAD\_BAND &  & long &  \verb|0| & broad-band impedance?  \\ \hline 
RA & $Ohm$ & double &  0.0 & shunt impedance  \\ \hline 
RS & $Ohm$ & double &  0.0 & shunt impedance (Ra=2*Rs)  \\ \hline 
Q &  & double &  0.0 & cavity Q  \\ \hline 
FREQ & $Hz$ & double &  0.0 & frequency (BROAD\_BAND=1)  \\ \hline 
ZREAL &  & STRING &   NULL            & $<$filename$>$=$<$x$>$+$<$y$>$ form specification of input file giving real part of impedance vs f (BROAD\_BAND=0)  \\ \hline 
ZIMAG &  & STRING &   NULL            & $<$filename$>$=$<$x$>$+$<$y$>$ form specification of input file giving imaginary part of impedance vs f (BROAD\_BAND=0)  \\ \hline 
BIN\_SIZE & $S$ & double &  0.0 & bin size for current histogram (use 0 for autosize)  \\ \hline 
N\_BINS &  & long &   128             & number of bins for current histogram  \\ \hline 
WAKES &  & STRING &   NULL            & filename for output of wake  \\ \hline 
WAKE\_INTERVAL &  & long &   1               & interval in passes at which to output wake  \\ \hline 
AREA\_WEIGHT &  & long &  \verb|0| & use area-weighting in assigning charge to histogram?  \\ \hline 
INTERPOLATE &  & long &  \verb|0| & interpolate wake?  \\ \hline 
SMOOTHING &  & long &  \verb|0| & smooth current histogram?  \\ \hline 
SG\_ORDER &  & long &   1               & Savitzky-Golay filter order for smoothing  \\ \hline 
SG\_HALFWIDTH &  & long &   4               & Savitzky-Golay filter halfwidth for smoothing  \\ \hline 
\end{tabular}

\begin{latexonly}
\newpage
\begin{center}{\Large\verb|ZTRANSVERSE|}\end{center}
\end{latexonly}\subsection{ZTRANSVERSE}
A simulation of a single-pass broad-band or functionally-specified transverse dipole impedance.
\\
\begin{tabular}{|l|l|l|l|p{\descwidth}|} \hline
Parameter Name & Units & Type & Default & Description \\ \hline 
CHARGE & $C$ & double &  0.0 & beam charge (or use CHARGE element)  \\ \hline 
BROAD\_BAND &  & long &  \verb|0| & broad-band impedance?  \\ \hline 
RS & $Ohm$ & double &  0.0 & shunt impedance (Ra=2*Rs)  \\ \hline 
Q &  & double &  0.0 & cavity Q  \\ \hline 
FREQ & $Hz$ & double &  0.0 & frequency (BROAD\_BAND=1)  \\ \hline 
INPUTFILE &  & STRING &   NULL            & name of file giving impedance (BROAD\_BAND=0)  \\ \hline 
FREQCOLUMN &  & STRING &   NULL            & column in INPUTFILE containing frequency  \\ \hline 
ZXREAL &  & STRING &   NULL            & column in INPUTFILE containing real impedance for x plane  \\ \hline 
ZXIMAG &  & STRING &   NULL            & column in INPUTFILE containing imaginary impedance for x plane  \\ \hline 
ZYREAL &  & STRING &   NULL            & column in INPUTFILE containing real impedance for y plane  \\ \hline 
ZYIMAG &  & STRING &   NULL            & column in INPUTFILE containing imaginary impedance for y plane  \\ \hline 
BIN\_SIZE & $S$ & double &  0.0 & bin size for current histogram (use 0 for autosize)  \\ \hline 
INTERPOLATE &  & long &  \verb|0| & interpolate wake?  \\ \hline 
N\_BINS &  & long &   128             & number of bins for current histogram  \\ \hline 
SMOOTHING &  & long &  \verb|0| & smooth current histogram?  \\ \hline 
SG\_ORDER &  & long &   1               & Savitzky-Golay filter order for smoothing  \\ \hline 
SG\_HALFWIDTH &  & long &   4               & Savitzky-Golay filter halfwidth for smoothing  \\ \hline 
DX & $M$ & double &  0.0 & misalignment  \\ \hline 
DY & $M$ & double &  0.0 & misalignment  \\ \hline 
\end{tabular}

