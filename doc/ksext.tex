This element simulates a sextupole using a kick method based on
symplectic integration.  The user specifies the number of kicks and
the order of the integration.  For computation of twiss parameters,
chromaticities, and response matrices, this element is treated like a
standard thick-lens sextuupole; i.e., the number of kicks and the
integration order become irrelevant.

\begin{raggedright}
Specification of systematic and random multipole errors is supported
through the \verb|SYSTEMATIC_MULTIPOLES|, \verb|EDGE_MULTIPOLES|, and 
\verb|RANDOM_MULTIPOLES|
fields.  These specify, respectively, fixed multipole strengths for the
body of the element, fixed multipole strengths for the edges of the element,
and random multipole strengths for the body of the element.
These fields give the names of SDDS files that supply the
multipole data.  The files are expected to contain a single page of
data with the following elements:
\end{raggedright}
\begin{enumerate}
\item Floating point parameter {\tt referenceRadius} giving the reference
 radius for the multipole data.
\item An integer column named {\tt order} giving the order of the multipole.
The order is defined as $(N_{poles}-2)/2$, so a quadrupole has order 1, a
sextupole has order 2, and so on.
\item Floating point columns {\tt normal} and {\tt skew} giving the values for the
normal and skew multipole strengths, respectively.  
(N.B.: previous versions used the names {\tt an} and {\tt bn}, respectively. This is still accepted but deprecated)
These are defined as a fraction 
of the main field strength measured at the reference radius, R: 
$f_n  = \frac{K_n R^n / n!}{K_m R^m / m!}$, where 
$m=2$ is the order of the main field and $n$ is the order of the error multipole.
A similar relationship holds for the skew multipole fractional strengths.
For random multipoles, the values are interpreted as rms values for the distribution.
\end{enumerate}

Specification of systematic higher multipoles due to steering fields is
supported through the \verb|STEERING_MULTIPOLES| field.  This field gives the
name of an SDDS file that supplies the multipole data.  The file is
expected to contain a single page of data with the following elements:
\begin{enumerate}
\item Floating point parameter {\tt referenceRadius} giving the reference
 radius for the multipole data.
\item An integer column named {\tt order} giving the order of the multipole.
The order is defined as $(N_{poles}-2)/2$.  The order must be an even number
because of the quadrupole symmetry.
\item Floating point column {\tt normal} giving the values for the normal
multipole strengths, which are driven by the horizontal steering field.
(N.B.: previous versions used the name {\tt an} for this data. This is still accepted but deprecated)
{\tt normal} is specifies the multipole strength as a fraction $f_n$ of the steering field strength measured at the reference radius, R: 
$f_n = \frac{K_n R^n / n!}{K_m R^m / m!}$, where 
$m=0$ is the order of the steering field and $n$ is the order of the error multipole.
The skew values (for vertical steering) are deduced from the {\tt normal} values, specifically,
$g_n = f_n*(-1)^{n/2}$.
\end{enumerate}

Apertures specified via an upstream \verb|MAXAMP| element or an \verb|aperture_input|
command will be imposed inside this element.
