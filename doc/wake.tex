The input file for this element gives the longitudinal Green function,
$W(t)$ versus time behind the particle. The units of the wake are V/C,
so this element simulates the integrated wake of some structure (e.g.,
a cell or series of cells).  If you have, for example, the wake for a
cell and you need the wake for N cells, then you may use the {\tt
FACTOR} parameter to make the appropriate multiplication.  The values
of the time coordinate should begin at 0 and be equi-spaced.  A
positive value of time represents the distance behind the exciting
particle.

Use of the {\tt CHARGE} parameter on the {\tt WAKE} element is
disparaged.  It is preferred to use the {\tt CHARGE} element as part
of your beamline to define the charge.  

Setting the {\tt N\_BINS} paramater to 0 is recommended.  This results
in auto-scaling of the number of bins to accomodate the beam.  The bin
size is fixed by the spacing of the time points in the wake.

The default degree of smoothing ({\tt SG\_HALFWIDTH=4}) may be excessive.
It is suggested that users vary this parameter to verify that results
are reliable if smoothing is employed ({\tt SMOOTHING=1}).

