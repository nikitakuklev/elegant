The input file for this element gives the longitudinal Green function,
$W(t)$ versus time behind the particle. The units of the wake are V/C,
so this element simulates the integrated wake of some structure (e.g.,
a cell or series of cells).  If you have, for example, the wake for a
cell and you need the wake for N cells, then you may use the {\tt
FACTOR} parameter to make the appropriate multiplication.  The values
of the time coordinate should begin at 0 and be equi-spaced, and be expressed in seconds.
A positive value of time represents the distance behind the exciting
particle.  

A positive value of $W(t)$ results in energy {\em loss}.  A physical
wake function should be positive at $t=0$.
Causality requires that $W(t)=0$ for $t<0$. Acasual wakes are supported, 
provided the user sets \verb|ACAUSAL_ALLOWED=0|. The data file must contain
a value of $W(t)$ at $t=0$, and should have equal spans of time to the
negative and positive side of $t=0$.

Use of the {\tt CHARGE} parameter on the {\tt WAKE} element is
disparaged.  It is preferred to use the {\tt CHARGE} element as part
of your beamline to define the charge.  

Setting the {\tt N\_BINS} paramater to 0 is recommended.  This results
in auto-scaling of the number of bins to accomodate the beam.  The bin
size is fixed by the spacing of the time points in the wake.

The default degree of smoothing ({\tt SG\_HALFWIDTH=4}) may be excessive.
It is suggested that users vary this parameter to verify that results
are reliable if smoothing is employed ({\tt SMOOTHING=1}).

The algorithm for the wake element is as follows:
\begin{enumerate}
\item Compute the arrival time of each particle at the wake element. This
 is necessary because {\tt elegant} uses the longitudinal coordinate $s=\beta c t$.
\item Find the mean, minimum, and maximum arrival times ($t_{mean}$, $t_{min}$, and
 $t_{max}$, respectively).  If $t_{max}-t_{min}$ is greater than the duration of 
 the wakefield data, then {\tt elegant} either exits (default) or issues a warning (if 
 \verb|ALLOW_LONG_BEAM| is nonzero).  In the latter case, that part of the beam that
 is furthest from $t_{mean}$ is ignored for computation of the wake.
\item If the user has specified a fixed number of bins (not recommended), then {\tt elegant}
 centers those bins on $t_{mean}$.  Otherwise, the binning range encompasses $t_{min}-\Delta t$
 to $t_{max}+\Delta t$, where $\Delta t$ is the spacing of data in the wake file.
\item Create the arrival time histogram.  If any particles are outside the histogram range,
 issue a warning.
\item If \verb|SMOOTHING| is nonzero, smooth the arrival time histogram.
\item Convolve the arrival time histogram with the wake function.
\item Multiply the resultant wake by the charge and any user-defined factor.
\item Apply the energy changes for each particle.  This is done in such a way that
 the transverse momentum are conserved.
\item If \verb|CHANGE_P0| is nonzero, change the reference momentum of the beamline to 
 match the average momentum of the beam.
\end{enumerate}

Bunched-mode application of the short-range wake is possible using specially-prepared input
beams. 
See Section \ref{sect:bunchedBeams} for details.
The use of bunched mode for any particular \verb|WAKE| element is controlled using the \verb|BUNCHED_BEAM_MODE| parameter.
