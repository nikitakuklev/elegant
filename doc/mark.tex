If {\tt FITPOINT=0}, this element results only in generation of additional output rows in the various 
files that contain output vs s.  For example, Twiss parameters, closed orbits, and matrices vs s will
all contain a row for each occurrence of  each marker element.

If {\tt FITPOINT=1}, the element has additional functionality in the context of optimizations.  In particular,
for occurrence {\em N} of the defined element {\em Element}, a series of symbols are created of the
form {\em Element}\#{\em N}.{\em quantity}, where {\em quantity} has the following values:
\begin{itemize}
\item The quantity {\tt pCentral} will be available, giving the reference value of $\beta\gamma$ at the marker location.
\item The quantities {\tt Cx}, {\tt Cxp}, {\tt Cy}, {\tt Cyp}, {\tt Cs}, and {\tt Cdelta} will be available,
 giving coordinate centroid values from tracking to the marker location.
\item The quantities {\tt Sx}, {\tt Sxp}, {\tt Sy}, {\tt Syp}, {\tt Ss}, and {\tt Sdelta} will be available,
 giving coordinate rms values $\sqrt{\langle (x_i - \langle x_i \rangle)^2\rangle}$ at the marker location from tracking.
\item The quantity {\tt Particles} will be available, giving the number of particles tracked to the marker location.
\item The quantities {\tt s{\em i}{\em j}} will be available, giving $\langle ( x_i -\langle x_i\rangle)( x_j - \langle x_j \rangle)\rangle$
 from tracking at the marker location, where $1\leq i\leq 6$ and $i<j\leq 6$.
\item The quantities {\tt betaxBeam}, {\tt alphaxBeam}, {\tt betayBeam}, and {\tt alphayBeam}, which are the twiss parameters computed
      from the beam moments obtained by tracking, will be available.
\item The quantities {\tt R{\em i}{\em j}} will be available, for $1\leq i \leq 6$ and $1\leq j \leq 6$,
  giving the accumulated first-order transport matrix to the marker location.
\item If the default matrix order (as set in {\tt run\_setup}) is 2 or greater, the quantities {\tt T{\em i}{\em j}{\em k}} 
  will be available, for $1\leq i \leq 6$, $1\leq j \leq 6$, and $1\leq k \leq j$,
  giving the accumulated second-order transport matrix to the marker location.
\item If Twiss parameter calculations are being performed (via {\tt twiss\_output} with {\tt output\_at\_each\_step=1}), then the quantities
  {\tt alphax}, {\tt betax}, {\tt nux}, {\tt psix}, {\tt etax}, {\tt etapx}, and {\tt etaxp}, along with similarly-named
  quantities for the vertical plane, will be available, giving twiss parameter values at the marker location.
  Note that {\tt etapx} and {\tt etaxp} are the same, being alternate names for $\eta_x^\prime$.  If radiation integrals are requested,
  the values of the radiation integrals are available in the quantities {\tt I1}, {\tt I2}, etc.
\item If coupled Twiss parameter calculations are being performed (via {\tt coupled\_twiss\_output} with {\tt output\_at\_each\_step=1}),
  then the quantities  \verb|betax1|, \verb|betax2|, \verb|betay1|, \verb|betay2|, \verb|cetax|, \verb|cetay|, and \verb|tilt| will be available.
  (These are the two beta functions for x and y, the coupled dispersion values for x and y, and the beam tilt). 
\item If moments calculations are being performed (via {\tt moments\_output} with  {\tt output\_at\_each\_step=1}), then the quantities 
 {\tt s{\em i}{\em j}m}, $1 \leq i\leq j\leq 6$, giving the 21 unique elements of the sigma matrix.  The quantities {\tt c{\em i}m}, $1\leq i \leq 6$,
 are also created, giving the 6 centroids from the moments computation. In addition, the emittances of the three modes are available using 
 {\tt e{\em i}m}, $1\leq i \leq 3$. The {\tt m} on the end of the symbols is to distinguish them from the moments computed from tracking.
\item If closed orbit calculations are being performed (via {\tt correct} or {\tt closed\_orbit}), then
  the quantities {\tt xco}, {\tt yco}, {\tt xpco}, and {\tt ypco} will be available, giving the
  x and y closed orbits and their slopes, respectively, at the marker location.
\item If floor coordinate calculations are begin performed (via {\tt floor\_coordinates}), then the quantities
  {\tt X}, {\tt Y}, {\tt Z}, {\tt theta}, {\tt phi}, {\tt psi}, and {\tt s} will be available.  These are,
  respectively, the three position coordinates, the three angle coordinates, and the total arc length
  at the marker location.
\end{itemize}

The misalignment controls for this element are non-functional, in the sense that they do not affect the beam.
However, when combined with external scripts and the \verb|GROUP| parameter, one can use this feature to
implement girder misalignments using pairs of markers to indicate the ends of the girders.  A future version
of {\tt elegant} will implement this internally.
