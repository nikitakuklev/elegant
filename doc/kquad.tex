This element simulates a quadrupole using a kick method based on
symplectic integration.  The user specifies the number of kicks and
the order of the integration.  For computation of twiss parameters and
response matrices, this element is treated like a standard thick-lens
quadrupole; i.e., the number of kicks and the integration order become
irrelevant.

\begin{raggedright}
Specification of systematic and random multipole errors is supported
through the \verb|SYSTEMATIC_MULTIPOLES| and 
\verb|RANDOM_MULTIPOLES|
fields.  These fields give the names of SDDS files that supply the
multipole data.  The files are expected to contain a single page of
data with the following elements:
\end{raggedright}
\begin{enumerate}
\item Floating point parameter {\tt referenceRadius} giving the reference
 radius for the multipole data.
\item An integer column named {\tt order} giving the order of the multipole.
The order is defined as $(N_{poles}-2)/2$, so a quadrupole has order 1, a
sextupole has order 2, and so on.
\item Floating point columns {\tt an} and {\tt bn} giving the values for the
normal and skew multipole strengths, respectively.  These are defined as a fraction 
of the main field strength measured at the reference radius, R: 
$a_n  = \frac{K_n r^n / n!}{K_m r^m / m!}$, where 
$m=1$ is the order of the main field and $n$ is the order of the error multipole.
A similar relationship holds for the skew multipoles.
For random multipoles, the values are interpreted as rms values for the distribution.
\end{enumerate}

Specification of systematic higher multipoles due to steering fields is
supported through the \verb|STEERING_MULTIPOLES| field.  This field gives the
name of an SDDS file that supplies the multipole data.  The file is
expected to contain a single page of data with the following elements:
\begin{enumerate}
\item Floating point parameter {\tt referenceRadius} giving the reference
 radius for the multipole data.
\item An integer column named {\tt order} giving the order of the multipole.
The order is defined as $(N_{poles}-2)/2$.  The order must be an even number
because of the quadrupole symmetry.
\item Floating point column {\tt an} giving the values for the normal
multipole strengths, which are driven by the horizontal steering field.
{\tt an} is specifies the multipole strength as a fraction of the steering field strength measured at the reference radius, R: 
$a_n  = \frac{K_n r^n / n!}{K_m r^m / m!}$, where 
$m=0$ is the order of the steering field and $n$ is the order of the error multipole.
The {\tt bn} values are deduced from the {\tt an} values, specifically,
$b_n = a_n*(-1)^{n/2}$.
\end{enumerate}

The dominant systematic multipole term in the steering field is a
sextupole.  Note that {\tt elegant} presently {\em does not} include
such sextupole contributions in the computation of the chromaticity
via the {\tt twiss\_output} command.  However, these chromatic effects
will be seen in tracking.

Apertures specified via an upstream \verb|MAXAMP| element or an \verb|aperture_input|
command will be imposed inside this element, with the following rules/limitations:
\begin{itemize}
\item If apertures from both sources are present, the smallest is used.
\item The apertures are assumed to be rectangular, even if the \verb|ELLIPTICAL| qualifier
 is set for \verb|MAXAMP|.
\end{itemize}

Fringe field effects  are based on publications of D.  Zhuo {\em et al.} \cite{Zhou-IPAC10} and  J. Irwin {\em et
  al.} \cite{Irwin-PAC95}, as well as unpublished work of C. X. Wang (ANL).  The fringe field is characterized by 
10 integrals given in equations 19, 20, and 21 of \cite{Zhou-IPAC10}.  However, the values input into {\tt elegant}
should be normalized by $K_1$ or $K_1^2$, as appropriate.

For the exit-side fringe field, let $s_1$ be the center of the magnet, $s_0$ be the location of the nominal end of the magnet
(for a hard-edge model), and let $s_2$ be a point well outside the magnet.  
Using $K_{1,he}(s)$ to represent the hard edge model and $K_1(s)$ the actual field profile, we 
define the normalized difference as $\tilde{k}(s) = (K_1(s) - K_{1,he}(s))/K_1(s_1)$.  (Thus, $\tilde{k}(s) = \tilde{K}(s)/K_0$, using
the notation of Zhou {\em et al.})

The integrals to be input to {\tt elegant} are defined as 
\begin{eqnarray}
i_0^- = \int_{s_1}^{s_0} \tilde{k}(s) ds & & i_0^+ = \int_{s_0}^{s_2} \tilde{k}(s) ds \\
i_1^- = \int_{s_1}^{s_0} \tilde{k}(s) (s-s_0) ds & & i_1^+ = \int_{s_0}^{s_2} \tilde{k}(s) (s-s_0) ds \\
i_2^- = \int_{s_1}^{s_0} \tilde{k}(s) (s-s_0)^2 ds & & i_2^+ = \int_{s_0}^{s_2} \tilde{k}(s) (s-s_0)^2 ds \\
i_3^- = \int_{s_1}^{s_0} \tilde{k}(s) (s-s_0)^3 ds & & i_3^+ = \int_{s_0}^{s_2} \tilde{k}(s) (s-s_0)^3 ds \\
\lambda_2^- = \int_{s_1}^{s_0} ds \int_s^{s_0} ds\prime \tilde{k}(s) \tilde{k}(s\prime) (s\prime-s) & & 
\lambda_2^+ = \int_{s_0}^{s_2} ds \int_s^{s_2} ds\prime \tilde{k}(s) \tilde{k}(s\prime) (s\prime-s) 
\end{eqnarray}

Normally, the effects are dominated by $i_1^-$ and $i_1^+$.  

The \verb|EDGE1_EFFECTS| and \verb|EDGE2_EFFECTS| parameters can be used to turn fringe field effects on and off, but also
to control the order of the implementation.  If the value is 1, linear fringe effects are included.  If the value is 2, 
leading-order (cubic) nonlinear effects are included.  If the value is 3 or higher, higher order effects are included.
