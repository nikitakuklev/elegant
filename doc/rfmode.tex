This element simulates a beam-driven monopole mode cavity using the fundamental theorem of beam loading and phasor rotation.
In addition, a generator-driven field may be included using a feedback system \cite{Berenc-IPAC15-MOPMA006}.

The feedback implementation uses amplitude and phase feedback.
It is active when a non-zero value is given for \verb|DRIVE_FREQUENCY| and when
both \verb|AMPLITUDE_FILTER| and \verb|PHASE_FILTER| are given. 
More information is available in \cite{Berenc-IPAC15-MOPMA006}.

Normally, the field dumped in the cavity by one particle affects trailing particles in the same turn.
However, if one is also using a \verb|WAKE| or \verb|ZLONGIT| element to simulate the short-range wake of the cavity, this would be double-counting.
In that case, one can use \verb|LONG_RANGE_ONLY=1| to suppress the same-turn effects of the \verb|RFMODE| element.

Two output files are available: the \verb|RECORD| file includes bunch-by-bunch data on the beam-induced fields and the total cavity fields.
The \verb|FEEDBACK_RECORD| file includes tick-by-tick data from the feedback system simulation; writing this file this can significantly impact performance.


NB: when \verb|BUNCHED_BEAM_MODE| is set to a value other than 1, in order to obtain the effect of several bunches while tracking
only one bunch, the total charge set with the \verb|TOTAL| parameter of the \verb|CHARGE| element should equal the charge in
a single bunch, not the entire beam. However, when \verb|BUNCHED_BEAM_MODE|=1 (allowing an indeterminant number of bunches to be
actually present), then \verb|TOTAL| should be the total for all bunches together.
