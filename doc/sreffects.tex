This element allows simulation of synchrotron radiation effects in a
lumped fashion for quick, approximate results.  There are two ways to
set up the element: explicit initialization or automatic
initialization.  

In explicit initialization, the user supplies the quantities {\tt
EXREF}, {\tt EYREF}, {\tt SDELTAREF}, {\tt DDELTAREF}, and {\tt
PREF}.  These are, respectively, the reference values for the x-plane
emittance, y-plane emittance, fractional momentum spread, energy loss
per turn, and momentum.  The first four values pertain to the
reference momentum.  {\tt JX}, {\tt JY}, and {\tt JDELTA} may also
be given, although the defaults work for typical lattices.

In automatic initialization, the user turns on the radiation integral
feature in {\tt twiss\_output}, causing {\tt elegant} to automatically
compute the above quantities.  This will occur only if {\tt PREF=0}.
The {\tt COUPLING} parameter can be used to change the partitioning of
quantum excitation between the horizontal and vertical planes.

In versions 19.0 and later, {\tt elegant} includes the effect of {\tt
SREFFECTS} on the closed orbit.  This presents a dilemna when
automatic initialization is used, because in order to perform
automatic initialization, {\tt elegant} has to compute the optics
functions.  However, it must determine the closed orbit to compute the
optics functions.  The solution to this is for the user to pre-compute
the twiss parameters and radiation integrals using \verb|twiss_output|
with \verb|output_at_each_step=0|.  The user is free to subsequently give
\verb|twiss_output| with \verb|output_at_each_step=1| to obtain the
results on the closed orbit.

N.B.: Computation of Twiss parameters does not fully include the
effects of synchrotron radiation losses when these are imposed using
{\tt SREFFECTS} elements.  If {\tt PREF=0} (automatic initialization),
these effects are completely missing.  If {\tt PREF} is non-zero, then
{\tt elegant} will use the {\tt DDELTAREF} parameter to compute the
energy offset from the element, and thus its effect on the beam
trajectory.



