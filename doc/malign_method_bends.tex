There are three modes for implementing alignment errors. Which is used
is controlled by the value of the \verb|MALIGN_METHOD| parameter:
\begin{itemize}
\item \verb|MALIGN_METHOD=0| --- This selects the original method, which was
  the only one available before version 2021.1. The misalignment is
  referenced to the entrance face. The \verb|EYAW| and \verb|EPITCH| parameters
  are ignored. 
\item \verb|MALIGN_METHOD=1| --- This selects a linearized method based on M. Venturini's
  work \cite{Venturini2021}, with misalignment referenced to the entrance face.
  The \verb|EYAW| and \verb|EPITCH| parameters are implemented.
\item \verb|MALIGN_METHOD=2| --- This selects a linearized method based on M. Venturini's
  work \cite{Venturini2021}, with misalignment referenced to the magnet center.
  The \verb|EYAW| and \verb|EPITCH| parameters are implemented.
\item \verb|MALIGN_METHOD=3| --- This selects an exact method based on M. Venturini's
  work \cite{Venturini2021}, with misalignment referenced to the entrance face.
  The \verb|EYAW| and \verb|EPITCH| parameters are implemented.
\item \verb|MALIGN_METHOD=4| --- This selects an exact method based on M. Venturini's
  work \cite{Venturini2021}, with misalignment referenced to the magnet center.
  The \verb|EYAW| and \verb|EPITCH| parameters are implemented.
\end{itemize}

For elements with non-zero \verb|TILT|, error displacements and rotations are performed in the lab frame.
An exception is the \verb|CCBEND|, where error displacements and rotations are performed in the tilted frame.
