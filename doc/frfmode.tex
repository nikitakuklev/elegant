This element is similar to {\tt RFMODE}, but it allows faster simulation of more than
one mode.  Also, the mode data is specified in an SDDS file.  This file can be
generated using the APS version of URMEL, or by hand. It must have the following
columns and units:
\begin{enumerate}
\item {\tt Frequency} --- The frequency of the mode in Hz.  Floating point.
\item {\tt Q} --- The quality factor.  Floating point.
\item {\tt ShuntImpedance} or {\tt ShuntImpedanceSymm} --- The shunt
  impedance in Ohms, defined as $V^2/(2*P)$.   Floating point. By default, {\tt ShuntImpedance} is
  used.  However, if the parameter \verb|USE_SYMM_DATA| is non-zero, then 
  {\tt ShuntImpedanceSymm}  is used.  The latter is the full-cavity 
  shunt impedance that URMEL computes
  by assuming that the input cavity used is one half of a symmetric cavity.
\end{enumerate}

The file may also have the following column:
\begin{enumerate}
\item {\tt beta} --- Normalized load impedance (dimensionless).   Floating point.  If not given, the
 $\beta=0$ is assumed for all modes.
\end{enumerate}

In many simulations, a transient effect may occur when using this
element because, in the context of the simulation, the impedance is
switched on instantaneously.  This can give a false indication of the
threshold for instability. The {\tt RAMP\_PASSES} parameter should
be used to prevent this by slowly ramping the impedance to full
strength.  This idea is from M. Blaskiewicz (BNL).
