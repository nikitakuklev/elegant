This element provides serial and parallel modeling of long range, multi-bunch, multi-pass, non-resonant wakes.
Resonant wakes can be modeled using the \verb|*RFMODE| elements, while short-range wakes are modeled with
\verb|WAKE|, \verb|TRWAKE|, \verb|ZLONGIT|, \verb|ZTRANSVERSE|, and \verb|RFCW|.

For the \verb|LRWAKE| element, the beam is assumed to be bunched and wakes are computed bunch-to-bunch.
The long-range wake is assumed to be constant within any single bunch.

To use this element, the beam has to be prepared in a special way so that {\tt elegant} can recognize which
particles belong to which bunches.
See Section \ref{sect:bunchedBeams} for details.
Given a properly prepared beam, the algorithm works as follows.
\begin{itemize}
\item Each processor uses arrays to record
\begin{itemize}
  \item How many particles are in each of $B$ bunches,
  \item The sum of the arrival times $t$ at the LRWAKE element for the particles in each bunch, and
  \item The sum of x and y at the LRWAKE element for the particles in each bunch.
  \end{itemize}
\item These arrays are summed across all the processors and used to compute the moments
  $\langle t \rangle$, $\langle x \rangle$, and $\langle y \rangle$ for each bunch, as
  well as the charge in each bunch.
\item Arrays of length $B$ from $N$ prior turns are kept in a buffer
  \begin{itemize}
  \item Buffer for turns $N-1$ to $1$ is copied to slots $N$ through $2$, thus overwriting the data for
    the oldest turn.
  \item The data for latest turn is copied into slot 1.
  \end{itemize}
\item  For each bunch, sums are performed over all prior bunches/turns to compute the voltage. For the 
  longitudinal wake, we have
  \begin{equation}
    V_z(b) = \sum\limits_{i=b}^{N*B} q_i W_z(\langle t_b \rangle - \langle t_i \rangle).
  \end{equation}
A positive value {\em decelerates} the particle.
For the horizontal dipole wake we have
  \begin{equation}
    V_x(b) = \sum\limits_{i=b}^{N*B} q_i \langle x_i \rangle W_x(\langle t_b \rangle - \langle t_i \rangle),
  \end{equation}
with the vertical wake being similar.
In both cases, a positive value deflects the particle toward positive $x$ or $y$ for a positive offset of the
driving particle.
\item The quadrupole wakes may also be included. In this case, the contribution to the horizontal wake is 
  \begin{equation}
    V_x(b) = \sum\limits_{i=b}^{N*B} q_i x_p W_x(\langle t_b \rangle - \langle t_i \rangle),
  \end{equation}
where $x_p$ is the coordinate of the probe particle. The vertical wake is similar.
\end{itemize}

To use LRWAKE, the user provides the wakes (functions of t) in an SDDS file.
These wakes may extend over an arbitrary number of turns, with the user declaring how many turns
to actually use as part of the element definition.
However, they should be zero within the region occupied by a single bunch, to avoid
double-counting with the true short-range wake.
(Note that the above sums include the self-wake.)
Similarly, the short-range should be zero for times comparable to the bunch spacing.

Note that the quadrupole wakes are related to the dipole wakes by constant numerical factors \cite{Chao-PRSTAB-111001}. Hence,
one may name the same column for \verb|QXCOLUMN| (\verb|QYCOLUMN|) and \verb|WXCOLUMN| (\verb|WYCOLUMN|)
and then specify \verb|QXFACTOR| (\verb|QYFACTOR|) appropriately.

