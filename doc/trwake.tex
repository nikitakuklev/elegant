The input file for this element gives the transverse-wake Green
functions, $W_x(t)$ and $W_y(t)$, versus time behind the particle. The
units of the wakes are V/C/m, so this element simulates the integrated
wake of some structure (e.g., a cell or series of cells).  If you
have, for example, the wake for a cell and you need the wake for N
cells, then you may use the {\tt FACTOR} parameter to make the
appropriate multiplication.  The values of the time coordinate should
begin at 0 and be equi-spaced, and be expressed in seconds.  A positive value of time represents
the distance behind the exciting particle.   Time values must be equally
spaced.

The sign convention for $W_q$ ($q$ being $x$ or $y$) is as follows: a
particle with $q>0$ will impart a positive kick ($\Delta q^\prime >
0$) to a trailing particle following $t$ seconds behind if $W_q(t)>0$.
A physical wake function should be zero at $t=0$ and also be initially
positive as $t$ increases from 0.

Use of the {\tt CHARGE} parameter on the {\tt TRWAKE} element is
disparaged.  It is preferred to use the {\tt CHARGE} element as part
of your beamline to define the charge.  

Setting the {\tt N\_BINS} paramater to 0 is recommended.  This results
in auto-scaling of the number of bins to accomodate the beam.  The bin
size is fixed by the spacing of the time points in the wake.

The default degree of smoothing ({\tt SG\_HALFWIDTH=4}) may be excessive.
It is suggested that users vary this parameter to verify that results
are reliable if smoothing is employed ({\tt SMOOTHING=1}).

The {\tt XFACTOR} and {\tt YFACTOR} parameters can be used to adjust
the strength of the wakes if the location at which you place the {\tt
TRWAKE} element has different beta functions than the location at
which the object that causes the wake actually resides.  

The {\tt X\_DRIVE\_EXPONENT} and {\tt Y\_DRIVE\_EXPONENT} parameters can be used to change the
dependence of the wake on the x and y coordinates, respectively, of the particles.  
Normally, these have the value 1, which corresponds to 
an ordinary dipole wake in a symmetric chamber.  

If you have an asymmetric chamber, then you will have a transverse
wake kick even if the beam is centered.  (Of course, you'll need a 3-D
wake code like GdfidL or MAFIA to compute this wake.)  This part of
the transverse wake is modeled by setting {\tt X\_DRIVE\_EXPONENT=0} and {\tt
Y\_DRIVE\_EXPONENT=0}.  It will result in an orbit distortion, but conceivably
could have other effects, such as emittance dilution.  In this case, 
the units for the x and y wake must be $V/C$.  A negative value of the wake
corresponds to a kick toward negative x (or y).

In addition, a quadrupole wake can be modeled by setting {\tt X\_DRIVE\_EXPONENT=0}, {\tt
Y\_DRIVE\_EXPONENT=0}, {\tt X\_PROBE\_EXPONENT=1}, and {\tt Y\_PROBE\_EXPONENT=1}.
The kick to a particle now depends on {\em it's} displacement, not on the displacement of
the leading particles.
In this case, the units for the wakes must be $V/C/m$.

Bunched-mode application of the short-range wake is possible using specially-prepared input
beams. 
See Section \ref{sect:bunchedBeams} for details.
The use of bunched mode for any particular \verb|TRWAKE| element is controlled using the \verb|BUNCHED_BEAM_MODE| parameter
