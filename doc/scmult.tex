
\documentclass[11pt]{article}
\usepackage{html}
\pagestyle{plain}
%\voffset=-0.75in
\newenvironment{req}{\begin{equation} \rm}{\end{equation}}
\setlength{\topmargin}{0.15 in}
\setlength{\oddsidemargin}{0 in}
\setlength{\evensidemargin}{0 in} % not applicable anyway
\setlength{\textwidth}{6.5 in}
\setlength{\headheight}{-0.5 in} % for 11pt font size
%\setlength{\footheight}{0 in}
\setlength{\textheight}{9 in}
\begin{document}


This element simulates transverse space charge (SC) kick using
K.Y. Ng's formula ( K.Y. Ng, FNAL, private communication). 

The linear SC force is given by:
\[
\Delta x'=\frac{K_{sc}Le^{-z^2/(2\sigma_z^2)}}{\sqrt{2\pi}\sigma_z}
\frac{x}{\sigma_x(\sigma_x+\sigma_y)}
\]
\begin{equation}
\Delta y'=\frac{K_{sc}Le^{-z^2/(2\sigma_z^2)}}{\sqrt{2\pi}\sigma_z}
\frac{y}{\sigma_y(\sigma_x+\sigma_y)}
\end{equation}
where $K_{sc}=\frac{2Nr_e}{\gamma^3\beta^2}$,
$L$ is the integrating length, $\sigma_{x,y,z}$ are rms beam size.

The non-linear SC force is given by:
\[
\Delta x'=\frac{K_{sc}Le^{-z^2/(2\sigma_z^2)}}{2\sigma_z\sqrt{\sigma_x^2-\sigma_y^2}}
Im\left [ w\left( \frac{x+iy}{\sqrt{2(\sigma_x^2-\sigma_y^2)}} \right)
-e^{-\frac{x^2}{2 \sigma_x^2}-\frac{y^2}{2 \sigma_y^2}}
w\left(\frac{x\frac{\sigma_y}{\sigma_x}+iy\frac{\sigma_x}{\sigma_y}}
{\sqrt{2(\sigma_x^2-\sigma_y^2)}}\right)\right ]
\]
\begin{equation}
\Delta y'=\frac{K_{sc}Le^{-z^2/(2\sigma_z^2)}}{2\sigma_z\sqrt{\sigma_x^2-\sigma_y^2}}
Re\left [ w\left( \frac{x+iy}{\sqrt{2(\sigma_x^2-\sigma_y^2)}} \right)
-e^{-\frac{x^2}{2 \sigma_x^2}-\frac{y^2}{2 \sigma_y^2}}
w\left(\frac{x\frac{\sigma_y}{\sigma_x}+iy\frac{\sigma_x}{\sigma_y}}
{\sqrt{2(\sigma_x^2-\sigma_y^2)}}\right)\right ]
\label{equa2}
\end{equation}
where $w(z)$ is the complex error function
\begin{equation}
w(z)=e^{-z^2}\left [ 1+\frac{2i}{\sqrt{\pi}}\int\limits_0^z e^{\zeta^2}d\zeta\right ]
\end{equation}
Equation~\ref{equa2} appear to diverge when $\sigma_x=\sigma_y$. In fact, this is not
true, because the expressions inside the square brackets will provide zero too at
$\sigma_x=\sigma_y$ to cancel the poles outside. In our code, we calculate this equation
at  $1.01 \sigma_x$ and $0.99\sigma_x$, and average the total effects. 
 
To invoke the calculation, one must use set up command
``insert\_sceffects'' proceed ``run\_setup'' and ``Twiss\_output''
command proceed ``track''.
\end{document}
