This element allows expanding {\tt elegant} by using external scripts
(or programs) as elements in a beamline.    Here are requirements for the 
script:
\begin{itemize}
\item It must be executable from the commandline.
\item It must read the initial particle distribution from an SDDS file.
This file will have the usual columns that an {\tt elegant} phase-space
output file  has, along with the parameter {\tt Charge} giving the 
beam charge in Coulombs. The file will contain a single data page.
\item It must write the final particle distribution to an SDDS file.
This file should have all of the columns and parameters that appear in the
initial distribution file.  Additional columns and parameters will be ignored,
as will all pages but the first. 
\item The {\tt Charge} parameter in the file is used to determine the total beam charge; the script must ensure that
  this parameter is set correctly; when particles are lost or created, simply copying or retaining the value from the
  input file will not be correct. Normally, the charge per particle is constant in simulations. Hence, if {\tt elegant}
  sees a change in charge per particle after the {\tt SCRIPT} element, it issues a warning.
\end{itemize}

The {\tt SCRIPT} element works best if the script accepts commandline
arguments.  In this case, the {\tt COMMAND} parameter is used to
provide a template for creating a command to run the script.  The {\tt COMMAND}
string may contain the following substitutable fields:
\begin{enumerate}
\item \verb|%i| --- Will be replaced by the name of the input file to the script.
({\tt elegant} writes the initial particle distribution to this file.)
\item \verb|%o| --- Will be replaced by the name of the output file from the script.
({\tt elegant} expects the script to write the final particle distribution to this file.)
\item \verb|%p| --- Will be replaced by the pass number, which starts from 0.
\item \verb|%np0|, \verb|%np1|, ..., \verb|%np9| --- Will be replaced by the value of
 Numerical Parameter 0, 1, ..., 9.  This can be used to pass to the script values that
 are parameters of the element definition.  For example, if one wanted to vary parameters 
 or add errors to the parameter, one would use this facility.
\item \verb|%sp0|,  \verb|%sp1|, ..., \verb|%sp9| --- Will be replaced by the value of
 String Parameter 0, 1, ..., 9.  This can be used to pass to the script values that
 are parameters of the element definition. 
\end{enumerate}

In some cases, one may wish to keep the input file delivered to the \verb|SCRIPT| as
well as the output file returned by it. This is facilitated by using the \verb|ROOTNAME| 
parameter, which allows specifying the rootname for these files, as well as the
\verb|INPUT_EXTENSION| and \verb|OUTPUT_EXTENSION| parameters.
The \verb|ROOTNAME| parameter may contain a simple string, but may also contain several
sustainable fields:
\begin{itemize}
\item \verb|%s| --- The global rootname, which may be given by the \verb|rootname| parameter
  in the \verb|run_setup| command.
\item \verb|%p| --- The pass index.
\item \verb|%ld| --- The occurence number of the element.
\end{itemize}

Here's an example of a {\tt SCRIPT} {\tt COMMAND}:
\begin{flushleft}
\begin{verbatim}
myScript -input %i -output %o -accuracy %np0 -type %sp0
\end{verbatim}
\end{flushleft}
In this example, the script {\tt myScript} takes four commandline arguments, giving
the names of the input and output files, an accuracy requirement, and a type specifier.
By default, {\tt elegant} will choose unique, temporary filenames to use in communicating
with the script.  The actual command when executed might be something like
\begin{flushleft}
\begin{verbatim}
myScript -input tmp391929.1 -output tmp391929.2 -accuracy 1.5e-6 -type scraper
\end{verbatim}
\end{flushleft}
where for this example I've assumed {\tt NP0=1.5e-6} and {\tt SP0=''scraper''}.

If you have a program (e.g., a FORTRAN program) that does not accept
commandline arguments, you can easily wrap it in a Tcl/Tk simple script to
handle this.  Alternatively, you can force {\tt elegant} to use specified 
files for communicating with such a script.  This is done using the {\tt ROOTNAME},
{\tt INPUT\_EXTENSION}, and {\tt OUTPUT\_EXTENSION} parameters.
So if your program was {\tt crass} and it expected its input (output) in files
{\tt crass.in} ({\tt crass.out}), then you'd use
\begin{flushleft}
\begin{verbatim}
S1: script,command=''crass'',rootname=''crass'',input_extension=''in'',&
output_extension=''out''
\end{verbatim}
\end{flushleft}

For purposes of computing concatenated transport matrices, Twiss
parameters, response matrices, etc., {\tt elegant} will perform
initial tracking through the \verb|SCRIPT| element using an ensemble
of 25 particles.  If this is not desirable, then set the parameter
\verb|DRIFT_MATRIX| to a non-zero value.  This will force
\verb|elegant| to treat the element as a drift space for any
calculations that involve transport matrices.  Examples of where one
might want to use this feature would be a \verb|SCRIPT| that involves
randomization (e.g., scattering), particle loss, or particle creation.

