This element provides an alternative to element-by-element modeling of synchrotron radiation.
There are several instances in which one may want to use this element:
\begin{itemize}
\item Simulation of instabilities or other dynamics in which quantum excitation and radiation
  damping play a role.
\item Simulation of dynamics with an rf cavity when the synchronous phase is significantly
  different from 180 degrees, so that average radiation losses must be included.
\item Computation of dynamic and momentum aperture in the presence of radiation damping.
\end{itemize}

The major parameters (\verb|JX|, \verb|JY|, \verb|EXREF|, \verb|SDELTAREF|,
\verb|DDELTAREF|, and \verb|PREF|) can be supplied by the user, or filled in by {\tt elegant}
if the \verb|twiss_output| command is given with \verb|radiation_integrals=1|.
Because the radiation integrals computation in \verb|twiss_output| pertains to the
horizontal plane only, the user must supply either \verb|EYREF| or \verb|COUPLING| if
non-zero vertical emittance is desired.

The user may elect to turn off some aspects of the synchrotron radiation model.  These should be
changed from the default values with care!
\begin{itemize}
\item \verb|DAMPING| --- Default is 1.  If set to 0, then no radiation damping effects will be included.
  More precisely, it is equivalent to setting \verb|JX=JY=JDELTA=1|.  Damping still occurs at any
  rf cavities (since {\tt elegant} works in trace space).
\item \verb|QEXCITATION| --- Default is 1.  If set to 0, then no quantum excitation effects are included,
  which is to say that all particles will experience the same perturbation.  
\item \verb|LOSSES| --- Default is 1.  If set to 0, no average energy losses are included.
\end{itemize}
  

