Bending magnet RAy Tracing using (Bx, By, Bz) vs (x, y, z).
This element is a companion to the commandline program {\tt abrat}.
It integrates through a 3-D field map for a bending magnet, including
coordinate transformations.
No synchrotron radiation calculations are included at this time.

{\bf Coordinates}

The coordinates of the field map are right-handed system (x, y, z), where z is along the length of the magnet, x is 
to the right as viewed along the direction of beam propagation, and y is up.
The user must specify the (x, z) coordinates of three points:
\begin{itemize}
\item Nominal entrance point: \verb|XENTRY| and \verb|ZENTRY|. These give the coordinates of reference trajectory at
  the exit of the previous element. In the limit of a hard-edge model, this would be at the entrance to the magnetic field 
  region.
\item Vertex point: \verb|XVERTEX| and \verb|ZVERTEX|. These give the coordinates of vertex point, which is the intersection
  of the reference lines from the entrance and exit.
\item Nominal exit point: \verb|XEXIT| and \verb|ZEXIT|. These give the coordinates of reference trajectory at
  the exit of the previous element. In the limit of a hard-edge model, this would be at the exit from the magnetic field 
  region.
\end{itemize}
The bending angle is equal to the angle between two lines: the line from \verb|ENTRY| to \verb|VERTEX| and the
line from \verb|VERTEX| to \verb|EXIT|.
The \verb|L| and \verb|ANGLE| parameters supplied by the user are used for geometry calculations (e.g., floor coordinates) only.

The \verb|DXMAP|, \verb|DZMAP|, \verb|YAWMAP|, and \verb|FSE| values can be used to optimize the field map to ensure that
the horizontal reference trajectory is not displaced at the exit of the element.
The optimization feature of the \verb|abrat| program can be used to determine these values.

{\bf Matrix generation}

{\tt elegant} will use tracking to determine the transport matrix for \verb|BRAT| elements, which 
is needed for computation of twiss parameters and other operations.
This can require some time, so {\tt elegant} will cache the matrices and re-use them for
identical elements.

If matrices are not of particular interest, significant time savings can be realized by setting
\verb|USE_SBEND_MATRIX=1|. Of course, any matrix-based results (e.g., twiss parameters) are then
dubious at best.

{\bf Symmetry}

By default, the \verb|BRAT| element should be supplied with the full 3D field map of the magnet.
To allow saving memory and reducing the file to load data, partial magnetic field maps can be 
loaded as well, but the user must specify the symmetry of the magnet to ensure that the fields are
modeled correctly in the full volume.
This is done using the \verb|ySymmetry| parameter in the input file, which may have one of three
values: \verb|none| (default), \verb|even|, and \verb|odd|.  
A normal (upright) multipole magnet would have
\verb|ySymmetry=odd|, while a skew multipole would have \verb|ySymmetry=even|.

Note that when using these symmetries, the user is not required to limit the field map to
verb|y\geq 0|, though doing so saves the most memory.
If possible, it is recommended to provide the fields over \verb|y\geq -2\Delta y|,
so that the transverse interpolation as a few points on both sides of $y=0$.
This will ensure better results near the origin.

{\bf Integration methods}

The original (and default) integration method is Bulirsch-Stoer integration of the Lorentz force equation.
As an alternative, one can use the faster, rotation-based method of the \verb|FTABLE| element.
For repeated use, the two methods should be compared and a choice made based on the user's needs.
