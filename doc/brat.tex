Bending magnet RAy Tracing using (Bx, By, Bz) vs (x, y, z).
This element is a companion to the commandline program {\tt abrat}.
It integrates through a 3-D field map for a bending magnet, including
coordinate transformations.
No synchrotron radiation calculations are included at this time.

{\bf Coordinates}

The coordinates of the field map are right-handed system (x, y, z), where z is along the length of the magnet, x is 
to the right as viewed along the direction of beam propagation, and y is up.
The user must specify the (x, z) coordinates of three points:
\begin{itemize}
\item Nominal entrance point: \verb|XENTRY| and \verb|ZENTRY|. These give the coordinates of reference trajectory at
  the exit of the previous element. In the limit of a hard-edge model, this would be at the entrance to the magnetic field 
  region.
\item Vertex point: \verb|XVERTEX| and \verb|ZVERTEX|. These give the coordinates of vertex point, which is the intersection
  of the reference lines from the entrance and exit.
\item Nominal exit point: \verb|XEXIT| and \verb|ZEXIT|. These give the coordinates of reference trajectory at
  the exit of the previous element. In the limit of a hard-edge model, this would be at the exit from the magnetic field 
  region.
\end{itemize}
The bending angle is equal to the angle between two lines: the line from \verb|ENTRY| to \verb|VERTEX| and the
line from \verb|VERTEX| to \verb|EXIT|.
The \verb|L| and \verb|ANGLE| parameters supplied by the user are used for geometry calculations (e.g., floor coordinates) only.

{\bf Matrix generation}

{\tt elegant} will use tracking to determine the transport matrix for \verb|CCBEND| elements, which 
is needed for computation of twiss parameters and other operations.
This can require some time, so {\tt elegant} will cache the matrices and re-use them for
identical elements.

