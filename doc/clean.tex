The three options and their mode of operation are as follows:
\begin{itemize}

\item \verb|ABSDEV| : compute the mean of the coordinate values, then compute absolute value of difference
between the mean and the coordinate value for each particle. If this absolute deviation exceeds
the user-specified limit, then the particle is removed. This could be used, for
example, to remove particles outside of 100ps of the mean arrival time.

\item \verb|STDEVIATION|: compute the mean and standard deviation of the
coordinate values, then compute the absolute value of difference between the
mean and the coordinate value for each particle, normalized by the standard
deviation. If this value exceeds the user-specified limit, then the
particle is removed. This could be used, for example, to remove
particles outside of five sigma of the horizontal beam size from the centroid.

\item \verb|ABSVALUE|: compare the absolute value of the particle coordinate value to the
user-specified limit. If it exceeds this limit, then the particle is removed.
This could be used, for example, to remove particles with slopes that exceed 100 mrad.

\end{itemize}
