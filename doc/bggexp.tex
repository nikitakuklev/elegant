This element simulates transport through a 3D magnetic field
specified in terms of a generalized gradient expansion \cite{Venturini-NIMA427-387}.
After reconstructing the field, it simply integrates the equations of motion
based on the Lorentz force equation in cartesian coordinates.  It does not
incorporate changes in the design trajectory resulting from the
fields, so dipoles cannot be simulated with this element.

The generalized gradients are provided in an SDDS file with the following floating-point columns:
\begin{itemize}
\item {\bf z} --- Longitudinal coordinate. Units should be ``m''.
\item {\bf Cnm{\em n}} --- The $n^{th}$ generalized gradient of the $m^{th}$ harmonic, where $n=0,2,4,...$.
  There is no preset limit to the number of generalized gradients. Units are ignored,
  but should be SI.
\item {\bf dCnm{\em n}/dz} --- The longitudial derivative of the $n^{th}$ generalized gradient, 
  for the $m^{th}$ harmonic, where $n=0,2,4,...$.
  The number of derivatives must match the number of generalized gradients {\bf Cnm{\em n}}.
\end{itemize}
In addition, the file must contain a parameter:
\begin{itemize}
\item {\bf m} --- The multipole index, using the convention where $m=1$ is dipole, $m=2$ is quadrupole,
  etc. N.B.: this convention conforms with \cite{Venturini-NIMA427-387} but is not the usual one used by
  {\tt elegant}. This should be stored as a short integer.
  N. B.: for $m=1$, if the system has net bending, the results will not be correct as the required coordinate
  transformations are not performed.
\end{itemize}

The generalized gradient file can be prepared using the script {\tt computeGeneralizedGradients}, which is
provided with {\tt elegant}. The input file for that script must be organized into many pages, with
each page giving $B_r(\phi)$ on radius $r=R$ for a single $z$ location. The file must contain two floating-point columns:
\begin{itemize}
\item {\bf phi} --- The angle, in radians. $\phi=0$ corresponds to $x=R$ and $y=0$, while $\phi=\pi/2$ corresponds
  to $x=0$ and $y=R$. It is assumed that $\phi$ runs from $0$ to $2\pi - \Delta \phi$ in steps of $\Delta \phi$.
\item {\bf Br} --- The radial field at the reference radius R, Tesla.
\end{itemize}
In addition, the file must contain two floating-point parameters:
\begin{itemize}
\item {\bf R} --- The radius, in meters. 
\item {\bf z} --- The longitudinal coordinate, in meters. $z$ should extend from the zero-field region upstream of the magnet to 
  the zero-field region downstream of the magnet.
\end{itemize}
