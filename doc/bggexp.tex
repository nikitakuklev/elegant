This element simulates transport through a 3D magnetic field
specified in terms of a generalized gradient expansion \cite{Venturini-NIMA427-387}.
After reconstructing the field, it simply integrates the equations of motion
based on the Lorentz force equation in cartesian coordinates.  

The generalized gradients are provided in SDDS files.
In addition to several columns describing the gradients, the file must contain a parameter:
\begin{itemize}
\item \verb|m| --- The multipole index, using the convention where $m=0$ is solenoid, $m=1$ is dipole, $m=2$ is quadrupole,
  etc. N.B.: this convention conforms with \cite{Venturini-NIMA427-387} but is not the usual one used by
  {\tt elegant}. This should be stored as a short integer.
\end{itemize}
The files may also include optional parameters \verb|xCenter| and \verb|yCenter| giving the center of the expansion
in meters.

In the original implementation, which is still supported, only normal field components were included
In that case, the user should use the \verb|FILENAME| field to provide a file 
with the following floating-point columns:
\begin{itemize}
\item \verb|z| --- Longitudinal coordinate. Units should be ``m''.
\item \verb|Cnm|{\em n} --- The $n^{th}$ generalized gradient of the $m^{th}$ harmonic, where $n=0,2,4,...$.
  There is no preset limit to the number of generalized gradients. Units are ignored,
  but should be SI.
\item \verb|dCnm|{\em n}\verb|/dz| --- The longitudial derivative of the $n^{th}$ generalized gradient, 
  for the $m^{th}$ harmonic, where $n=0,2,4,...$.
  The number of derivatives must match the number of generalized gradients {\bf Cnm{\em n}}.
\end{itemize}
The field expansion in this case is
\begin{equation}
\begin{array}{lcl}
B_r & = & \sum\limits_{m=1}^\infty \sum\limits_{n=0}^\infty \frac{(-1)^n m! (2n + m)}{4^n n! (n+m)!} r^{2 n + m-1}
\left\{C_m^{\left[2n\right]}(z) \sin m\phi\right\}  \\
B_\phi & = & \sum\limits_{m=1}^\infty \sum\limits_{n=0}^\infty \frac{(-1)^n m! (2 n + m)}{4^n n! (n+m)!} r^{2 n + m-1}
\left\{C_m^{\left[2n\right]}(z) \cos m\phi\right\} \\
B_z & = & \sum\limits_{m=0}^\infty \sum\limits_{n=0}^\infty \frac{(-1)^n m!}{4^n n! (n+m)!} r^{2 n + m}
\left\{C_m^{\left[2n+1\right]}(z) \sin m\phi\right\}
\end{array}
\end{equation}
where it is understood that the expansion is about the {\tt xCenter} and {\tt yCenter} values, if given.

Note that there is potential confusion between the {\tt xCenter} parameter in the input files
and the \verb|DXEXPANSION| parameter in the element definition. 
These provide similar functionality and only one is needed.
Both give the position of the horizontal center of the expansion relative to the magnetic field coordinate system.

In version 2020.5 and later, both normal and skew expansions are supported.
In this case, the user may provide filenames via the \verb|NORMAL_FILENAME| and \verb|SKEW_FILENAME|
fields.
In this, case, the files must contain the following floating-point columns:
\begin{itemize}
\item {\tt z} --- Longitudinal coordinate. Units should be ``m''.
\item {\tt CnmS}{\em n} (normal) or {\tt CnmC}{\em n} (skew) --- The $n^{th}$ generalized gradient of the $m^{th}$
  harmonic, where $n=0,2,4,...$.  There is no preset limit to the number of generalized gradients. Units are ignored,
  but should be SI.  Note that the ``S'' in the name for the normal components make be confusing. It refers to the fact
  that these terms appear in the potential with a factor of $\sin m\phi$, whereas the skew terms have $\cos m\phi$
  factors (hence the ``C'').
\item {\tt dCnmS}{\em n}{\tt /dz} (normal) or {\tt dCnmC}{\em n}{\tt /dz} (skew) --- The longitudial derivative of the
  $n^{th}$ generalized gradient, for the $m^{th}$ harmonic, where $n=0,2,4,...$.  The number of derivatives must match
  the number of generalized gradients {\bf Cnm{\em n}}.
\end{itemize}
The field expansion in this case is
\begin{equation}
\begin{array}{lcl}
B_r & = & \sum\limits_{m=1}^\infty \sum\limits_{n=0}^\infty \frac{(-1)^n m! (2n + m)}{4^n n! (n+m)!} r^{2 n + m-1}
\left\{C_{m,s}^{\left[2n\right]}(z) \sin m\phi + C_{m,c}^{\left[2n\right]}(z) \cos m\phi\right\} +
\sum\limits_{n=1}^\infty \frac{(-1)^n 2 n}{4^n n! n!} r^{2n-1} C_{0,c}^{\left[2n\right]}(z) \\
B_\phi & = & \sum\limits_{m=1}^\infty \sum\limits_{n=0}^\infty \frac{(-1)^n m! (2 n + m)}{4^n n! (n+m)!} r^{2 n + m-1}
\left\{C_{m,s}^{\left[2n\right]}(z) \cos m\phi - C_{m,c}^{\left[2n\right]}(z) \sin m\phi\right\} \\
B_z & = & \sum\limits_{m=0}^\infty \sum\limits_{n=0}^\infty \frac{(-1)^n m!}{4^n n! (n+m)!} r^{2 n + m}
\left\{C_{m,s}^{\left[2n+1\right]}(z) \sin m\phi + C_{m,c}^{\left[2n+1\right]}(z) \cos m\phi\right\}
\end{array}
\end{equation}
where it is understood that the expansion is about the {\tt xCenter} and {\tt yCenter} values, if given.
Users should note that the skew field sign convention used by \cite{Venturini-NIMA427-387} and \verb|BGGEXP| differs
from that used in {\tt elegant}.  In particular, to convert a normal field to a skew field while conforming to {\tt elegant}'s
conventions, one must use $C_{m,s}^{p} \rightarrow -C_{m,c}^{p}$ and  $dC_{m,s}^{p}/dz \rightarrow -dC_{m,c}^{p}/dz$.

Data for use with \verb|BGGEXP| can be prepared with the programs \verb|computeCBGGE| (section \ref{computeCBGGE})
and \verb|computeRBGGE| (section \ref{computeRBGGE}),
which are distributed with \verb|elegant|.

Synchrotron radiation can be included by setting \verb|SYNCH_RAD=1| for classical radiation only and
also \verb|ISR=1| for incoherent (quantum) effects. 
This will impact the results of \verb|moments_output| calculation as well as tracking.

Important notes and limitations:
\begin{enumerate}
\item The calculations of \verb|twiss_output|, including radiation integrals, are at this point not affected,
nor is the setup of rf cavities for storage rings via the \verb|rf_setup| command.
\item The symplectic integrator, in addition to being symplectic, is typically more accurate than the non-symplectic integrator.
  It is also considerably slower.
  However, at minimum, users should use the symplectic integrator to verify that the accuracy of the non-symplectic integrator
  is adequate.
\item The \verb|BX| and \verb|BY| parameters allow imposing uniform horizontal and vertical magnetic fields 
  on the device.
  This can be helpful if the terminal trajectory deviates from the expected value, e.g., an on-axis particle  
  ends up off-axis. This may happen if the device has
  a dipolar field that is truncated at the ends before it has decayed sufficiently.
  Note that these values are multiplied by the \verb|STRENGTH| factor before being applied to the beam.
\end{enumerate}

In addition to the \verb|STRENGTH| factor, there are five parameters that can be used to scale multipoles
of different orders: \verb|FACTOR0|, \verb|FACTOR1|, \verb|FACTOR2|, \verb|FACTOR3|, and \verb|FACTOR4| scale the
solenoidal, dipolar, quadrupolar, sextupolar, and octupolar fields, respectively.
The \verb|BXFACTOR|, \verb|BYFACTOR|, and \verb|BZFACTOR| allow multiplying the indicated field components by the given factors.
The the exception of the \verb|STRENGTH| factor, these scaling parameters may result in unphysical fields.

If \verb|IS_BEND| is non-zero, the magnet is assumed to be a bending magnet, in which case additional
parameters are required.
\begin{itemize}
\item \verb|ZVERTEX|, \verb|XVERTEX| --- Coordinates of the vertex point in coordinate frame of the field data.
  For a symmetric dipole, \verb|ZVERTEX| is typically zero, while \verb|XVERTEX| would be the displacement of
  the vertex point from the cylinder axis.
\item \verb|ZENTRY|, \verb|XENTRY| --- Coordinates of the nominal entry plane.
\item \verb|ZEXIT|, \verb|XEXIT| --- Coordinates of the nominal exit plane.
\end{itemize}
