This element simulates transport through a 3D magnetic field
specified in terms of a generalized gradient expansion \cite{Venturini-NIMA427-387}.
After reconstructing the field, it simply integrates the equations of motion
based on the Lorentz force equation in cartesian coordinates.  

The generalized gradients are provided in an SDDS file with the following floating-point columns:
\begin{itemize}
\item {\bf z} --- Longitudinal coordinate. Units should be ``m''.
\item {\bf Cnm{\em n}} --- The $n^{th}$ generalized gradient of the $m^{th}$ harmonic, where $n=0,2,4,...$.
  There is no preset limit to the number of generalized gradients. Units are ignored,
  but should be SI.
\item {\bf dCnm{\em n}/dz} --- The longitudial derivative of the $n^{th}$ generalized gradient, 
  for the $m^{th}$ harmonic, where $n=0,2,4,...$.
  The number of derivatives must match the number of generalized gradients {\bf Cnm{\em n}}.
\end{itemize}
In addition, the file must contain a parameter:
\begin{itemize}
\item {\bf m} --- The multipole index, using the convention where $m=1$ is dipole, $m=2$ is quadrupole,
  etc. N.B.: this convention conforms with \cite{Venturini-NIMA427-387} but is not the usual one used by
  {\tt elegant}. This should be stored as a short integer.
  N. B.: for $m=1$, if the system has net bending, the results will not be correct as the required coordinate
  transformations are not performed.
\end{itemize}

The generalized gradient file can be prepared using the script {\tt computeGeneralizedGradients}, which is
provided with {\tt elegant}. The input file for that script must be organized into many pages, with
each page giving $B_r(\phi)$ on radius $r=R$ for a single $z$ location. The file must contain two floating-point columns:
\begin{itemize}
\item {\bf phi} --- The angle, in radians. $\phi=0$ corresponds to $x=R$ and $y=0$, while $\phi=\pi/2$ corresponds
  to $x=0$ and $y=R$. It is assumed that $\phi$ runs from $0$ to $2\pi - \Delta \phi$ in steps of $\Delta \phi$.
\item {\bf Br} --- The radial field at the reference radius R, Tesla.
\end{itemize}
In addition, the file must contain two floating-point parameters:
\begin{itemize}
\item {\bf R} --- The radius, in meters. 
\item {\bf z} --- The longitudinal coordinate, in meters. $z$ should extend from the zero-field region upstream of the magnet to 
  the zero-field region downstream of the magnet.
\end{itemize}

Synchrotron radiation can be included by setting \verb|SYNCH_RAD=1| for classical radiation only and
also \verb|ISR=1| for incoherent (quantum) effects. 
This will impact the results of \verb|moments_output| calculation as well as tracking.

Important notes and limitations:
\begin{enumerate}
\item The calculations of \verb|twiss_output|, including radiation integrals, are at this point not affected,
nor is the setup of rf cavities for storage rings via the \verb|rf_setup| command.
\item The symplectic integrator, in addition to being symplectic, is typically more accurate than the non-symplectic integrator.
  It is also considerably slower.
  However, at minimum, users should use the symplectic integrator to verify that the accuracy of the non-symplectic integrator
  is adequate.
\item The \verb|BX| and \verb|BY| parameters allow imposing uniform horizontal and vertical magnetic fields 
  on the device.
  This can be helpful if the terminal trajectory deviates from the expected value, e.g., an on-axis particle  
  ends up off-axis. This may happen if the device has
  a dipolar field that is truncated at the ends before it has decayed sufficiently.
  Note that these values are multiplied by the \verb|STRENGTH| factor before being applied to the beam.
\end{enumerate}

If \verb|IS_BEND| is non-zero, the magnet is assumed to be a bending magnet, in which case additional
parameters are required.
\begin{itemize}
\item \verb|ZVERTEX|, \verb|XVERTEX| --- Coordinates of the vertex point in coordinate frame of the field data.
  For a symmetric dipole, \verb|ZVERTEX| is typically zero, while \verb|XVERTEX| would be the displacement of
  the vertex point from the cylinder axis.
\item \verb|ZENTRY|, \verb|XENTRY| --- Coordinates of the nominal entry plane.
\item \verb|ZEXIT|, \verb|XEXIT| --- Coordinates of the nominal exit plane.
\end{itemize}
