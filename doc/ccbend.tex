This element provides a symplectic straight-pole, bending magnet with the exact
Hamiltonian in Cartesian coordinates.  
The quadrupole, sextupole, and other multipole terms are defined in Cartesian coordinates.
The magnet at present is restricted to having rectangular ends with symmetric entry and
exit.
This is quite different from \verb|CSBEND|, where the edge angles are user-defined and
where the field expansion is in curvilinear coordinates.
Strictly speaking, \verb|CSBEND| is only valid when the dipole is built with curved,
beam-following poles.

Integration of particles in \verb|CCBEND| is very similar to what's done for
\verb|KQUAD|, \verb|KSEXT|, and \verb|KOCT|. 
The only real difference is that coordinate transformations are performed at the
entrance and exit to orient the incoming central trajectory to the straight magnet axis.
In addition, the fractional strength error is adjusted to ensure that the 
outgoing central trajectory is correct.

By default, two adjustments are made at start-up and whenever the length, angle, 
gradient, or sextupole term change:
\begin{enumerate}
\item The fractional strength error is altered to ensure the correct deflecting angle. 
      This is required because the bending field varies along the trajectory.
      By default, this affects all field 
      components together, per the usual convention in {\tt elegant}. To restrict
      the strength change to the dipole term, set \verb|COMPENSATE_KN=1|.
\item The transverse position is adjusted to center the trajectory in the magnet.
      If the sagitta is $\sigma$ and \verb|ANGLE| is positive, the initial and final $x$ 
      coordinates are $x=-\sigma/2$, while the center coordinate is $x=\sigma/2$.
\end{enumerate}
One can block the re-optimization of these parameters by setting 
\verb|OPTIMIZE_FSE_ONCE| and \verb|OPTIMIZE_DX_ONCE| to 1.

{\bf Edge angles and edge effects} 

The edge angle treatment in \verb|CCBEND| is relatively simple, consisting of a 
vertical focusing effect with momentum dependence to all orders.
Also included are edge pseudo-sextupoles (due to the body $K_1$ term) and
pseudo-octupoles (due to the body $K_2$ term).
The user may also specify edge multipoles using the \verb|EDGE_MULTIPOLE| parameter.

{\bf Multipole errors}

Multipole errors are specified for the body and edge in the same fashion as for the
\verb|KQUAD| element.
The reference is the dipole field by default, but this may be changed using the
\verb|REFERENCE_ORDER| parameter.

{\bf Radiation effects}

Incoherent synchrotron radiation, when requested with {\tt ISR=1},
normally uses gaussian distributions for the excitation of the electrons.
Setting {\tt USE\_RAD\_DIST=1} invokes a more sophisticated algorithm that
uses correct statistics for the photon energy and number distributions.
In addition, if {\tt USE\_RAD\_DIST=1} one may also set {\tt ADD\_OPENING\_ANGLE=1},
which includes the photon angular distribution when computing the effect on 
the emitting electron.  

{\bf Adding errors}

When adding errors, care should be taken to choose the right
parameters.  The \verb|FSE| and \verb|ETILT| parameters are used for
assigning errors to the strength and alignment relative to the ideal
values given by \verb|ANGLE| and \verb|TILT|.  One can also assign 
errors to \verb|ANGLE| and \verb|TILT|, but this has a different meaning:
in this case, one is assigning errors to the survey itself.  The reference
beam path changes, so there is no orbit/trajectory error. The most common
thing is to assign errors to \verb|FSE| and \verb|ETILT|.  Note that when
adding errors to \verb|FSE|, the error is assumed to come from the power
supply, which means that multipole strengths also change.

{\bf Splitting dipoles}

The \verb|CCBEND| element does not support splitting.
{\bf Important}: Users {\em should not} attempt to split \verb|CCBEND| elements by hand, since this
will not result in the correct geometry entering and exiting the various parts.

{\bf Matrix generation}

{\tt elegant} will use tracking to determine the transport matrix for \verb|CCBEND| elements, which 
is needed for computation of twiss parameters and other operations.
This can require some time, so {\tt elegant} will cache the matrices and re-use them for
identical elements.

