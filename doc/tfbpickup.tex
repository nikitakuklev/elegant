This element is used together with the {\tt TFBDRIVER} element to
simulate a digital turn-by-turn feedback system.  Each {\tt TFBPICKUP}
element must have a unique identification string assigned to it using
the {\tt ID} parameter.  This is used to identify which drivers get
signals from the pickup.

A 30-term FIR filter can be defined using the {\tt A0} through {\tt
A29} parameters.  The input to the filter is the turn-by-turn beam
centroid at the pickup location.  The output of the filter is simply
$\sum_{i=0}^{29} a_i C_i$, where $C_i$ is the centroid from $i*U$ turns
ago, where $U$ is the value specified by the \verb|UPDATE_INTERVAL| parameter.
Note that $\sum_{i=0}^{29} a_i$ should generally be zero. Otherwise, the
system will attempt to correct the DC orbit.  The output of the filter
is the input to the driver element(s).

The \verb|PLANE| parameter can take three values: ``x'', ``y'', and ``delta'', specifying
what centroid property of the beam is measured by the pickup. The ``delta''-mode pickup
is nonphysical, but could have applications to cases where is not convenient to put a 
pickup in a high-dispersion area.

See Section 7.2.14 of {\em Handbook of Accelerator Physics and Engineering}
(Chao and Tigner, eds.) for a discussion of feedback systems.
