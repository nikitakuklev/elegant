This element is used together with the {\tt TFBPICKUP} element to
simulate a digital turn-by-turn feedback system.  Each {\tt TFBDRIVER}
element must have a unique identification string assigned to it using
the {\tt ID} parameter.  The same identifier must be used on a {\tt
TFBPICKUP} element.  This is the pickup from which the driver gets its
signal.  Each pickup may feed more than one driver, but a driver can
use only one pickup.

A 30-term FIR filter can be defined using the {\tt A0} through {\tt
A29} parameters.  The output of the filter is simply $\sum_{i=0}^{29}
a_i P_i$, where $P_i$ is the pickup filter output from $i*U$ turns ago,
where $U$ is the \verb|UPDATE_INTERVAL| value specified for the pickup.
The output of the filter is optionally delayed by the number of update intervals
given by the {\tt DELAY} parameter.

To some extent, the {\tt DELAY} is redundant.  For example, the filter
$a_0=0, a_1=1$ with a delay of 0 is equivalent to $a_0=1, a_1=0$ with
a delay of 1.  However, for long delays or delays combined with
many-term filters, the {\tt DELAY} feature must be used.

The output of the filter is multiplied by the {\tt STRENGTH} parameter
to get the kick to apply to the beam.  The {\tt KICK\_LIMIT} parameter
provides a very basic way to simulate saturation of the kicker output.

The plane that the \verb|TFBDRIVER| kicks is determined by the 
\verb|PLANE| parameter on the corresponding \verb|TFBPICKUP| element, and
additionally by the \verb|LONGITUDINAL| parameter, as described in 
Table \ref{tab:tfbdriver}

\begin{table}[htb]
\begin{tabular}{llll}
\hline
\verb|TFBPICKUP| & \verb|TFBDRIVER| & coordinate & note \\
\verb|PLANE| & \verb|LONGITUDINAL| & kicked & \\
\hline
x & 0 & $x^\prime$ & \\
x & 1 & $\delta$ & pickup should have $\eta_x\neq 0$ \\
y & 0 & $y\prime$ & \\
y & 1 & $\delta$ & pickup should have $\eta_y\neq 0$ \\
delta & 0 & - & invalid \\
delta & 1 & $\delta$ & \\
\hline
\end{tabular}
\caption{Correspondence between {\tt PLANE} parameter of {\tt TFBPICKUP}, {\tt LONGITUDINAL} parameter of {\tt TFBDRIVER}, and action of feedback loop.}
\label{tab:tfbdriver}
\end{table}

Note: The \verb|OUTPUT_FILE| will produce a file with missing data at the end of
the buffer if the \verb|OUTPUT_INTERVAL| parameter is not a divisor of the number of passes.

The \verb|FREQUENCY| and \verb|PHASE| parameters may be used to specify the resonant frequency of 
the driving cavity and its phase relative to the center of the bunch.
If the frequency is not specified, the kicker is assumed to kick all particles in a bunch by the
same amount.

For longitudinal feedback only, a more sophicated approach is available using a
circuit model developed by T. Berenc (APS) may be employed to simulate driving the cavity resonance.
To invoke this, the user must provide the loaded Q of the cavity using the \verb|QLOADED| parameter,
the $(R_a/Q)$ using \verb|RAOVERQ|, and the resonant frequency of the unloaded 
cavity using \verb|FREQUENCY|. Optionally, the drive frequency may be specified using
\verb|DRIVE_FREQUENCY|; it defaults to the unloaded resonant frequency.

Typically one should choose the resonant frequency to be $(n\pm \frac{1}{4})f_b$, where
$f_b$ is the bunch frequency and $n$ is an integer.
This will ensure that the kick to one bunch from the residual voltage from the previous
bunch (both beam-loading and generator terms), is approximately minimized.
Checking the \verb|ResidualVoltage| column in the output file to confirm this is advised.

In addition to the resonant and drive frequencies, one must specify a clock frequency with
\verb|CLOCK_FREQUENCY| and a clock offset with \verb|CLOCK_OFFSET|. The clock used used to
determine when the drive current changes, which happens at regular intervals. The clock offset is used to 
ensure that the change does not occur during passage of the bunch. If the clock offset is too small and
the bunch length too long, this will happen and results in an error.
The phase shift that results from the clock offset is automatically compensated.

Beam loading is not included in the model, but can be superimposed by inserting an \verb|RFMODE| element
with matching parameters.


See Section 7.2.14 of {\em Handbook of Accelerator Physics and Engineering}
(Chao and Tigner, eds.) for a discussion of feedback systems.
