This element simulates a wiggler or undulator.  There are two aspects
to the simulation: the effect on radiation integrals and the vertical
focusing.  Both are included as of release 15.2 of elegant.  Also, we
include the half-strength end poles, but only for the radiation
integral calculation.  These half-poles ensure that the dispersion
contribution of the wiggler matches back to zero at the end of the
device.  For the focusing, we assume all the poles are full strength.

The radiation integrals are computed by summing the contributions for
a series of half-poles.  The integrals for a single half-pole were
computed analytically using Mathematica, using a sinusoidal field
variation.  The horizontal beta function and dispersion are propogated
correctly for these computations.  Of course, the beta function
propagates as in a drift space.

The vertical focusing is implemented as a distributed quadrupole-like
term (affecting ony the vertical, unlike a true quadrupole).  The
strength of the quadrupole is (see Wiedemann, {\em Particle Accelerator
Physics II}, section 2.3.2)
\begin{equation}
K_1 = \frac{1}{2\rho^2},
\end{equation}
where $\rho$ is the bending radius at the center of a pole.  The
undulator is focusing in the vertical plane.

The wiggler field strength may be specified either as a peak bending 
radius $\rho$ (RADIUS parameter) or using the dimensionless strength parameter
K (K parameter).  These are related by
\begin{equation}
K = \frac{\gamma \lambda_u}{2 \pi \rho},
\end{equation}
where $\gamma$ is the relativistic factor for the beam and $\lambda_u$ is
the period length.

The number of poles should be an odd integer.  If it is not, it is rounded up
to the nearest odd integer.
